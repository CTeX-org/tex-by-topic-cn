% -*- coding: utf-8 -*-
\documentclass{book}

% -*- coding: utf-8 -*-

\usepackage[b5paper,text={5in,8in},centering]{geometry}
\usepackage{amsmath}
\usepackage{amssymb}
\usepackage[heading = false, scheme = plain]{ctex}
% \usepackage[CJKchecksingle]{xeCJK}
\setmainfont[Mapping=tex-text]{TeX Gyre Schola}
%\setsansfont{URW Gothic L Book}
%\setmonofont{Nimbus Mono L}
% \setCJKmainfont[BoldFont=FandolHei,ItalicFont=FandolKai]{FandolSong}
% \setCJKsansfont{FandolHei}
% \setCJKmonofont{FandolFang}
\xeCJKsetup{PunctStyle = kaiming}

\linespread{1.25}
\setlength{\parindent}{2em}
\setlength{\parskip}{0.5ex}
\usepackage{indentfirst}

\usepackage{xcolor}
\definecolor{myblue}{rgb}{0,0.2,0.6}

\usepackage{titlesec}
\titleformat{\chapter}
    {\normalfont\Huge\sffamily\color{myblue}}
    {第\thechapter 章}
    {1em}
    {}
%\titlespacing{\chapter}{0pt}{50pt}{40pt}
\titleformat{\section}
    {\normalfont\Large\sffamily\color{myblue}}
    {\thesection}
    {1em}
    {}
%\titlespacing{\section}{0pt}{3.5ex plus 1ex minus .2ex}{2.3ex plus .2ex}
\titleformat{\subsection}
    {\normalfont\large\sffamily\color{myblue}}
    {\thesubsection}
    {1em}
    {}
%\titlespacing{\subsection}{0pt}{3.25ex plus 1ex minus .2ex}{1.5ex plus .2ex}
%
\newpagestyle{special}[\small\sffamily]{
  \headrule
  \sethead[\usepage][][\chaptertitle]
  {\chaptertitle}{}{\usepage}}
\newpagestyle{main}[\small\sffamily]{
  \headrule
  \sethead[\usepage][][第\thechapter 章\quad\chaptertitle]
  {\thesection\quad\sectiontitle}{}{\usepage}}

\usepackage{titletoc}
%\setcounter{tocdepth}{1}
%\titlecontents{标题层次}[左间距]{上间距和整体格式}{标题序号}{标题内容}{指引线和页码}[下间距]
\titlecontents{chapter}[1.5em]{\vspace{.5em}\bfseries\sffamily}{\color{myblue}\contentslabel{1.5em}}{}
    {\titlerule*[20pt]{$\cdot$}\contentspage}[]
\titlecontents{section}[4.5em]{\sffamily}{\color{myblue}\contentslabel{3em}}{}
    {\titlerule*[20pt]{$\cdot$}\contentspage}[]
%\titlecontents{subsection}[8.5em]{\sffamily}{\contentslabel{4em}}{}
%    {\titlerule*[20pt]{$\cdot$}\contentspage}

\usepackage{enumitem}
\setlist{topsep=2pt,itemsep=2pt,parsep=1pt,leftmargin=\parindent}

\usepackage{fancyvrb}
\DefineVerbatimEnvironment{verbatim}{Verbatim}
  {xleftmargin=2em,baselinestretch=1,formatcom=\color{teal}\upshape}

\usepackage{etoolbox}
\makeatletter
\preto{\FV@ListVSpace}{\topsep=2pt \partopsep=0pt }
\makeatother

\usepackage[colorlinks,plainpages,pagebackref]{hyperref}
\hypersetup{
   pdfstartview={FitH},
   citecolor=teal,
   linkcolor=myblue,
   urlcolor=black,
   bookmarksnumbered
}

\usepackage{comment,makeidx,multicol}

%\usepackage{german}
%% german
%\righthyphenmin=3
%\mdqoff
%\captionsenglish
\usepackage[english]{babel}
{\catcode`"=13 \gdef"#1{\ifx#1"\discretionary{}{}{}\fi\relax}}
\def\mdqon{\catcode`"=13\relax}
\def\mdqoff{\catcode`"=12\relax}
\makeindex
\hyphenation{ex-em-pli-fies}

\newdimen\tempdima \newdimen\tempdimb

% these are fine
\def\handbreak{\\ \message{^^JManual break!!!!^^J}}
\def\nl{\protect\\}\def\n#1{{\tt #1}}
\protected\def\cs#1{\texttt{\textbackslash#1}}
\pdfstringdefDisableCommands{\def\cs#1{\textbackslash#1}}
\let\csc\cs
\def\lb{{\tt\char`\{}}\def\rb{{\tt\char`\}}}
\def\gr#1{\texorpdfstring{$\langle$#1$\rangle$}{<#1>}} %\def\gr#1{$\langle$#1$\rangle$}
\def\marg#1{{\tt \{}#1{\tt \}}}
\def\oarg#1{{\tt [}#1{\tt }}
\def\key#1{{\tt#1}}
\def\alt{}\def\altt{}%this way in manstijl
\def\ldash{\unskip\ ——\nobreak\ \ignorespaces}
\def\rdash{\unskip\nobreak\ ——\ \ignorespaces}
% check these
\def\hex{{\tt"}}
\def\ascii{{\sc ascii}}
\def\ebcdic{{\sc ebcdic}}
\def\IniTeX{Ini\TeX}\def\LamsTeX{LAMS\TeX}\def\VirTeX{Vir\TeX}
\def\AmsTeX{Ams\TeX}
\def\TeXbook{the \TeX\ book}\def\web{{\sc web}}
% needs major thinking
\newenvironment{myquote}{\list{}{%
    \topsep=2pt \partopsep=0pt%
    \leftmargin=\parindent \rightmargin=\parindent
    }\item[]}{\endlist}
\newenvironment{disp}{\begin{myquote}}{\end{myquote}}
\newenvironment{Disp}{\begin{myquote}}{\end{myquote}}
\newenvironment{tdisp}{\begin{myquote}}{\end{myquote}}
\newenvironment{example}{\begin{myquote}\noindent\itshape 例子:}{\end{myquote}}
\newenvironment{inventory}{\begin{description}\raggedright}{\end{description}}
\newenvironment{glossinventory}{\begin{description}}{\end{description}}
\def\gram#1{\gr{#1}}%???
\def\meta{\gr}% alias
%
% index
%
\def\indexterm#1{\emph{#1}\index{#1}}
\def\indextermsub#1#2{\emph{#1 #2}\index{#1!#2}}
\def\indextermbus#1#2{\emph{#1 #2}\index{#2!#1}}
\def\cindextermsub#1#2{\emph{#1#2}\index{#1!#1#2}}
\def\cindextermbus#1#2{\emph{#1#2}\index{#2!#1#2}}
\def\term#1\par{\index{#1}}
\def\howto#1\par{}
\def\cstoidx#1\par{\index{#1@\cs{#1}@}}
\def\thecstoidx#1\par{\index{#1@\protect\csname #1\endcsname}}
\def\thecstoidxsub#1#2{\index{#1, #2@\protect\csname #1\endcsname, #2}\ignorespaces}
\def\csterm#1\par{\cstoidx #1\par\cs{#1}}
\def\csidx#1{\cstoidx #1\par\cs{#1}}

\def\tmc{\tracingmacros=2 \tracingcommands\tracingmacros}

%%%%%%%%%%%%%%%%%%%
\makeatletter
\def\snugbox{\hbox\bgroup\setbox\z@\vbox\bgroup
    \leftskip\z@
    \bgroup\aftergroup\make@snug
    \let\next=}
\def\make@snug{\par\sn@gify\egroup \box\z@\egroup}
\def\sn@gify
   {\skip\z@=\lastskip \unskip
    \advance\skip\z@\lastskip \unskip
    \unpenalty
    \setbox\z@\lastbox
    \ifvoid\z@ \nointerlineskip \else {\sn@gify} \fi
    \hbox{\unhbox\z@}\nointerlineskip
    \vskip\skip\z@
    }

\newdimen\fbh \fbh=60pt % dimension for easy scaling:
\newdimen\fbw \fbw=60pt % height and width of character box

\newdimen\dh \newdimen\dw % height and width of current character box
\newdimen\lh % height of previous character box
\newdimen\lw \lw=.4pt % line weight, instead of default .4pt

\def\hdotfill{\noindent
    \leaders\hbox{\vrule width 1pt height\lw
                  \kern4pt
                  \vrule width.5pt height\lw}\hfill\hbox{}
    \par}
\def\hlinefill{\noindent
    \leaders\hbox{\vrule width 5.5pt height\lw         }\hfill\hbox{}
    \par}
\def\stippel{$\qquad\qquad\qquad\qquad$}
\makeatother
%%%%%%%%%%%%%%%%%%%

%\def\SansSerif{\Typeface:macHelvetica }
%\def\SerifFont{\Typeface:macTimes }
%\def\SansSerif{\Typeface:bsGillSans }
%\def\SerifFont{\Typeface:bsBaskerville }
\let\SansSerif\relax \def\italic{\it}
\let\SerifFont\relax \def\MainFont{\rm}
\let\SansSerif\relax
\let\SerifFont\relax
\let\PopIndentLevel\relax \let\PushIndentLevel\relax
\let\ToVerso\relax \let\ToRecto\relax

%\def\stop@command@suffix{stop}
%\let\PopListLevel\PopIndentLevel
%\let\FlushRight\relax
%\let\flushright\FlushRight
%\let\SetListIndent\LevelIndent
%\def\awp{\ifhmode\vadjust{\penalty-10000 }\else
%    \penalty-10000 \fi}
\let\awp\relax
\let\PopIndentLevel\relax \let\PopListLevel\relax

\showboxdepth=-1

%\input figs
\def\endofchapter{\vfill\noindent}

\newcommand{\liamfnote}[1]{\footnote{译注(Liam0205):#1}}
\newcommand{\cstate}[1]{状态 \textit{#1}}

\setcounter{chapter}{6}

\begin{document}

%\chapter{Numbers}\label{number}
\chapter{数字}\label{number}

%In this chapter integers and their
%denotations will be treated,
%the conversions that are possible either way,
%allocation and use of \cs{count} registers, and
%arithmetic with integers.
本章将讨论整数和罗马数字等不同的表示形式,以及两者之间的转换。还将讨论\cs{count}寄存器的分配和使用,以及整数的算术运算。

%\label{cschap:number}\label{cschap:romannumeral}\label{cschap:count}\label{cschap:countdef}\label{cschap:newcount}\label{cschap:advance}\label{cschap:multiply}\label{cschap:divide}
%\begin{inventory}
%\item [\cs{number}]
%      Convert a \gr{number} to decimal representation.
\label{cschap:number}\label{cschap:romannumeral}\label{cschap:count}\label{cschap:countdef}\label{cschap:newcount}\label{cschap:advance}\label{cschap:multiply}\label{cschap:divide}
\begin{inventory}
\item [\cs{number}]
      %Convert a \gr{number} to decimal representation.
将一个\gr{number}转换为十进制数。

%\item [\cs{romannumeral}]
%      Convert a positive \gr{number} to lowercase roman representation.
\item [\cs{romannumeral}]
      %Convert a positive \gr{number} to lowercase roman representation.
      
      将一个正的\gr{number}转换为小写罗马数字。

%\item [\cs{ifnum}]
%      Test relations between numbers.
\item [\cs{ifnum}]
      %Test relations between numbers.
      用于测试两个数字间的大小关系

%\item [\cs{ifodd}]
%      Test whether a number is odd.
\item [\cs{ifodd}]
      %Test whether a number is odd.
      用于测试数字是否为奇数

%\item [\cs{ifcase}]
%      Enumerated case statement.
\item [\cs{ifcase}]
      %Enumerated case statement.
      用于枚举条件声明


%\item [\cs{count}]
%      Prefix for count registers.
\item [\cs{count}]
      %Prefix for count registers.
      计数寄存器的前缀

%\item [\cs{countdef}]
%      Define a control sequence to be a synonym for
%      a~\cs{count} register.
\item [\cs{countdef}]
      %Define a control sequence to be a synonym for
      %a~\cs{count} register.
      为一个\cs{count}寄存器定义一个同义的控制序列

%\item [\cs{newcount}]
%      Allocate an unused \cs{count} register.
\item [\cs{newcount}]
      %Allocate an unused \cs{count} register.
      分配一个未使用的\cs{count}寄存器

%\item [\cs{advance}]
%      Arithmetic command to add to or subtract from
%      a~\gr{numeric variable}.
\item [\cs{advance}]
      %Arithmetic command to add to or subtract from
      %a~\gr{numeric variable}.
      算术运算命令用于对\gr{numeric variable}做加法和减法

%\item [\cs{multiply}]
%      Arithmetic command to multiply a \gr{numeric variable}.
\item [\cs{multiply}]
      %Arithmetic command to multiply a \gr{numeric variable}.
      算术运算命令用于对\gr{numeric variable}做乘法

%\item [\cs{divide}]
%      Arithmetic command to divide a \gr{numeric variable}.
\item [\cs{divide}]
      %Arithmetic command to divide a \gr{numeric variable}.
      算术运算命令用于对\gr{numeric variable}做除法

%\end{inventory}
\end{inventory}


%\section{Numbers and \gr{number}s}
\section{数字和 \gr{number}s}

%An important part of the grammar of \TeX\ is the rigorous definition
%of a \gr{number}\index{numbers}, the syntactic entity that
%\TeX\ expects when semantically an \indexterm{integer} is
%expected. This definition will take the largest part of this
%chapter. Towards the end, \cs{count} registers, arithmetic, and tests
%for numbers are discussed.

对\gr{number}\index{numbers}的严格定义是\TeX\ 语法的一个重要部分,当从语义上需要一个
\indexterm{整数}时,\TeX\ 要求这样的语法实体。这一定义将占据本章的最大一部分内容。本章最后将讨论\cs{count}寄存器、算术运算和数字间的关系判断。 

%For clarity of discussion a distinction will be made
%here between integers and numbers,
%but note that a \gr{number} can be both
%an `integer' and a `number'.
%`Integer'  will be taken to denote a mathematical number:
%a~quantity that can be added or multiplied.
%`Number' will be taken to refer to the printed representation
%of an integer: a string of digits, in other words.

为清楚起见,讨论之前先在此强调一下整数(integer)和数字(number)间的区别
\footnote{译者注:这里number本质指的是整数的打印形式,暂译为数字。而integer作为数学上的整数很好理解。而\gr{number}是本文中表示整数或数字的形式,也可以称为表示整数或数字的量}。
但要注意
\gr{number}既可以是整数也可以是数字。“整数”代表数学上的数字,一个可以做算术运算的量。
而“数字”用来指代打印一个整数的表示形式: 换句话说就是一个数字构造的字符串。

%\section{Integers}
\section{整数}

%Quite a few different sorts of objects can function
%as integers in \TeX. In this section they will all
%be treated, accompanied by the relevant lines from
%the grammar of \TeX.

\TeX 中有相当多的对象用来表示整数。本节将讨论所有这些对象,及其相关的 \TeX 语法内容。

%First of all, an integer can be positive or negative:
%\begin{disp}\gr{number} $\longrightarrow$
%\gr{optional signs}\gr{unsigned number}\nl
%\gr{optional signs} $\longrightarrow$ \gr{optional spaces}\nl
%\indent $|$ \gr{optional signs}\gr{plus or minus}\gr{optional spaces}
%\end{disp}
First of all, an integer can be positive or negative:
\begin{disp}\gr{number} $\longrightarrow$
\gr{optional signs}\gr{unsigned number}\nl
\gr{optional signs} $\longrightarrow$ \gr{optional spaces}\nl
\indent $|$ \gr{optional signs}\gr{plus or minus}\gr{optional spaces}
\end{disp}

首先,一个整数可正可负:
\begin{disp}\gr{number} $\longrightarrow$
\gr{optional signs}\gr{unsigned number}\nl
\gr{optional signs} $\longrightarrow$ \gr{optional spaces}\nl
\indent $|$ \gr{optional signs}\gr{plus or minus}\gr{optional spaces}
\end{disp}

%A first possibility for an unsigned integer is a string of digits
%in decimal, octal, or hexadecimal notation.
%Together with the alphabetic constants these will be named
%here \gr{integer denotation}.
%Another possibility for an integer is an
%internal integer quantity, an \gr{internal integer};
%together with the denotations these form the
%\gr{normal integer}s.
%Lastly an integer can be a \gr{coerced integer}:
%an internal \gr{dimen} or \gr{glue}
%quantity that is converted to an integer value.
%\begin{disp}\gr{unsigned number} $\longrightarrow$ \gr{normal integer}
%$|$ \gr{coerced integer}\nl
%\gr{normal integer} $\longrightarrow$ \gr{integer denotation}
%$|$ \gr{internal integer}\nl
%\gr{coerced integer} $\longrightarrow$ \gr{internal dimen}
%$|$ \gr{internal glue}\end{disp}
%All of these possibilities will be treated in sequence.

一个无正负号的整数的第一种可能是以十进制、八进制或十六进制数字表示的字符串。连同字母常量,这些可以称为整数记号(\gr{integer denotation})。整数的另一种可能是内部整数量,
即\gr{internal integer}; 加上它的表示形式称为规范整数(\gr{normal integer}s)。
最后整数可以是一个强制整数(\gr{coerced integer}):内部的尺寸(\gr{dimen}) 或 粘连(\gr{glue})量总是转换为一个整数值。

\begin{disp}\gr{unsigned number} $\longrightarrow$ \gr{normal integer}
$|$ \gr{coerced integer}\nl
\gr{normal integer} $\longrightarrow$ \gr{integer denotation}
$|$ \gr{internal integer}\nl
\gr{coerced integer} $\longrightarrow$ \gr{internal dimen}
$|$ \gr{internal glue}\end{disp}

所有这些形式下面将一一介绍:

%%\spoint[int:denotation] Denotations: integers
%\subsection{Denotations: integers}
%\label{int:denotation}
%\spoint[int:denotation] Denotations: integers
\subsection{表示整数的记号}
\label{int:denotation}

%Anything that looks like a number
%can be used as a \gr{number}: thus \verb-42- is a number.
%However, bases other than decimal can also be used:
%\begin{verbatim}
%'123
%\end{verbatim}
%is the octal notation for $1\times8^2+2\times8^1+3\times8^0=83$,
%and
%\begin{verbatim}
%"123
%\end{verbatim}
%is the hexadecimal notation
%for $1\times16^2+2\times16^1+3\times16^0=291$.
%\begin{disp}\gr{integer denotation} $\longrightarrow$
%\gr{integer constant}\gr{one optional space} \nl
%\indent $|$ \n{\char`\'}\gr{octal constant}\gr{one optional space}\nl
%\indent $|$ \n{\char`\"}\gr{hexadecimal constant}\gr{one optional space}
%\end{disp}
%The octal digits are \n0--\n7; a~digit \n8 or~\n9 following an
%octal denotation is not part of the number:
%after
%\begin{verbatim}
%\count0='078
%\end{verbatim}
%the \cs{count0} will have the value~7, and the
%digit~\n8 is typeset.

任何看起来像平常数字的对象都可以当作\gr{number}使用:因此\verb-42- 是一个数字。
而且,也可以使用非十进制的数字符号:

\begin{verbatim}
'123
\end{verbatim}

是一个八进制数,其值为$1\times8^2+2\times8^1+3\times8^0=83$,而

\begin{verbatim}
"123
\end{verbatim}

是一个十六进制数,其值为$1\times16^2+2\times16^1+3\times16^0=291$.

\begin{disp}\gr{integer denotation} $\longrightarrow$
\gr{integer constant}\gr{one optional space} \nl
\indent $|$ \n{\char`\'}\gr{octal constant}\gr{one optional space}\nl
\indent $|$ \n{\char`\"}\gr{hexadecimal constant}\gr{one optional space}
\end{disp}

八进制数的数字包括 \n0--\n7; 
八进制数后面跟着的数字\n8 或 \n9 就不再是该八进制数的一部分。比如:

\begin{verbatim}
\count0='078
\end{verbatim}
其中 \cs{count0} 的值为 7,而数字~\n8 只是一个排印的数字。

%The hexadecimal digits are \n0--\n9, \n A--\n F, where the \n A--\n F
%can have category code 11\index{category!11}
%or~12\index{category!12}. The latter has a somewhat far-fetched
%justification: the characters resulting from a \cs{string} operation
%have category code~12.  Lowercase \n a--\n f are not hexadecimal
%digits, although (in \TeX3) they are used for hexadecimal notation in
%the `circumflex method' for accessing all character codes (see
%Chapter~\ref{char}).

十六进制数的数字包括\n0--\n9, \n A--\n F, 其中\n A--\n F 的字符分类码为11 \index{category!11} 或 12\index{category!12}。后者有一个有些勉强的合理解释:
来自\cs{string}操作结果中字符的分类码为 12。小写的\n a--\n f 不是十六进制数字,尽管(在 \TeX3)中,他们也可用作十六进制表示,用于访问所有字符编码的“取码方法(circumflex method)”中(见第~\ref{char}~章)。

%%\spoint Denotations: characters
%\subsection{Denotations: characters}
%\spoint Denotations: characters
\subsection{表示字符的记号}

%A character token is a pair consisting of a character code,
%which is a~number in the range 0--255,
%and a category code. Both of these codes are accessible,
%and can be used as a \gr{number}.
一个字符记号是由范围在 0--255 的字符编码和分类码组成的一对数字。
两者均可访问,可以用作\gr{number}。

%The character code of a character token, or of a single letter
%control sequence, is accessible through the left quote command:
%both \verb-`a- and~\verb-`\a- denote the character code of~{\tt a},
%which can be used as an integer.
%\begin{disp}\gr{integer denotation} $\longrightarrow$
%\n{\char`\`}\gr{character token}\gr{one optional space}\end{disp}
字符记号或一个单字母控制序列的字符编码,可以通过左引号命令访问:
\verb-`a- 和 \verb-`\a- 均代表 {\tt a} 的字符编码,可以用作整数。

\begin{disp}\gr{integer denotation} $\longrightarrow$
\n{\char`\`}\gr{character token}\gr{one optional space}\end{disp}

%In order to emphasize that accessing the character code is
%in a sense using a denotation, the syntax of \TeX\ allows
%an optional space after such a `character constant'.
%The left quote must have category~12\index{category!12}.

为了强调:访问字符编码某种程度上是在使用记法,\TeX\ 语法允许在这一字符常量后面带有可选的空格。反引号的分类码必须是~12\index{category!12}。

%%\spoint Internal integers
%\subsection{Internal integers}
%\spoint Internal integers
\subsection{内部整数}

%The class of \gr{internal integers} can
%be split into five parts.
%The \gr{codename}s and \gr{special integer}s
%will be treated separately below; furthermore, there are the following.
内部整数(\gr{internal integers})可以分为五类。包括如下,其中
编码名(\gr{codename})和特殊整数(\gr{special integer})将在后面单独介绍。

%\begin{itemize} \item The contents of \cs{count} registers;
%either explicitly used by writing for instance \cs{count23},
%or by referring to such a register by means of a
%control sequence
%that was defined by \cs{countdef}:
%after
%\begin{verbatim}
%\countdef\MyCount=23
%\end{verbatim}
%\cs{MyCount} is called a
%\gr{countdef token}, and it is fully equivalent to \cs{count23}.
\begin{itemize} \item \cs{count} 寄存器的内容;
即可以显式的使用,比如以\cs{count23}的方式,
也可以通过指代该寄存器的控制序列(通过\cs{countdef}定义)使用:
\begin{verbatim}
\countdef\MyCount=23
\end{verbatim}
\cs{MyCount}称为一个\gr{countdef token},完全等价于\cs{count23}。

%\item All parameters of \TeX\ that hold integer values;
%this includes obvious ones such as \cs{linepenalty}, but
%also parameters such as
%\cs{hyphenchar}\gr{font} and \cs{parshape}
%(if a paragraph shape has been defined for $n$ lines,
%using \cs{parshape} in the context of a \gr{number}
%will yield this value of~$n$).
\item 所有保存整数值的\TeX\ 参数,包括一些显而易见的参数如\cs{linepenalty},但还有其它一些参数比如\cs{hyphenchar} 、\gr{font} 和 \cs{parshape}(如果段落形状共定义$n$ 行,在\gr{number}上下文使用\cs{parshape}将得到值 $n$)等。

%\item\label{num:chardef} Tokens defined by \cs{chardef} or \cs{mathchardef}.
%After
%\begin{verbatim}
%\chardef\foo=74
%\end{verbatim}
%the control sequence \cs{foo}
%can be used on its own to mean \cs{char74}, but in a context
%where a \gr{number} is wanted it can be used to denote~74:
%\begin{verbatim}
%\count\foo
%\end{verbatim}
%is equivalent to \verb=\count74=.
%This fact is
%exploited in the allocation routines for registers (see
%Chapter~\ref{alloc}).
\item\label{num:chardef} 由\cs{chardef} or \cs{mathchardef}定义的记号,比如定义:

\begin{verbatim}
\chardef\foo=74
\end{verbatim}

控制序列\cs{foo}可以用来表示原义\cs{char74},但在一个\gr{number}上下文中,也可以用于表示74:

\begin{verbatim}
\count\foo
\end{verbatim}

等价于\verb=\count74=。其原理详见寄存器的分配程序(见第 \ref{alloc} 章)。

%A control sequence thus defined by \cs{chardef} is called a
%\gr{chardef token}; if it is defined by \cs{mathchardef} it
%is called a \gr{mathchardef token}.

一个由\cs{chardef}定义的控制序列称为\gr{chardef token};
而由\cs{mathchardef}定义的控制序列称为\gr{mathchardef token}。

%\end{itemize}
\end{itemize}

%Here is the full list:
%\begin{disp}\gr{internal integer} $\longrightarrow$
%\gr{integer parameter} \nl
%\indent $|$ \gr{special integer} $|$ \cs{lastpenalty}\nl
%\indent $|$ \gr{countdef token} $|$ \cs{count}\gr{8-bit number}\nl
%\indent $|$ \gr{chardef token} $|$ \gr{mathchardef token}\nl
%\indent $|$ \gr{codename}\gr{8-bit number}\nl
%\indent $|$ \cs{hyphenchar}\gr{font} $|$ \cs{skewchar}\gr{font}
%$|$ \cs{parshape}\nl
%\indent $|$ \cs{inputlineno} $|$ \cs{badness}\nl
%\gr{integer parameter} $\longrightarrow$\vadjust{\nobreak}
%$|$ \cs{adjdemerits} $|$ \cs{binoppenalty}\nl
%\indent $|$ \cs{brokenpenalty} $|$ \cs{clubpenalty} $|$ \cs{day}%
%\nl
%\indent $|$ \cs{defaulthyphenchar} $|$ \cs{defaultskewchar} \nl
%\indent $|$ \cs{delimiterfactor} $|$ \cs{displaywidowpenalty} \nl
%\indent $|$ \cs{doublehyphendemerits} $|$ \cs{endlinechar}
%        $|$ \cs{escapechar}\nl
%\indent $|$ \cs{exhypenpenalty} $|$ \cs{fam} $|$ \cs{finalhyphendemerits}\nl
%\indent $|$ \cs{floatingpenalty} $|$ \cs{globaldefs} $|$ \cs{hangafter}\nl
%\indent $|$ \cs{hbadness} $|$ \cs{hyphenpenalty}
%        $|$ \cs{interlinepenalty}\nl
%\indent $|$ \cs{linepenalty} $|$ \cs{looseness} $|$ \cs{mag}\nl
%\indent $|$ \cs{maxdeadcycles} $|$ \cs{month} \nl
%\indent $|$ \cs{newlinechar} $|$ \cs{outputpenalty} $|$ \cs{pausing}\nl
%\indent $|$ \cs{postdisplaypenalty} $|$ \cs{predisplaypenalty}\nl
%\indent $|$ \cs{pretolerance} $|$ \cs{relpenalty} $|$ \cs{showboxbreadth}\nl
%\indent $|$ \cs{showboxdepth} $|$ \cs{time} $|$ \cs{tolerance}\nl
%\indent $|$ \cs{tracingcommands} $|$ \cs{tracinglostchars}
%        $|$ \cs{tracingmacros}\nl
%\indent $|$ \cs{tracingonline} $|$ \cs{tracingoutput}
%        $|$ \cs{tracingpages}\nl
%\indent $|$ \cs{tracingparagraphs} $|$ \cs{tracingrestores}
%        $|$ \cs{tracingstats}\nl
%\indent $|$ \cs{uchyph} $|$ \cs{vbadness} $|$ \cs{widowpenalty}
%        $|$ \cs{year}
%\end{disp}
下面是完整的列表:
\begin{disp}\gr{internal integer} $\longrightarrow$
\gr{integer parameter} \nl
\indent $|$ \gr{special integer} $|$ \cs{lastpenalty}\nl
\indent $|$ \gr{countdef token} $|$ \cs{count}\gr{8-bit number}\nl
\indent $|$ \gr{chardef token} $|$ \gr{mathchardef token}\nl
\indent $|$ \gr{codename}\gr{8-bit number}\nl
\indent $|$ \cs{hyphenchar}\gr{font} $|$ \cs{skewchar}\gr{font}
$|$ \cs{parshape}\nl
\indent $|$ \cs{inputlineno} $|$ \cs{badness}\nl
\gr{integer parameter} $\longrightarrow$\vadjust{\nobreak}
$|$ \cs{adjdemerits} $|$ \cs{binoppenalty}\nl
\indent $|$ \cs{brokenpenalty} $|$ \cs{clubpenalty} $|$ \cs{day}%
\nl
\indent $|$ \cs{defaulthyphenchar} $|$ \cs{defaultskewchar} \nl
\indent $|$ \cs{delimiterfactor} $|$ \cs{displaywidowpenalty} \nl
\indent $|$ \cs{doublehyphendemerits} $|$ \cs{endlinechar}
        $|$ \cs{escapechar}\nl
\indent $|$ \cs{exhypenpenalty} $|$ \cs{fam} $|$ \cs{finalhyphendemerits}\nl
\indent $|$ \cs{floatingpenalty} $|$ \cs{globaldefs} $|$ \cs{hangafter}\nl
\indent $|$ \cs{hbadness} $|$ \cs{hyphenpenalty}
        $|$ \cs{interlinepenalty}\nl
\indent $|$ \cs{linepenalty} $|$ \cs{looseness} $|$ \cs{mag}\nl
\indent $|$ \cs{maxdeadcycles} $|$ \cs{month} \nl
\indent $|$ \cs{newlinechar} $|$ \cs{outputpenalty} $|$ \cs{pausing}\nl
\indent $|$ \cs{postdisplaypenalty} $|$ \cs{predisplaypenalty}\nl
\indent $|$ \cs{pretolerance} $|$ \cs{relpenalty} $|$ \cs{showboxbreadth}\nl
\indent $|$ \cs{showboxdepth} $|$ \cs{time} $|$ \cs{tolerance}\nl
\indent $|$ \cs{tracingcommands} $|$ \cs{tracinglostchars}
        $|$ \cs{tracingmacros}\nl
\indent $|$ \cs{tracingonline} $|$ \cs{tracingoutput}
        $|$ \cs{tracingpages}\nl
\indent $|$ \cs{tracingparagraphs} $|$ \cs{tracingrestores}
        $|$ \cs{tracingstats}\nl
\indent $|$ \cs{uchyph} $|$ \cs{vbadness} $|$ \cs{widowpenalty}
        $|$ \cs{year}
\end{disp}

%Any internal integer can function as an \gr{internal unit},
%which  \ldash preceded by \gr{optional spaces} \rdash
%can serve as a \gr{unit of measure}.
%Examples of this are given in Chapter~\ref{glue}.

任何内部整数可以用作内部单位(\gr{internal unit}),\gr{internal unit}(可以带前置可选空格)可以作为一个度量单位(\gr{unit of measure})。相应的示例见第 \ref{glue} 章。

%%\spoint Internal integers: other codes of a character
%\subsection{Internal integers: other codes of a character}
%\spoint Internal integers: other codes of a character
\subsection{内部整数: 字符的其它相关码}

%The \cs{catcode} command
%(which was  described in Chapter~\ref{mouth})
%is a \gr{codename}, and like the other code names
%it can be used as an integer.
%\begin{disp}\gr{codename} $\longrightarrow$ \cs{catcode} $|$ \cs{mathcode}
%$|$ \cs{uccode} $|$ \cs{lccode}\nl \indent $|$ \cs{sfcode} $|$ \cs{delcode}
%\end{disp}
%A~\gr{codename} has to be followed by an \gr{8-bit number}.
\cs{catcode} (描述见第 \ref{mouth} 章)命令是一个编码名(\gr{codename}),与其它编码名类似,可以用作一个整数。

\begin{disp}\gr{codename} $\longrightarrow$ \cs{catcode} $|$ \cs{mathcode}
$|$ \cs{uccode} $|$ \cs{lccode}\nl \indent $|$ \cs{sfcode} $|$ \cs{delcode}
\end{disp}

一个编码名(\gr{codename})后必须跟一个8位数字(\gr{8-bit number})。


%Uppercase and lowercase codes were treated in Chapter~\ref{char};
%the \cs{sfcode} is treated
%in Chapter~\ref{space};
%the \cs{mathcode} and~\cs{delcode} are treated in
%Chapter~\ref{mathchar}.

大写和小写编码见第~\ref{char}~章; \cs{sfcode}讨论见第~\ref{space}~章; 
\cs{mathcode} 和 \cs{delcode} 讨论见第~\ref{mathchar}~章。


%%\spoint[special:int:list] \gr{special integer}
%\subsection{\gr{special integer}}
%\label{special:int:list}
%\spoint[special:int:list] \gr{special integer}
\subsection{特殊整数(\gr{special integer})}
\label{special:int:list}

%One of the subclasses of the internal integers is
%that of the special integers.
%\begin{disp}\gr{special integer} $\longrightarrow$
%\cs{spacefactor} $|$ \cs{prevgraf}\nl
%\indent $|$ \cs{deadcycles} $|$ \cs{insertpenalties}
%\end{disp}
%An assignment to any of these is called an \gr{intimate
%assignment}, and is automatically global
%(see Chapter~\ref{group}).

内部整数的一个子类是特殊整数:
\begin{disp}\gr{special integer} $\longrightarrow$
\cs{spacefactor} $|$ \cs{prevgraf}\nl
\indent $|$ \cs{deadcycles} $|$ \cs{insertpenalties}
\end{disp}

对这些量赋值称为一个\gr{intimate assignment},是自动全局化的(见第 \ref{group} 章)。

%%\spoint Other internal quantities: coersion to integer
%\subsection{Other internal quantities: coersion to integer}
%\spoint Other internal quantities: coersion to integer
\subsection{其它内部量:强制转换的整数}

%\TeX\ provides a conversion between dimensions and integers:
%if an integer is expected, a \gr{dimen} or \gr{glue} used
%in that context is converted by taking its
%(natural) size
%in scaled points\index{point!scaled}.
%However, only \gr{internal dimen}s and \gr{internal glue}
%can be used this way: no dimension or glue denotations
%can be coerced to integers.

\TeX\ 提供了尺寸和整数间的转换方法。在一个需要整数的上下文中,使用的\gr{dimen} 或 \gr{glue}会转换为它在定标点中(自然)大小\index{point!scaled}。然而,只有\gr{internal dimen}s 和 \gr{internal glue}可以用于这种方式,其它尺寸和粘连标记不能强制转换。

%%\spoint Trailing spaces
%\subsection{Trailing spaces}
%\spoint Trailing spaces
\subsection{拖尾的空格}

%The syntax of \TeX\ defines integer denotations (decimal,
%octal, and hexadecimal) and `back-quoted' character tokens
%to be followed by \gr{one optional space}. This means that
%\TeX\ reads the token after the number, absorbing it
%if it was a space token, and backing up if it was not.

\TeX\ 定义整数记号(十进制、八进制、十六进制)的语法和反引号字符记号后可以跟着一个可选空格(optional space)。这意味着 \TeX\ 读取数字后面的记号,当它是空格时则吸收它,若不是空额,则将其返回到输入流中。

%Because \TeX's input processor goes into the state `skipping spaces'
%after it has seen one space token, this
%scanning behaviour implies that
%integer denotations can be followed by
%arbitrarily many space characters in the input.
%Also, a line end is admissible.
%However, only one space token is allowed.


因为 \TeX 的输入处理器,看到空格记号会进入跳过空格(skipping spaces)的状态,这一扫描机制表明整数记号后可以跟随任意多个空格字符,而且一个行末符号也是可以的。但空格记号只能允许一个。

%%\point Numbers
%\section{Numbers}
%\point Numbers
\section{数字}

%\TeX\ can perform an implicit \indextermsub{number}{conversion} from a string
%\cstoidx number\par\cstoidx romannumeral\par
%of digits to an integer. Conversion from a representation
%in decimal, octal, or hexadecimal notation was
%treated above. The conversion the other way,
%from an \gr{internal integer} to a printed representation,
%has to be performed explicitly.
%\TeX\ provides two conversion routines,
%\cs{number}, to decimal, and \cs{romannumeral} to
%\indexterm{roman numerals}.
%The command \cs{number} is equivalent to \cs{the}
%when followed by an internal integer.
%These commands are performed in the expansion processor of \TeX, that is,
%they are expanded whenever expansion has not been inhibited.

\TeX\ 可以执行一个隐式的转换\indextermsub{number}{conversion} 从一个数字组成的字符串到一个整数\cstoidx number\par\cstoidx romannumeral\par。以十进制、八进制、十六进制符号表示的转换前面已经介绍过。相反的,从整数到其它数字形式的转换则是显式的。\TeX\ 提供了两种机制包括:\cs{number}转换为十进制数字字符串,
\cs{romannumeral}转换为罗马数字字符串\indexterm{roman numerals}。
这些命令是在 \TeX 的展开处理器中执行,也就是说无论是否抑制展开,它们都会展开。


%Both commands
%yield a string of tokens with category code~12\index{category!12};
%their argument is a~\gr{number}.
%Thus \verb-\romannumeral51-, \verb-\romannumeral\year-,
%and~\verb-\number\linepenalty- are valid, and so is \verb-\number13-.
%Applying \cs{number} to a denotation has some uses:
%it removes leading zeros and superfluous plus and minus signs.

两个命令产生分类码为12\index{category!12}的符号字符串,它们的参数\gr{number}。因此 
\verb-\romannumeral51-、 \verb-\romannumeral\year-、 
、 \verb-\number\linepenalty- 和 \verb-\number13- 都是有效的。所以对\cs{number}一个数字表示形式应用\cs{number}会有多种用途,包括移除前置的0和多余的加减号。

%A roman numeral is a string of lowercase `roman digits',
%which are characters of category code~12.
%The sequence\howto Uppercase roman numberals\par
%\begin{verbatim}
%\uppercase\expandafter{\romannumeral ...}
%\end{verbatim}
%gives uppercase roman numerals.
%This works because \TeX\  expands
%tokens in order to find the opening brace of the argument
%of \verb=\uppercase=. If \cs{romannumeral} is applied to
%a negative number, the result is simply empty.

罗马数是一个小写的罗马数字(roman digits)构成的字符串,这些罗马数字的分类码为12。
序列 \howto Uppercase roman numberals\par

\begin{verbatim}
\uppercase\expandafter{\romannumeral ...}
\end{verbatim}

将给出大写的罗马数。因为\TeX\  展开记号来寻找\verb=\uppercase=参数的左括号,所以小写罗马数字能转换成大写。如果\cs{romannumeral}被用于一个负数,结果将为空。

%%\point Integer registers
%\section{Integer registers}
%\point Integer registers
\section{整数寄存器}

%Integers can be stored in \csidx{count} registers:
%\begin{Disp}\cs{count}\gr{8-bit number}\end{Disp}
%is an \gr{integer variable} and an \gr{internal integer}.
%As an integer variable it can be used in a
%\gr{variable assignment}:
%\begin{Disp}\gr{variable assignment} $\longrightarrow$
%     \gr{integer variable}\gr{equals}\gr{number} $|$ \dots\end{Disp}
%As an internal integer it can be used as a \gr{number}:
%\begin{Disp}\gr{number} $\rightarrow$ \gr{optional signs}\gr{internal integer}
%     $|$ \dots
%\end{Disp}
整数可以保存在\csidx{count}寄存器中:

\begin{Disp}\cs{count}\gr{8-bit number}\end{Disp}

是一个整数变量(\gr{integer variable})和一个内部整数(\gr{internal integer})。作为一个整数变量,它可以用于一个变量赋值(\gr{variable assignment})中:

\begin{Disp}\gr{variable assignment} $\longrightarrow$
     \gr{integer variable}\gr{equals}\gr{number} $|$ \dots\end{Disp}
     

一个内部整数可以用作数字(\gr{number}):
\begin{Disp}\gr{number} $\rightarrow$ \gr{optional signs}\gr{internal integer}
     $|$ \dots
\end{Disp}

%Synonyms for \cs{count} registers can be introduced by the
%\csidx{countdef} command in a \gr{shorthand definition}:
%\begin{Disp}\cs{countdef}\gr{control sequence}\gr{equals}\gr{8-bit number}
%\end{Disp} A control sequence defined this way
%is called a \gr{countdef token}, and it serves as an
%\gr{internal integer}.

\cs{count}寄存器的同名词可以由\csidx{countdef}命令以缩略定义(\gr{shorthand definition})的形式引入。

\begin{Disp}\cs{countdef}\gr{control sequence}\gr{equals}\gr{8-bit number}
\end{Disp} 

以此方式定义的控制序列称为\gr{countdef token},它可作为内部整数(\gr{internal integer})。


%The plain \TeX\ macro \csidx{newcount}
%(which is declared \cs{outer}) uses the \cs{countdef} command
%to allocate an unused \cs{count} register.
%Counters 0--9 are scratch registers, like all
%registers with numbers~0--9.
%However, counters 0--9 are used for page identification
%in the \n{dvi} file (see Chapter~\ref{TeXcomm}),
%so they should be used as scratch
%registers only inside a group.
%Counters 10--22 are
%used for plain \TeX's bookkeeping of allocation of registers.
%Counter 255 is also scratch.
plain \TeX\ 宏 \csidx{newcount} (以\cs{outer}声明) 使用了\cs{countdef} 命令来分配未使用的\cs{count}寄存器。计数器0--9 是保留寄存器,类似于其它名为0--9的寄存器。然而计数器0--9 用于在\n{dvi}文件中的页面识别(见第~\ref{TeXcomm}~章)。所以在一个编组中它们只能用作保留寄存器。计数器10--22,在 plain \TeX 中用于记录寄存器的分配情况,计数器255也是保留的。


%%\point Arithmetic
%\section{Arithmetic}
%\point Arithmetic
\section{算术运算}

%The user can perform some \indexterm{arithmetic}
%in \TeX, and
%\TeX\ also performs arithmetic internally. User arithmetic
%is concerned only with integers; the internal arithmetic
%is mostly on fixed-point quantities, and only in the
%case of glue setting on floating-point numbers.

用户可以在\TeX 中执行算术运算(\indexterm{arithmetic}),\TeX 在内部也执行算术运算。用户使用算术运算仅与整数相关; 内部的算术运算大多基于定点量,只有在粘连中才基于浮点量。


%%\spoint Arithmetic statements
%\subsection{Arithmetic statements}
%\spoint Arithmetic statements
\subsection{运算语句}

%\TeX\ allows the user to
%\cstoidx advance\par\cstoidx multiply\par\cstoidx divide\par
%perform some arithmetic on integers. The statement
%\begin{Disp}\cs{advance}\gr{integer variable}\gr{optional \n{by}}%
%     \gr{number}\end{Disp}
%adds the value of the \gr{number}
% \ldash which may be negative \rdash  to the \gr{integer variable}.
%Similarly,
%\begin{Disp}\cs{multiply}\gr{integer variable}\gr{optional \n{by}}%
%     \gr{number}\end{Disp}
%multiplies the value of the \gr{integer variable}, and
%\begin{Disp}\cs{divide}\gr{integer variable}\gr{optional \n{by}}%
%     \gr{number}\end{Disp}
%divides an \gr{integer variable}.

\TeX\ 运行用户\cstoidx advance\par\cstoidx multiply\par\cstoidx divide\par

对整数执行一些算术运算,语句包括:

\begin{Disp}\cs{advance}\gr{integer variable}\gr{optional \n{by}}%
     \gr{number}\end{Disp}
     
为整数变量(\gr{integer variable})加上一个\gr{number}的值,该值可以是负的。
  
类似的:

\begin{Disp}\cs{multiply}\gr{integer variable}\gr{optional \n{by}}%
     \gr{number}\end{Disp}
     
乘上一个\gr{integer variable}的值,而

\begin{Disp}\cs{divide}\gr{integer variable}\gr{optional \n{by}}%
     \gr{number}\end{Disp}
     
表示除以一个\gr{integer variable}。

%Multiplication and division are also available for any so-called
%\gr{numeric variable}: their most general form is
%\begin{disp}\cs{multiply}\gr{numeric variable}\gr{optional \n{by}}\gr{number}
%\end{disp} where
%\begin{disp}\gr{numeric variable} $\longrightarrow$
%\gr{integer variable} $|$ \gr{dimen variable}\nl
%\indent $|$ \gr{glue variable} $|$ \gr{muglue variable}\end{disp}

乘除运算可以用于所谓的数字变量(\gr{numeric variable}),主要的形式为:

\begin{disp}\cs{multiply}\gr{numeric variable}\gr{optional \n{by}}\gr{number}
\end{disp}

其中

\begin{disp}\gr{numeric variable} $\longrightarrow$
\gr{integer variable} $|$ \gr{dimen variable}\nl
\indent $|$ \gr{glue variable} $|$ \gr{muglue variable}\end{disp}

%The result of an arithmetic operation should not exceed
%$2^{30}$ in absolute value.

运算结果的绝对值不应超过$2^{30}$。

%Division of integers yields an integer; that is, the remainder
%is discarded. This raises the question of how rounding is performed
%when either operand is negative. In such cases \TeX\ performs
%the division with the absolute values of the operands, and
%takes the negative of the result if exactly one operand was negative.

整数的除得到一个整数,即丢弃了余数。这在运算数中有负数时将引起怎么取整的问题。这些情况下,\TeX\ 执行绝对值的除法,然后在结果中加上可能的负号。


%%\spoint Floating-point arithmetic
%\subsection{Floating-point arithmetic}
%\spoint Floating-point arithmetic
\subsection{浮点运算}

%Internally some \indextermsub{floating-point}{arithmetic}
%is performed, namely
%in the calculation of glue set ratios.
%%and slant for accents!!
%However, machine-dependent aspects of rounding cannot
%influence the decision process of \TeX, so machine independence
%of \TeX\ is guaranteed in this respect (sufficient
%accuracy of rounding is enforced by the \n{Trip} test of~\cite{K:trip}).

内部也执行一些浮点运算(\indextermsub{floating-point}{arithmetic}),比如在粘连比例设置计算中。然而依赖于机器的舍入方面不会对 \TeX 的处理过程产生影响,所以 \TeX 在舍入方面是保证与机器无关的(通过\cite{K:trip} 的\n{Trip} 测试来强制保证舍入的精度。)。

%%\spoint Fixed-point arithmetic
%\subsection{Fixed-point arithmetic}
%\spoint Fixed-point arithmetic
\subsection{定点计算}

%All fractional arithmetic in \TeX\ is performed in
%\indextermsub{fixed-point}{arithmetic}
%of `scaled integers': multiples of~$2^{-16}$.
%This ensures the machine independence of \TeX.
%Printed representations of scaled integers are rounded
%to 5 decimal digits.
所有的分数运算是对 `scaled integers' 执行的定点运算(\indextermsub{fixed-point}{arithmetic}): 乘以~$2^{-16}$。
scaled integer的排印精度取到5位小数。


%In ordinary 32-bit implementations of \TeX\ the largest
%integers are $2^{31}-1$ in absolute size.
%The user is not allowed to specify
%dimensions larger in absolute size than~$2^{30}-1$: two
%such dimensions can be added or subtracted without
%overflow on a 32-bit system.

在一般的32位 \TeX\ 实现中,最大的整数绝对值为 $2^{31}-1$。用户不允许设置超过 $2^{30}-1$ 的尺寸: 两个这样的尺寸可以在32位系统中相加或相减而不溢出。

%%\point Number testing
%\section{Number testing}
%\point Number testing
\section{数字测试}

%The most general test for integers in \TeX\ is
%\begin{disp}\cs{ifnum}\gr{number$_1$}\gr{relation}\gr{number$_2$}\end{disp}
%where \gr{relation} is a~\n<, \n>, or~\n= character,
%all of category~12\index{category!12}.
\TeX\ 中最一般的测试为:

\begin{disp}\cs{ifnum}\gr{number$_1$}\gr{relation}\gr{number$_2$}\end{disp}

其中\gr{relation}是一个~\n<, \n>, or~\n= 字符,其分类码为~12\index{category!12}。


%Distinguishing between odd and even numbers is done
%by \begin{disp}\cs{ifodd}\gr{number}\end{disp}

在奇数和偶数之间判断则使用:

\begin{disp}\cs{ifodd}\gr{number}\end{disp}

%A numeric case statement is provided by
%\begin{disp}\cs{ifcase}\gr{number}\gr{case$_0$}\cs{or}\n{...}\cs{or}%
%     \gr{case$_n$}\cs{else}\gr{other cases}\cs{fi}\end{disp}
%where the \cs{else}-part is optional. The tokens for \gr{case$_i$}
%are processed if the number turns out to be~$i$; other cases are
%skipped, similarly to what ordinarily happens in conditionals
%(see Chapter~\ref{if}).

提供的数字分支控制语句为:

\begin{disp}\cs{ifcase}\gr{number}\gr{case$_0$}\cs{or}\n{...}\cs{or}%
     \gr{case$_n$}\cs{else}\gr{other cases}\cs{fi}\end{disp}

其中 \cs{else} 部分是可选的。当数字等于$i$时,记号\gr{case$_i$}将被执行,其它情况将被忽略,逻辑类似于一般的条件判断(见第\ref{if}章)。


%%\point Remarks
%\section{Remarks}
%\point Remarks
\section{注意事项}

%%\spoint Character constants
%\subsection{Character constants}
%\spoint Character constants
\subsection{字符常量}

%In formats and macro collections numeric constants
%are often needed. There are several ways to implement these
%in \TeX.
在格式和宏集中,数字常量常被用到。\TeX 中有多种实现方法:

%Firstly,
%\begin{verbatim}
%\newcount\SomeConstant \SomeConstant=42
%\end{verbatim}
%This is wasteful, as it uses up a \cs{count} register.
第一种:
\begin{verbatim}
\newcount\SomeConstant \SomeConstant=42
\end{verbatim}

这比较浪费,因为它将占据计数器(\cs{count}寄存器)。

%Secondly,
%\begin{verbatim}
%\def\SomeConstant{42}
%\end{verbatim}
%Better but accident prone: \TeX\ has to expand to find the number
% \ldash which in itself is a slight overhead \rdash  and may inadvertently
%expand some tokens that should have been left alone.
第二种:
\begin{verbatim}
\def\SomeConstant{42}
\end{verbatim}

好一些,但偶尔会出问题: \TeX\ 必须展开来找到数字,但不注意的可能导致展开一些需要避免展开的记号。

%Thirdly,
%\begin{verbatim}
%\chardef\SomeConstant=42
%\end{verbatim}
%This one is fine.
%A \gr{chardef token} has the same status as a \cs{count}
%register: both are \gr{internal integer}s.
%Therefore a number defined this way can be used everywhere that
%a \cs{count} register is feasible.
%For large numbers the \cs{chardef} can be replaced by \cs{mathchardef},
%which runs to \verb>"7FFF>${}=32\,767$.
%Note that a \gr{mathchardef token} can usually only appear
%in math mode, but in the context of a number it can appear anywhere.

第三种:
\begin{verbatim}
\chardef\SomeConstant=42
\end{verbatim}

这是好的方法的。一个\gr{chardef token}具有与\cs{count}寄存器相同的状态,两者都是内部整数(\gr{internal integer})。因此以此定义数字可以在\cs{count}寄存器适用的地方适用。对于大的数字可以用\cs{mathchardef}代替\cs{chardef},该命令可以取到 \verb>"7FFF>${}=32\,767$。注意
\gr{mathchardef token}通常只出现在数学模式中,但可在数字的上下文中可以任意出现。


%%\spoint Expanding too far / how far
%\subsection{Expanding too far / how far}
%\spoint Expanding too far / how far
\subsection{展开太远/多远}

%It is a common mistake to write pieces of \TeX\ code
%where \TeX\ will inadvertently expand something because it
%is trying to compose a number. For example:
%\begin{verbatim}
%\def\par{\endgraf\penalty200}
%...\par \number\pageno
%\end{verbatim}
%Here the page number will be absorbed into the value of the penalty.
写\TeX\ 代码时的一个常见错误是\TeX\ 在努力处理数字时不注意地展开一些东西,比如:

\begin{verbatim}
\def\par{\endgraf\penalty200}
...\par \number\pageno
\end{verbatim}

这里页码将被吸收进行 penalty 的值中。


%Now consider
%\begin{verbatim}
%\newcount\midpenalty \midpenalty=200
%\def\par{\endgraf\penalty\midpenalty}
%...\par \number\pageno
%\end{verbatim}
%Here the page number is not scooped up by mistake:
%\TeX\ is trying to locate a \gr{number} after the \cs{penalty},
%and it finds a \gr{countdef token}. This is {\em not\/}
%converted to a representation in digits, so there is never any
%danger of the page number being touched.

现在考虑:
\begin{verbatim}
\newcount\midpenalty \midpenalty=200
\def\par{\endgraf\penalty\midpenalty}
...\par \number\pageno
\end{verbatim}

这里页码不会错误地被吸收,\TeX\ 努力定位\cs{penalty}命令后的\gr{number},
它找到了一个\gr{countdef token}。这{\em 不会} 转换为数字形式,所以碰到页码也不会有任何危险。

%It is possible to convert a \gr{countdef token} first to
%a representation in digits before assigning it:
%\begin{verbatim}
%\penalty\number\midpenalty
%\end{verbatim}
%and this brings back again all previous problems of expansion.

有可能会在赋值前先将\gr{countdef token}转换为数字形式:

\begin{verbatim}
\penalty\number\midpenalty
\end{verbatim}

而这又将再次导致前面出现过的展开问题。


%\endofchapter
%%%%% end of input file [number]
\endofchapter
%%%% end of input file [number]

\end{document}
