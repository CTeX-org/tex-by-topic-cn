%-*- coding: utf-8 -*-
% Translation: RadioNoiseE, 2023
\documentclass{book}


%\nofiles

\usepackage{comment,makeidx,multicol}
\usepackage{svn-multi}
\svnid{$Id: TeXbyTopic.tex 3 2008-02-02 11:19:13Z victor $}
\usepackage{newcent}
\renewcommand{\ttdefault}{cmtt}
\addtolength{\oddsidemargin}{-.5in}
\addtolength{\evensidemargin}{-.5in}
\addtolength{\textwidth}{1.2in}
\addtolength{\textheight}{15pt}

\usepackage[colorlinks,plainpages=true,pagebackref=true]{hyperref}

\begin{comment}
\usepackage{german}
% german
\righthyphenmin=3
\mdqoff
\captionsenglish
\end{comment}
\usepackage[english]{babel} 
{\catcode`"=13 \gdef"#1{\ifx#1"\discretionary{}{}{}\fi\relax}}
\def\mdqon{\catcode`"=13\relax}
\def\mdqoff{\catcode`"=12\relax}
\makeindex
\hyphenation{ex-em-pli-fies}

\usepackage{fancyhdr}
% headers & footers
\pagestyle{fancy}
% foot
\lfoot[\thepage]{\protect\small\protect\it Victor Eijkhout -- \protect\TeX\ by Topic}
\rfoot[{\protect\small\protect\it Victor Eijkhout -- \protect\TeX\ by Topic}]{\thepage}
\cfoot{}
% head is actually defined later.
\chead{}
\def\normalheads{
\lhead[\let\\\relax \let\uppercase\relax \leftmark]{\relax}
\rhead[\relax]{\let\\\relax \let\uppercase\relax \rightmark}
}
\lhead{}\rhead{}

\newdimen\tempdima \newdimen\tempdimb

% these are fine
\def\handbreak{\\ \message{^^JManual break!!!!^^J}}
\def\nl{\protect\\}\def\n#1{{\tt #1}}\def\cs#1{{\tt\char`\\#1}}\let\csc\cs
\def\lb{{\tt\char`\{}}\def\rb{{\tt\char`\}}}
\def\gr#1{$\langle$#1$\rangle$}\def\key#1{{\tt#1}}
\def\alt{}\def\altt{}%this way in manstijl
\def\ldash{\unskip\ --\nobreak\ \ignorespaces}
\def\rdash{\unskip\nobreak\ --\ \ignorespaces}
% check these
\def\hex{{\tt"}}
\def\ascii{{\sc ascii}}
\def\ebcdic{{\sc ebcdic}}
\def\IniTeX{Ini\TeX}\def\LamsTeX{LAMS\TeX}\def\VirTeX{Vir\TeX}
\def\AmsTeX{Ams\TeX}
\def\TeXbook{the \TeX\ book}\def\web{{\sc web}}
% needs major thinking
\newenvironment{disp}{\begin{quotation}}{\end{quotation}}
\newenvironment{Disp}{\begin{quotation}}{\end{quotation}}
\newenvironment{tdisp}{\begin{quotation}}{\end{quotation}}
\newenvironment{example}{\begin{quotation}}{\end{quotation}}
\newenvironment{inventory}{\begin{description}\raggedright}{\end{description}}
\newenvironment{glossinventory}{\begin{description}}{\end{description}}
\def\gram#1{\gr{#1}}%???
%
% index
%
\def\indexterm#1{\emph{#1}\index{#1}}
\def\indextermsub#1#2{\emph{#1 #2}\index{#1!#2}}
\def\indextermbus#1#2{\emph{#1 #2}\index{#2!#1}}
\def\term#1\par{\index{#1}}
\def\howto#1\par{}
\def\cstoidx#1\par{\index{#1@\cs{#1}@}}
\def\thecstoidx#1\par{\index{#1@\protect\csname #1\endcsname}}
\def\thecstoidxsub#1#2{\index{#1, #2@\protect\csname #1\endcsname, #2}\ignorespaces}
\def\csterm#1\par{\cstoidx #1\par\cs{#1}}
\def\csidx#1{\cstoidx #1\par\cs{#1}}

\def\tmc{\tracingmacros=2 \tracingcommands\tracingmacros}

%%%%%%%%%%%%%%%%%%%
\makeatletter
\def\snugbox{\hbox\bgroup\setbox\z@\vbox\bgroup
    \leftskip\z@
    \bgroup\aftergroup\make@snug
    \let\next=}
\def\make@snug{\par\sn@gify\egroup \box\z@\egroup}
\def\sn@gify
   {\skip\z@=\lastskip \unskip
    \advance\skip\z@\lastskip \unskip
    \unpenalty
    \setbox\z@\lastbox
    \ifvoid\z@ \nointerlineskip \else {\sn@gify} \fi
    \hbox{\unhbox\z@}\nointerlineskip
    \vskip\skip\z@
    }

\newdimen\fbh \fbh=60pt % dimension for easy scaling:
\newdimen\fbw \fbw=60pt % height and width of character box

\newdimen\dh \newdimen\dw % height and width of current character box
\newdimen\lh % height of previous character box
\newdimen\lw \lw=.4pt % line weight, instead of default .4pt

\def\hdotfill{\noindent
    \leaders\hbox{\vrule width 1pt height\lw 
                  \kern4pt 
                  \vrule width.5pt height\lw}\hfill\hbox{}
    \par}
\def\hlinefill{\noindent
    \leaders\hbox{\vrule width 5.5pt height\lw         }\hfill\hbox{}
    \par}
\def\stippel{$\qquad\qquad\qquad\qquad$}
\makeatother
%%%%%%%%%%%%%%%%%%%

\begin{comment}
\def\SansSerif{\Typeface:macHelvetica }
\def\SerifFont{\Typeface:macTimes }
\def\SansSerif{\Typeface:bsGillSans }
\def\SerifFont{\Typeface:bsBaskerville }
\end{comment}
\let\SansSerif\relax \def\italic{\it}
\let\SerifFont\relax \def\MainFont{\rm}
\let\SansSerif\relax
\let\SerifFont\relax
\let\PopIndentLevel\relax \let\PushIndentLevel\relax
\let\ToVerso\relax \let\ToRecto\relax

\begin{comment}
\def\stop@command@suffix{stop}
\let\PopListLevel\PopIndentLevel
\let\FlushRight\relax
\let\flushright\FlushRight
\let\SetListIndent\LevelIndent
\def\awp{\ifhmode\vadjust{\penalty-10000 }\else
    \penalty-10000 \fi}
\end{comment}
\let\awp\relax
\let\PopIndentLevel\relax \let\PopListLevel\relax

\showboxdepth=-1

%\input figs
\def\endofchapter{\vfill\noindent}

\setcounter{chapter}{11}

\begin{document}

%\chapter{Expansion}\label{expand}
\chapter{展开}\label{expand}

%\emph{Expansion}\index{expansion} in \TeX\ is rather different from
%procedure calls in most programming languages. This chapter treats the
%commands connected with expansion, and gives a number of (non-trivial)
%examples.
\index{expansion}{\TeX}中的\emph{展开}不同于其他多数程序设计语言中的过程调用。
本章将讨论于展开相关的命令,并给出一些其(不具体)的实例。

%\label{cschap:relax}\label{cschap:expandafter}\label{cschap:noexpand}\label{cschap:afterassignment}\label{cschap:the}
%\begin{inventory}
%\item [\cs{relax}] 
%     Do nothing.
\label{cschap:relax}\label{cschap:expandafter}\label{cschap:noexpand}\label{cschap:afterassignment}\label{cschap:the}
\begin{inventory}
\item [\cs{relax}] 
     什么都不做。


%\item [\cs{expandafter}]  
%      Take the next two tokens and place the expansion of the
%      second after the first.
\item [\cs{expandafter}]  
      (从输入流上)取走其后的两个记号,将第二个记号展开并置于第一个记号之后。

%\item [\cs{noexpand}]   
%      Do not expand the next token.
\item [\cs{noexpand}]   
      使其后的记号不被展开。


%\item [\cs{edef}] 
%      Start a macro definition; 
%      the replacement text is expanded at definition time.
\item [\cs{edef}] 
      进行宏定义;该宏的定义内容将会在定义时展开。
% 
%\item [\cs{aftergroup}]  
%      Save the next token for insertion after the current group.
 
\item [\cs{aftergroup}]  
      存储其后的一个记号并在当前分组结束后将其放入输入流中。

%\item [\cs{afterassignment}]   
%      Save the next token for execution after the next assignment
%      or macro definition.
\item [\cs{afterassignment}]   
      存储其后的一个记号并在下一个赋值抑或宏定义后执行之。

%\item [\cs{the}] 
%      Expand the value of various quantities in \TeX\ into a string
%      of character tokens.
\item [\cs{the}] 
      将{\TeX}中的某些量展开为包含字符记号的字符串。

%\end{inventory}
\end{inventory}


%%\point Introduction
%\section{Introduction}
%\point Introduction
\section{简介}

%\TeX's expansion processor accepts a stream of tokens
%coming out of the input processor, and its result is
%again a stream of tokens, which it feeds to the execution
%processor. For the input processor there are two
%kinds of tokens: expandable and unexpandable ones.
%The latter category is passed untouched, and it contains
%largely assignments and typesettable material;
%the former category
%is expanded, and the result of that expansion is examined anew.
{\TeX}的展开处理器接受来自输入处理器中的记号流,并将其输出为被执行处理器处理的记号流。
对于输入处理器而言有两种记号:可展开的及不可展开的。
那类主要包含赋值或供排版内容的(不可展开的)记号会原封不动通过展开处理器;
而对于可展开的记号则会被展开,同时展开的结果会被重新检验(以确保完全展开)。

%%\point Ordinary expansion
%\section{Ordinary expansion}
%\point Ordinary expansion
\section{一般展开}

%The following list gives those constructs
%that are expanded, unless
%expansion is inhibited:
%\index{expansion!expandable constructs}
%\begin{itemize}
%\item macros\label{expand:lijst}
%\item conditionals
%\item \cs{number}, \cs{romannumeral}
%\item \cs{string}, \cs{fontname}, \cs{jobname}, 
%      \cs{meaning}, \cs{the}
%\item \verb,\csname ... \endcsname,
%\item \cs{expandafter}, \cs{noexpand}
%\item \cs{topmark}, \cs{botmark}, \cs{firstmark}, 
%      \cs{splitfirstmark}, \cs{splitbotmark}
%\item \cs{input}, \cs{endinput}
%\end{itemize}
那些若不进行抑制,则会被展开的结构体将在下面的列表中给出。
\index{expansion!expandable constructs}
\begin{itemize}
\item macros\label{expand:lijst}
\item conditionals
\item \cs{number}, \cs{romannumeral}
\item \cs{string}, \cs{fontname}, \cs{jobname}, 
      \cs{meaning}, \cs{the}
\item \verb,\csname ... \endcsname,
\item \cs{expandafter}, \cs{noexpand}
\item \cs{topmark}, \cs{botmark}, \cs{firstmark}, 
      \cs{splitfirstmark}, \cs{splitbotmark}
\item \cs{input}, \cs{endinput}
\end{itemize}

%This is the list of all instances where
%expansion is inhibited:
%\begin{itemize}\label{noexp:list}
%\item when \TeX\ is reading a token to be defined by
%      
%\begin{itemize} \item a \gr{let assignment}, that is,
%           by \cs{let} or \cs{futurelet},
%        \item a \gr{shorthand definition}, that is,
%           by \cs{chardef} or \cs{mathchardef}, or a
%           \gr{register def}, that is, \cs{countdef},
%           \cs{dimendef}, \cs{skipdef}, \cs{muskipdef},
%           or~\cs{toksdef},
%        \item a \gr{definition}, that is a macro definition
%           with \cs{def}, \cs{gdef}, \cs{edef}, or~\cs{xdef},
%        \item the \gr{simple assignment}s \cs{read} and \cs{font};
%      \end{itemize}
%\item when a \gr{parameter text} or macro arguments
%      are being read; also when  the replacement text of a 
%      control sequence
%      being defined by \cs{def}, \cs{gdef}, or \cs{read}
%      is being read;
%\item when the token list for a \gr{token variable} or
%      \cs{uppercase}, \cs{lowercase}, or \cs{write}
%      is being read; however, the token list for \cs{write}
%      will be  expanded later when it is shipped out;
%\item when tokens are being deleted during error recovery;
%\item when part of a conditional is being skipped;
%\item in two instances when \TeX\ has to know what follows
%      
%\begin{itemize}\item after a left quote in a context where
%         that is used to denote an integer (thus in 
%         \verb-\catcode`\a- the \cs{a} is not expanded), or
%        \item after a math shift character that begins math mode
%         to see whether another math shift character follows (in which case
%         a display opens);
%        \end{itemize}
%\item when an alignment preamble is being scanned; however,
%      in this case a~token
%      preceded by \cs{span} and the tokens in a \cs{tabskip} 
%      assignment are still expanded.
%\end{itemize}
下面则是展开会被抑制的情况:
\begin{itemize}\label{noexp:list}
\item 当{\TeX}在读入被以下列方法定义的记号时。
    \begin{itemize} \item 一个\gr{let assignment},即,
           使用\cs{let}或\cs{futurelet},
        \item 一个\gr{shorthand definition},即,
           使用\cs{chardef}或\cs{mathchardef}也可能是一个
           \gr{register def},即,\cs{countdef}、
           \cs{dimendef}、\cs{skipdef}、\cs{muskipdef}、
           或\cs{toksdef},
        \item 一个\gr{definition},即一个宏定义使用
           \cs{def}、\cs{gdef}、\cs{edef}、或\cs{xdef},
        \item \gr{simple assignment},即使用 \cs{read}或\cs{font}的简单赋值;
    \end{itemize}
\item 当\gr{parameter text}或宏参量被读入时;
      也可能在当被使用\cs{def}、\cs{gdef}、\cs{read}定义的控制序列被读入时;
\item 当作为\gr{token variable}、\cs{uppercase}、\cs{lowercase}、或\cs{write}的参量被读取时;
      但是,作为\cs{write}的参量时则会被在作用时展开;
\item 当在错误恢复时记号被删除时;
\item 当作为条件判断(错误)的一部分被忽略时;
\item 在当{\TeX}知道紧随其后的是什么的以下两种情况下
    \begin{itemize}\item 在被用于表示整数的左引号后时(所以\verb-\catcode`\a-时\cs{a}不会被展开)、
        \item 在一个开始数学模式的数学模式切换符出现后判断是否紧随另一个此类字符时(若如此、则为公式展示模式)、
    \end{itemize}
\item 当一个对齐导言被读入时;
      但是,若前面出现\cs{span}时、或作为\cs{tabskip}参量的赋值仍会被展开。
\end{itemize}

%%\point Reversing expansion order
%\section{Reversing expansion order}
%\point Reversing expansion order
\section{颠倒展开顺序}

%Every once in a while you need to change the normal order of
%expansion of tokens. \TeX\ provides several mechanisms for
%this. Some of the control sequences in this section are
%not strictly concerned with expansion.
有时会出现需要调整记号的一般展开顺序的情况。
为此,{\TeX}提供了几种机制。
本节中的某些控制序列并不局限于展开。

%%\spoint One step expansion: \cs{expandafter}
%\subsection{One step expansion: \protect\cs{expandafter}}
%\spoint One step expansion: \cs{expandafter}
\subsection{提前一步展开:\protect\cs{expandafter}}

%The most obvious tool for reversed expansion order is
%\csidx{expandafter}. The sequence
%\begin{disp}\cs{expandafter}\gr{token$_1$}\gr{token$_2$}\end{disp}
%expands to 
%\begin{disp}\gr{token$_1$}\gr{\italic the expansion of token$_2$}
%\end{disp}
%Note the following.
%\begin{itemize} \item If \gr{token$_2$} is a macro, it is replaced
%by its replacement text, not by its final expansion.
%Thus, if 
%\begin{verbatim}
%\def\tokentwo{\ifsomecondition this \else that \fi}
%\def\tokenone#1{ ... }
%\end{verbatim}
%the call
%\begin{verbatim}
%\expandafter\tokenone\tokentwo
%\end{verbatim}
%will give \cs{ifsomecondition} as the parameter
%to \cs{tokenone}:
%\begin{verbatim}
%\tokenone #1-> ...
%#1<-\ifsomecondition
%\end{verbatim}
%\item If the \cs{tokentwo} is a macro with one or more
%parameters, sufficiently many subsequent tokens will be absorbed
%to form the replacement text.\end{itemize}
最常用的可用于颠倒展开顺序的工具是
\csidx{expandafter}。以下
\begin{disp}\cs{expandafter}\gr{token$_1$}\gr{token$_2$}\end{disp}
序列等同于
\begin{disp}\gr{token$_1$}\gr{\italic the expansion of token$_2$}
\end{disp}
若在以下两种情况出现时则须额外注意:
\begin{itemize} \item 若\gr{token$_2$}是一个宏,
则其会被它的宏定义结果替换,而非它的最终完全展开结果。所以,
\begin{verbatim}
\def\tokentwo{\ifsomecondition this \else that \fi}
\def\tokenone#1{ ... }
\end{verbatim}
以下的
\begin{verbatim}
\expandafter\tokenone\tokentwo
\end{verbatim}
会将\cs{ifsomecondition}作为
\cs{tokenone}的参量:
\begin{verbatim}
\tokenone #1-> ...
#1<-\ifsomecondition
\end{verbatim}
\item 若\cs{tokentwo}是一个使用一个或更多参量的宏,
则足够多的其后的记号会被吸收,
从而完成宏替换内容的构成。
\end{itemize}

%\subsection{Total expansion: \protect\cs{edef}}
%\label{expand:edef}
\subsection{完全展开:\protect\cs{edef}}
\label{expand:edef}

%Macros are usually defined by \cs{def}, but for the cases where
%one wants the replacement text to reflect current conditions
%(as opposed to conditions at the time of the call),
%there is an `expanding define', \csidx{edef}, which expands
%everything in the replacement text, before assigning it to the
%control sequence. 
宏通常是被使用\cs{def}定义的,
但当我们希望使用宏替换内容去反映定义时的状况
(与通过其反映被调用是状况不同)时,
即可使用「展开定义」,\cs{edef}。
它会将宏替换内容在将其赋值给(被定义的)宏之前完全展开。

%\begin{example}
%\begin{verbatim}
%\edef\modedef{This macro was defined in
%    `\ifvmode vertical\else \ifmmode math
%     \else horizontal\fi\fi' mode}
%\end{verbatim}
%The mode tests will be executed at definition time, so the 
%replacement text will be a single string.
\begin{example}
\begin{verbatim}
\edef\modedef{This macro was defined in
    `\ifvmode vertical\else \ifmmode math
     \else horizontal\fi\fi' mode}
\end{verbatim}
因为模式测试在定义时即被执行,故该宏的替换文本将是一个字符串。

%As a more useful example, suppose that in a file that will be
%\cs{input} the category code of the~\n@ will be changed.
%One could then write
%\begin{verbatim}
%\edef\restorecat{\catcode`@=\the\catcode`@}
%\end{verbatim}
%at the start, and 
%\begin{verbatim}
%\restorecat
%\end{verbatim}
%at the end. See page~\pageref{store:cat}
%for a fully worked-out version of this.
%\end{example}
举一个更有用的例子,若要使用\cs{input}读入一个
\texttt{\@} 的类别码将会发生改变的文件,则可以有
\begin{verbatim}
\edef\restorecat{\catcode`@=\the\catcode`@}
\end{verbatim}
在文件开头,以及
\begin{verbatim}
\restorecat
\end{verbatim}
在文件末尾。可见第\pageref{store:cat}页的一个更完整的版本。
\end{example}

%Contrary to the `one step expansion' of
%\cs{expandafter}, the expansion inside an \cs{edef} is complete:
%it goes on
%until only unexpandable character and control sequence
%tokens remain.
%There are two exceptions to this total expansion:
%\begin{itemize} \item any control sequence preceded by \cs{noexpand}
%is not expanded, and,
%\item if \cs{sometokenlist} is a token list, the expression
%\begin{verbatim}
%\the\sometokenlist 
%\end{verbatim}
%is expanded to the contents
%of the list, but the contents are not expanded
%any further (see Chapter~\ref{token} for examples).\end{itemize}
与「只展开一步」的\cs{expandafter}相异,在\cs{edef}内的展开是彻底的:
它会持续展开,直到只有不可展开的字符或控制序列留在输入流中。
对于这类彻底的展开,有两个例外情况:
\begin{itemize} \item 任何\cs{noexpand}之后的控制序列不会被展开,及
    \item 若\cs{sometokenlist}是一个记号列表,则以下
\begin{verbatim}
\the\sometokenlist 
\end{verbatim}
只会展开为记号列表的内容、而其内容则不会被进一步展开
(例见第\ref{token}章)。
\end{itemize}

%On certain occasions the \cs{edef} can conveniently be
%abused, in the sense that one is not interested in defining
%a control sequence, but only in the result of the  expansion.
%For example, with the definitions
%\alt
%\begin{verbatim}
%\def\othermacro{\ifnum1>0 {this}\else {that}\fi}
%\def\somemacro#1{ ... }
%\end{verbatim}
%the call
%\begin{verbatim}
%\expandafter\somemacro\othermacro 
%\end{verbatim}
%gives the parameter assignment
%\begin{verbatim}
%#1<-\ifnum
%\end{verbatim}
%This can be repaired by calling
%\begin{verbatim}
%\edef\next{\noexpand\somemacro\othermacro}\next
%\end{verbatim}
%Conditionals are completely expanded inside an \cs{edef},
%so the replacement text of \cs{next} will consist of the sequence
%\begin{verbatim}
%\somemacro{this}
%\end{verbatim} 
%and a~subsequent call to \cs{next} executes this statement.
在某些情况之下,\cs{edef}可被方便的「滥用」,
即当我们无意定义一个控制序列、仅关注展开结果时。
比如,使用如下定义
\alt
\begin{verbatim}
\def\othermacro{\ifnum1>0 {this}\else {that}\fi}
\def\somemacro#1{ ... }
\end{verbatim}
宏调用
\begin{verbatim}
\expandafter\somemacro\othermacro 
\end{verbatim}
将会给出如下参量的赋值
\begin{verbatim}
#1<-\ifnum
\end{verbatim}
为使其正确,我们可以这么写
\begin{verbatim}
\edef\next{\noexpand\somemacro\othermacro}\next
\end{verbatim}
其中的条件判断被\cs{edef}完全展开,
所以\cs{next}的替换文本将会是该序列
\begin{verbatim}
\somemacro{this}
\end{verbatim} 
同时其后的\cs{next}调用执行该定义。


%%\spoint \cs{afterassignment}
%\subsection{\protect\cs{afterassignment}}
%\label{sec:afterassignment}
%\spoint \cs{afterassignment}
\subsection{\protect\cs{afterassignment}}
\label{sec:afterassignment}

%The \csterm afterassignment\par\ command
%takes one token and sets it aside for insertion
%in the token stream
%after the next assignment or macro definition.
%If the first assignment is of a~box
% to a box register,
%the token will be inserted right after the opening
%\alt
%brace of the box (see page~\pageref{every:box:assign}).
\csterm afterassignment\par 指令从输入流上取走一个记号,
并在下一个赋值或宏定义后将其插入到记号流中。
如果随后出现的是将盒子赋给盒子寄存器的赋值,
则该记号会在界定盒子内容的左花括号之后被插入
(见第\pageref{every:box:assign}页)。

%Only one token can be saved this way; a subsequent token 
%saved by \cs{afterassignment} will override the first.
使用这种方式只能存储一个记号;
之后的\cs{afterassignment}存储的记号会覆盖之前的。

%Let us consider an example of the use of \cs{afterassignment}.
%It is often desirable to have a macro that will
%\begin{itemize} \item assign the argument to some variable, and then
%\item do a little calculation, based on the new value
%of the variable.\end{itemize}
%The following example illustrates the
%straightforward approach:
%\begin{verbatim}
%\def\setfontsize#1{\thefontsize=#1pt\relax
%    \baselineskip=1.2\thefontsize\relax}
%\setfontsize{10}
%\end{verbatim}
%A more elegant solution is possible using \cs{afterassignment}:
%\begin{verbatim}
%\def\setbaselineskip
%   {\baselineskip=1.2\thefontsize\relax}
%\def\fontsize{\afterassignment\setbaselineskip
%    \thefontsize}
%\fontsize=10pt
%\end{verbatim}
%Now the macro looks like an assignment: the equals sign
%is even optional. In reality its expansion
%ends with a variable to be assigned to. The control sequence
%\cs{setbaselineskip} is saved for execution after
%the assignment to \cs{thefontsize}.
不妨考虑实际使用\cs{afterassignment}的例子。
一个诱人的想法可能是能够定义一个宏来完成如下:
\begin{itemize} \item 将参数赋予某个值,并在之后
\item 进行一些基于该参量新值的简单计算。
\end{itemize}
这是对其较直接的实现方法:
\begin{verbatim}
\def\setfontsize#1{\thefontsize=#1pt\relax
    \baselineskip=1.2\thefontsize\relax}
\setfontsize{10}
\end{verbatim}
此时,基于\cs{afterassignment},我们可以有如下更优雅的实现方法:
\begin{verbatim}
\def\setbaselineskip
   {\baselineskip=1.2\thefontsize\relax}
\def\fontsize{\afterassignment\setbaselineskip
    \thefontsize}
\fontsize=10pt
\end{verbatim}
此时,这个宏看上去更像一个赋值:且等于号是可选的。
实际上,它的展开以一个将会被赋值的变量结束。
控制序列\cs{setbaselineskip}被存储直至对\cs{thefontsize}赋值时才执行。

%Examples of \cs{afterassignment} in plain \TeX\ are
%the \cs{magnification} and \cs{hglue} macros.
%See \cite{Maus} for another creative application of
%this command.
在plain {\TeX}中,使用\cs{afterassignment}的实例有
\cs{magnification}
以及\cs{hglue}宏。
对于其的另一个创造性的应用可见\cite{Maus}。

%\subsection{\protect\cs{aftergroup}}
%\label{sec:aftergroup}
\subsection{\protect\cs{aftergroup}}
\label{sec:aftergroup}

%Several tokens can be saved for insertion after the current
%\cstoidx aftergroup\par
%group with an 
%\begin{disp}\cs{aftergroup}\gr{token}\end{disp} command.
%The tokens are inserted after the group in the sequence
%the \cs{aftergroup} commands were given in.
%The group can be delimited either by implicit or explicit
%braces, or by \cs{begingroup} and \cs{endgroup}.
某些记号可被
\cstoidx aftergroup\par
\begin{disp}\cs{aftergroup}\gr{token}\end{disp}
控制序列存储以备当前分组结束后再行插入。
这些记号会在\cs{aftergroup}序列出现的分组结束后被插入。
分组可同时被使用直接或间接的花括号来定界,
或使用\cs{begingroup}及\cs{endgroup}。

%\begin{example}
%\begin{verbatim}
%{\aftergroup\a \aftergroup\b}
%\end{verbatim}
%is equivalent to
%\begin{verbatim}
%\a \b
%\end{verbatim}
%\end{example}
\begin{example}
\begin{verbatim}
{\aftergroup\a \aftergroup\b}
\end{verbatim}
等同于
\begin{verbatim}
\a \b
\end{verbatim}
\end{example}

%This command has many applications. One can be found
%\alt
%in the \cs{textvcenter} macro on page~\pageref{text:vcenter};
%another one is provided
%by the footnote mechanism of plain \TeX.
此命令可以有许多应用。
其中\cs{textvcenter}宏对其的应用
可在第\pageref{text:vcenter}页上被找到;
另一个则是plain {\TeX}中的脚注机制。

%The footnote command of plain \TeX\ has the layout
%\label{footnote:ex}
%\begin{disp}\cs{footnote}\gr{footnote symbol}\lb\gr{footnote text}\rb
%\end{disp} which looks like a macro with two arguments.
%However, it is undesirable to scoop up the footnote text,
%since this precludes for
%instance category code changes in the footnote.
在plain {\TeX}中,使用脚注命令的语法如下
\label{footnote:ex}
\begin{disp}\cs{footnote}\gr{footnote symbol}\lb\gr{footnote text}\rb
\end{disp} 这让它看起来像是使用两个参数的宏。
然而,令人不快的是它需要读入脚注内容,
而这则妨碍了脚注中出现如类别码变动等情况。

%What happens in the plain footnote macro is (globally) the following.
%\begin{itemize}\item The \cs{footnote} command opens
%an insert,
%\begin{verbatim}
%\def\footnote#1{ ...#1... %treat the footnote sign
%    \insert\footins\bgroup
%\end{verbatim}
%\item In the insert box a group is opened,
%and an \cs{aftergroup} command
%is given to close off the insert properly:
%\begin{verbatim}
%    \bgroup\aftergroup\@foot
%\end{verbatim}
%This command is meant to wind up after the closing brace of
%the text that the user typed to end the footnote text;
%the opening brace of the user's footnote text must
%be removed by 
%\begin{verbatim}
%    \let\next=}%end of definition \footnote
%\end{verbatim}
%which assigns the next token, the brace, to \cs{next}.
%\item The footnote text is set as ordinary text
%in this insert box.
%\item After the footnote the command \cs{@foot}
%defined by
%\begin{verbatim}
%\def\@foot{\strut\egroup}
%\end{verbatim}
%will be executed.\end{itemize}
实际上,该脚注宏主要是以如下的方式实现的。
\begin{itemize}\item 首先,\cs{footnote}宏开启一个浮动体插入物:
\begin{verbatim}
\def\footnote#1{ ...#1... %treat the footnote sign
    \insert\footins\bgroup
\end{verbatim}
\item 在该插入物盒子中,
\cs{aftergroup}的命令使得该盒子能够被正确的关闭:
\begin{verbatim}
    \bgroup\aftergroup\@foot
\end{verbatim}
其起到的作用是,
使用户输入的结束脚注文本的右花括号无效化;
而左花括号则需使用如下方式无效化:
\begin{verbatim}
    \let\next=}%end of definition \footnote
\end{verbatim}
即将随后的左花括号记号赋值给\cs{next}序列。
\item 脚注文本即可如一般文本一样在浮动体插入物盒子中排版。
\item 脚注后,如下定义的\cs{@foot}序列
\begin{verbatim}
\def\@foot{\strut\egroup}
\end{verbatim}
会被最后执行。
\end{itemize}


%%\point Preventing expansion
%\section{Preventing expansion}
%\point Preventing expansion
\section{抑制展开}

%Sometimes it is necessary to prevent expansion in a place
%where it normally occurs. For this purpose the control
%sequences \csidx{string} and \csidx{noexpand} are available.
有时在一般会被展开的地方抑制它的展开是必要的。
为了该用途,
控制序列\csidx{string}和\csidx{noexpand}可供使用。
%The use of \cs{string} is rather limited, since it converts
%a control sequence token into a string of characters, with
%the value of \cs{escapechar} used for the character of
%category code~0\index{category!0}.
%It is eminently suitable for use in a 
%\cs{write}, in order to output a control sequence name
%(see also Chapter~\ref{io}); for another application see
%the explanation of \cs{newif} in Chapter~\ref{if}.
\cs{string}的使用非常局限,
这是由于它将一个控制序列记号转换为一个字符串
(此时\cs{escapechar}的值会被用于替换原类别码为0的字符)。
显而易见,它很适合用在使用\cs{write}的场合,                        
来输出一个控制序列的名字(亦见第\ref{io}章);
其别的用途可见第\ref{if}章中对\cs{newif}的介绍。

%All characters resulting from \cs{string} have category
%code~12, `other', except for space characters; they receive
%code~10. See also Chapter~\ref{char}.
所有\cs{string}输出的字符除类别码10的空格外,
都属于类别码12(「其他」)。
也见第\ref{char}章。

%%\spoint \cs{noexpand}
%\subsection{\protect\cs{noexpand}}
%\spoint \cs{noexpand}
\subsection{\protect\cs{noexpand}}

%The \cs{noexpand} command is expandable, and its expansion
%is the following token. The meaning of that token is
%made temporarily equal to \cs{relax}, so that it cannot
%be expanded further.
命令\cs{noexpand}是可展开的,且其展开为其后面的记号。
(它后面的)记号的意义会暂时等同于\cs{relax},
如此使其不会被进一步展开。

%For \cs{noexpand} the most important application is probably
%in \cs{edef} commands (but in write statements it can often
%replace \cs{string}). Consider as an example
%\begin{verbatim}
% \edef\one{\def\noexpand\two{\the\prevdepth}}
%\end{verbatim}
%Without the \cs{noexpand}, \TeX\ would try to expand
%\cs{two}, thus giving an `undefined control sequence' error.
对于控制序列\cs{noexpand}而言,
最重要的应用莫过于在\cs{edef}中时
(但在\cs{write}的情况下,它也经常能够替代\cs{string})。
考虑下例:
\begin{verbatim}
 \edef\one{\def\noexpand\two{\the\prevdepth}}
\end{verbatim}
若没有\cs{noexpand},则{\TeX}会试图展开\cs{two},
便会出现「未定义控制序列」的报错。

%A  (rather pointless)
%illustration of the fact that \cs{noexpand} makes the following
%token effectively into a \cs{relax} is
%\begin{verbatim}
%\def\a{b}
%\noexpand\a
%\end{verbatim}
%This will not produce any output, because the
%effect of the \cs{noexpand} is to make the control sequence
%\cs{a} temporarily equal to \cs{relax}.
一个(没太大意义的)
下例可表明\cs{noexpand}使其后的记号等同于\cs{relax}:
\begin{verbatim}
\def\a{b}
\noexpand\a
\end{verbatim}
因\cs{noexpand}使控制序列\cs{a}暂时等同于\cs{relax},
故其将不会有任何输出。

%%\spoint \cs{noexpand} and active characters
%\subsection{\protect\cs{noexpand} and active characters}
%\spoint \cs{noexpand} and active characters
\subsection{\protect\cs{noexpand}及活动字符}

%The combination \cs{noexpand}\gr{token} 
%is\index{character!active, and \cs{noexpand}}
%equivalent to \cs{relax}, even if the token
%is an active character. Thus,
%\begin{verbatim}
%\csname\noexpand~\endcsname
%\end{verbatim}
%will not be the same as~\verb>\char`\~>.
%Instead it will give an error message, because
%unexpandable commands \ldash such as \cs{relax} \rdash  are not allowed to appear
%in between \cs{csname} and \cs{endcsname}.
%The solution is to use \cs{string} instead; see page~\pageref{store:cat}
%for an example.
序列\cs{noexpand}\gr{token}的组合
总是\index{character!active, and \cs{noexpand}}
等同于\cs{relax},即使该记号是一个活动字符。
所以,
\begin{verbatim}
\csname\noexpand~\endcsname
\end{verbatim}
不意味着\verb>\char`\~>。
相反,因为不可展开的命令——比如说\cs{relax}——
不被允许出现在\cs{csname}和\cs{endcsname}中
,它会提示出现错误。
解决方法则是转而使用\cs{string};例见\pageref{store:cat}页。

%In another context, however, the sequence
%\cs{noexpand}\gr{active character} is equivalent
%to the character, but in unexpandable form. This is
%when the conditionals \cs{if} and \cs{ifcat} are used
%(for an explanation of these, see Chapter~\ref{if}).
%Compare
%\begin{verbatim}
%\if\noexpand~\relax % is false
%\end{verbatim}
%where the character code of the tilde is tested, with
%\begin{verbatim}
%\def\a{ ... } \if\noexpand\a\relax % is true
%\end{verbatim}
%where two control sequences are tested.
但在某些其他情境下,\cs{noexpand}\gr{active character}的序列
等同于该字符不可展开的形式。
这时条件判断命令\cs{if}和\cs{ifcat}可被使用说明这一点
(这两个控制序列的解释可见第\ref{if}章)。
比较下两种情况
\begin{verbatim}
\if\noexpand~\relax % is false
\end{verbatim}
此时\texttt{\~}的字符编码被测试;而
\begin{verbatim}
\def\a{ ... } \if\noexpand\a\relax % is true
\end{verbatim}
此时两个控制序列被比较。

%%\point \cs{relax}
%\section{\protect\cs{relax}}
%\point \cs{relax}
\section{\protect\cs{relax}}

%The control sequence \csidx{relax} cannot be expanded, but
%when it is executed nothing happens.
控制序列\csidx{relax}不可展开,
但它被执行的时候则什么也不会做。

%This statement sounds a bit paradoxical, so  consider
%an example. Let  counters
%\begin{verbatim}
%\newcount\MyCount
%\newcount\MyOtherCount \MyOtherCount=2
%\end{verbatim}
%be given.
%In the assignment
%\begin{verbatim}
%\MyCount=1\number\MyOtherCount3\relax4
%\end{verbatim}
%the command \cs{number} is expandable, and \cs{relax} is not.
%When \TeX\ constructs the number that is to be assigned
%it will expand all commands, either until a non-digit is 
%found, or until an unexpandable command is encountered.
%Thus it reads the~\n1; it expands the sequence \verb>\number\MyOtherCount>,
%which gives~\n2; it reads the~\n3; it sees the \cs{relax}, and
%as this is unexpandable it halts. The number to be assigned
%is then \n{123}, and the whole call has been expanded into
%\begin{verbatim}
%\MyCount=123\relax4
%\end{verbatim}
%Since the \cs{relax} token has no effect when it is executed,
%the result of this line is that \n{123} is assigned to
%\cs{MyCount}, and the digit 4 is printed.
以上说明感觉上有些似是而非,不妨看一个例子。
给出如下计数器
\begin{verbatim}
\newcount\MyCount
\newcount\MyOtherCount \MyOtherCount=2
\end{verbatim}
在赋值过程中,
\begin{verbatim}
\MyCount=1\number\MyOtherCount3\relax4
\end{verbatim}
命令\cs{number}是可展开的,
而\cs{relax}则不可以。
当{\TeX}构建将被赋给的值的时候,
它会展开所有命令
直到遇到非数字或不可展开的命令为止。
所以它读入\n1;
它会展开\verb>\number\MyOtherCount>
(其展开后值为\n2);
它接着读入\n3;
它又读入\cs{relax},
而由于其不可展开,
于是便最终停止对赋值过程中值的构建。
于是最终赋给\cs{MyCount}的值即为\n{123},
因为展开的结果为:
\begin{verbatim}
\MyCount=123\relax4
\end{verbatim}
又因为\cs{relax}的执行结果是什么也不做,
所以该行将会把值\n{123}
赋给\cs{MyCount},
并排版数字\n4

%Another example of how \cs{relax} can be used to indicate
%the end of a command\label{fil:l:l}\ is
%\begin{verbatim}
%\everypar{\hskip 0cm plus 1fil }
%\indent Later that day, ... 
%\end{verbatim}
%This will be misunderstood: \TeX\ will see
%\begin{verbatim}
%\hskip 0cm plus 1fil L
%\end{verbatim}
%and \hbox{\n{fil L}} is a~valid,
%if bizarre,
%way of writing \n{fill} (see Chapter~\ref{gramm}).
%One remedy is to write
%\begin{verbatim}
%\everypar{\hskip 0cm plus 1fil\relax}
%\end{verbatim}
命令\cs{relax}还可被用于(显式)表示一个命令的结束,
如下例\label{fil:l:l}:
\begin{verbatim}
\everypar{\hskip 0cm plus 1fil }
\indent Later that day, ... 
\end{verbatim}
这段代码是有歧义的:{\TeX}实际上读到的是
\begin{verbatim}
\hskip 0cm plus 1fil L
\end{verbatim}
而\hbox{\n{fil L}}虽然看起来不太正常,
但实际上是一种合法的表示\n{fill}的方式
(详见第\ref{gramm}章)。
可用\cs{relax}漂亮的解决:
\begin{verbatim}
\everypar{\hskip 0cm plus 1fil\relax}
\end{verbatim}

%%\spoint[relax:cs] \cs{relax} and \cs{csname}
%\subsection{\cs{relax} and \cs{csname}}
%\label{relax:cs}
%\spoint[relax:cs] \cs{relax} and \cs{csname}
\subsection{\cs{relax}与\cs{csname}}
\label{relax:cs}

%If a \verb-\csname ... \endcsname- command forms the name
%of a previously undefined control sequence,
%that control sequence is made equal to \cs{relax},
%and the whole statement is also equivalent to \cs{relax}
%(see also page~\pageref{cs:name}).
若\verb-\csname ... \endcsname-被用于构建一个未被定义的控制序列,
该序列会被等同于\cs{relax},
同时整个构建其的语句也会被等同于\cs{relax}
(亦可见第\pageref{cs:name}页)。

%However,  this assignment of \cs{relax} is
%\altt
%only local:
%\begin{verbatim}
%{\xdef\test{\expandafter\noexpand\csname xx\endcsname}}
%\test
%\end{verbatim}
%gives an error message for an
%undefined control sequence~\cs{xx}.
但是,被赋给的\cs{relax}值
\altt
仅在局部生效:
\begin{verbatim}
{\xdef\test{\expandafter\noexpand\csname xx\endcsname}}
\test
\end{verbatim}
会给出未定义控制序列\cs{xx}的错误。

%Consider as an example the \LaTeX\ environments,
%which are delimited by
%\begin{verbatim}
%\begin{...} ... \end{...}
%\end{verbatim}
%The begin and end commands are (in essence)
%defined as follows:
%\begin{verbatim}
%\def
%\begin#1{\begingroup\csname#1\endcsname}
%\def\end#1{\csname end#1\endcsname \endgroup}
%\end{verbatim}
%Thus, for the list environment the commands
%\cs{list} and \cs{endlist} are defined, but any
%command can be used as an environment name,
%even if no corresponding \cs{end...} has been defined.
%For instance,
%\begin{verbatim}
%\begin{it} ... \end{it}
%\end{verbatim}
%is equivalent to 
%\begin{verbatim}
%\begingroup\it ... \relax\endgroup
%\end{verbatim}
%See page~\pageref{begin:end:macros} for the rationale
%behind using \cs{begingroup} and \cs{endgroup}
%instead of \cs{bgroup} and \cs{egroup}.
考虑{\LaTeX}中的环境:
其使用
\begin{verbatim}
\begin{...} ... \end{...}
\end{verbatim}
定界。
这两个表示开始和结束的命令(究其本质)
是如此定义的:
\begin{verbatim}
\def\begin#1{\begingroup\csname#1\endcsname}
\def\end#1{\csname end#1\endcsname \endgroup}
\end{verbatim}
故,对于列表环境,实际定义的是命令\cs{list}和\cs{endlist}。
但是即使没有定义对应的\cs{end...}
该命令仍可用来定义环境。
如:
\begin{verbatim}
\begin{it} ... \end{it}
\end{verbatim}
等价于
\begin{verbatim}
\begingroup\it ... \relax\endgroup
\end{verbatim}
此处使用\cs{begingroup}、\cs{endgroup}
而非\cs{bgroup}、\cs{egroup}
的原理则见第\pageref{begin:end:macros}页。

%%\spoint Preventing expansion with \cs{relax}
%\subsection{Preventing expansion with \cs{relax}}
%\spoint Preventing expansion with \cs{relax}
\subsection{使用\cs{relax}抑制展开}

%Because \cs{relax} 
%cannot be expanded, a control sequence can be prevented
%from being expanded (for instance in an \cs{edef} or a \cs{write})
%by making it temporarily equal to \cs{relax}:
%\begin{verbatim}
%{\let\somemacro=\relax \write\outfile{\somemacro}}
%\end{verbatim}
%will write the string `\cs{somemacro}' to an output file.
%It would write the expansion 
%of the macro \cs{somemacro} (or give an error message
%if the macro is undefined) if the \cs{let} statement
%had been omitted.
因为命令\cs{relax}不可展开的缘故,
(使用如\cs{edef}或\cs{write}时)
通过使某个控制序列暂时地等价于\cs{relax}
即可使该序列不被展开。
\begin{verbatim}
{\let\somemacro=\relax \write\outfile{\somemacro}}
\end{verbatim}
该代码片段则会将`\cs{somemacro}'写入到输出文件中。
若\cs{let}段被遗漏,
则会将宏\cs{somemacro}的展开结果写入到文件中
(若其未定义则给出该宏未定义之错误)。

%%\spoint[bump:relax] \TeX\ inserts a \cs{relax}
%\subsection{\TeX\ inserts a \cs{relax}}
%\label{bump:relax}
%\spoint[bump:relax] \TeX\ inserts a \cs{relax}
\subsection{{\TeX}自动插入\cs{relax}}
\label{bump:relax}

%\TeX\ itself inserts \cs{relax} on some occasions.
%For instance, \cs{relax} is inserted if \TeX\ encounters an
%\cs{or}, \cs{else}, or~\cs{fi} while still determining
%the extent of the test. 
%\begin{example}
%\begin{verbatim}
%\ifvoid1\else ... \fi
%\end{verbatim}
%is changed into
%\begin{verbatim}
%\ifvoid1\relax \else ...\fi
%\end{verbatim}
%internally.
%\end{example}
{\TeX}自己也会在某些情况下插入\cs{relax}。
具体来说,
即是当{\TeX}仍在确定文本范围时读到序列
\cs{or}、\cs{else}、或\cs{fi}时。
\begin{example}
\begin{verbatim}
\ifvoid1\else ... \fi
\end{verbatim}
在内部被转换为
\begin{verbatim}
\ifvoid1\relax \else ...\fi
\end{verbatim}
\end{example}

%Similarly, if one of the tests \cs{if}, \cs{ifcat}
%is given only one comparand, as in
%\begin{verbatim}
%\if1\else ...
%\end{verbatim}
%a \cs{relax} token is inserted. Thus this test
%is equivalent to
%\begin{verbatim}
%\if1\relax\else ...
%\end{verbatim}
类似的,
如果判断语句\cs{if}、\cs{ifcat}仅有一条分支,
如
\begin{verbatim}
\if1\else ...
\end{verbatim}
{\TeX}也会插入一个\cs{relax}。
于是上段代码等同于
\begin{verbatim}
\if1\relax\else ...
\end{verbatim}

%Another place where \cs{relax} is used is the following.
%While a control sequence is being defined in a \gr{shorthand
%definition} \ldash that is, a \gr{registerdef} or \cs{chardef}
%or \cs{mathchardef} \rdash  its meaning is temporarily made
%equal to \cs{relax}. This makes it possible to write
%\verb>\chardef\foo=123\foo>.
还有另一\cs{relax}被使用的情况。
当某一序列被使用\gr{shorthand definition}——
即,一个\gr{registerdef}或\cs{chardef}、\cs{mathchardef}——
定义时,它的含义被短暂地定义为\cs{relax}。
这使得我们能够这样写:
\verb>\chardef\foo=123\foo>。

%%\spoint The value of non-macros; \cs{the}
%\subsection{The value of non-macros; \cs{the}}
%\spoint The value of non-macros; \cs{the}
\subsection{表示非宏的值:\cs{the}}

%Expansion is a precisely defined activity in \TeX.
%\cstoidx the\par
%The full list of tokens that can be expanded
%was given above. 
%Other tokens than those in the above list may have an `expansion'
%in an informal sense. For instance one may wish to `expand'
%the \cs{parindent} into its value, say~\n{20pt}.
在{\TeX}中,展开被准确地定义。
\cstoidx the\par
所有能被展开的记号都已在上方给出。
除此之外,
其中也有能够以不严格的方式「展开」的记号:
如将\cs{parindent}展开为它的值,比方说\n{20pt}。

%Converting  the value of (among others) an
%\gr{integer parameter}, a \gr{glue parameter}, 
%\gr{dimen parameter} or a \gr{token parameter}
%into a string of character tokens is done by the expansion processor.
%The command \cs{the}
%is expanded whenever expansion is not inhibited,
%and it takes the value of various sorts of parameters.
%Its result (in most cases)
%is a string of tokens of category~12\index{category!12}, except
%that spaces have category code~10\index{category!10}.
故除上面描述的外,
展开处理器还会将一个
\gr{integer parameter}、
一个\gr{glue parameter}、
一个\gr{dimen parameter}、
抑或是\gr{token parameter}的值
转换为一串字符记号。
而\cs{the}(除非被抑制)被展开后将拿取变量作为参数。
(在大多数情况下),
其展开的结果是一串
除类别码为10\index{category!10}的空格之外
类别码为12\index{category!12}的记号。

%Here is the list of everything that can be prefixed with \cs{the}.
%\begin{description}\item [\gr{parameter} or \gr{register}]
%If the parameter or register is of type integer, glue, dimen
%or muglue,
%its value is given as a string of character tokens;
%if it is of type token list (for instance
%\cs{everypar} or \cs{toks5}), the result is a string of tokens.
%Box registers are excluded here.
%\item [\gr{codename}\gr{8-bit number}]
%See page~\pageref{codename}.
%\item [\gr{special register}]
%The integer registers \cs{prevgraf}, \cs{deadcycles}, \cs{insertpenalties}
%\cs{inputlineno}, \cs{badness}, \cs{parshape}, \cs{spacefactor}
%(only in horizontal mode), or \cs{prevdepth} (only in vertical mode).
%The dimension registers \cs{pagetotal}, \cs{pagegoal}, \cs{pagestretch},
%\handbreak \cs{pagefilstretch}, \cs{pagefillstretch}, \cs{pagefilllstretch},
%\cs{pageshrink}, or \cs{pagedepth}.
%\item [Font properties:]
%\cs{fontdimen}\gr{parameter number}\gr{font}, 
%\cs{skew\-char}\gr{font},
%\cs{hy\-phen\-char}\gr{font}.
%\item [Last quantities:]
%\cs{lastpenalty}, \cs{lastkern}, \cs{lastskip}.
%\item [\gr{defined character}]
%Any control sequence defined by \cs{chardef} or \cs{mathchardef};
%the result is the decimal value.
%\end{description}
%In some cases \cs{the} can give a control sequence token 
%or list of such tokens.
%\begin{description}\item [\gr{font}]
%The result is the control sequence that stands for the
%font.
%\item [\gr{token variable}]
%Token list registers and \gr{token parameter}s can be prefixed
%with \cs{the}; the result is their contents.
%\end{description}
所有能以\cs{the}作为前缀的如下所列:
\begin{description}\item [\gr{parameter}或\gr{register}]
如果寄存器的参量是一种整数、伸缩胶、长度或数学模式中的伸缩胶,
则其值以一串字符记号的形式被给出;若为记号列表(如
\cs{everypar}或\cs{toks5}),
结果则会是一串记号。
盒子寄存器在这里不予考虑。
\item [\gr{codename}\gr{8-bit number}]
见第\pageref{codename}页。
\item [\gr{special register}]
整数寄存器如\cs{prevgraf}、\cs{deadcycles}、\cs{insertpenalties}、
\cs{inputlineno}、\cs{badness}、\cs{parshape}、\cs{spacefactor}
(仅出现在水平模式中)、和\cs{prevdepth}(仅出现在竖直模式中)。
长度寄存器如\cs{pagetotal}、\cs{pagegoal}、\cs{pagestretch}、
\handbreak \cs{pagefilstretch}、\cs{pagefillstretch}、
\cs{pagefilllstretch}、\cs{pageshrink}、和\cs{pagedepth}。
\item [字体属性:]
\cs{fontdimen}\gr{parameter number}\gr{font}、
\cs{skew\-char}\gr{font}、和
\cs{hy\-phen\-char}\gr{font}。
\item [前量:]
\cs{lastpenalty}、\cs{lastkern}、和\cs{lastskip}。
\item [\gr{defined character}]
任何使用\cs{chardef}或\cs{mathchardef}定义的控制序列;
结果是十进制数。
\end{description}
有时,\cs{the}会返回一个控制序列记号或是前者的列表。
\begin{description}\item [\gr{font}]
结果是代表该字体的控制序列。
\item [\gr{token variable}]
记号列表寄存器以及\gr{token parameter}可以使用
\cs{the}为前缀;展开结果是它们的内容。
\end{description}

%Let us consider an example of the use of \cs{the}.
%If in a file that is to be \cs{input} the
%category code of a character, say the at~sign, is changed,
%one could write
%\begin{verbatim}
%\edef\restorecat{\catcode`@=\the\catcode`@}
%\end{verbatim}
%and call \cs{restorecat} at the end of the file.
%If the category code was~11, \cs{restorecat}
%is defined equivalent to
%\begin{verbatim}
%\catcode`@=11
%\end{verbatim}
%See page~\pageref{store:cat} for more elaborate macros
%for saving and restoring catcodes.
来思考一个使用\cs{the}的例子。
若有一个将使用\cs{input}读入的
某个字符(不妨让它为\texttt{\@})
的类别码被改变了的文件。
我们可以写出
\begin{verbatim}
\edef\restorecat{\catcode`@=\the\catcode`@}
\end{verbatim}
并在该文件的末尾写上\cs{restorecat}。
若类别为11,则\cs{restorecat}的定义等同于
\begin{verbatim}
\catcode`@=11
\end{verbatim}
更复杂的被用来保存并恢复类别码的宏可见
第\pageref{store:cat}页。


%%\point Examples
%\section{Examples}
%\point Examples
\section{示例}

%%\spoint Expanding after
%\subsection{Expanding after}
%\spoint Expanding after
\subsection{滞后展开
\footnote{译注:实际上是越过一个记号展开后面的记号,
这里为了使章节名紧凑,且原文为「expand after」,
故译为「滞后展开」。
其意指将目前的记号滞后,先行展开其后的记号。}}

%The most obvious use of \cs{expandafter} is to reach over
%a control sequence:
%\begin{verbatim}
%\def\stepcounter
%    #1{\expandafter\advance\csname 
%             #1:counter\endcsname 1\relax}
%\stepcounter{foo}
%\end{verbatim}
%Here the \cs{expandafter} lets the \cs{csname} command form
%the control sequence \cs{foo:counter}; after \cs{expandafter}
%is finished the statement has reduced to
%\begin{verbatim}
%\advance\foo:counter 1\relax
%\end{verbatim}
%It is possible to reach over tokens other than control sequences: in
%\begin{verbatim}
%\uppercase\expandafter{\romannumeral \year}
%\end{verbatim}
%it expands  \cs{romannumeral} on the other side of the opening
%brace.
对于控制序列\cs{expandafter},
最显而易见的使用方式莫过于提前展开一个(位于后部)控制序列:
\begin{verbatim}
\def\stepcounter
    #1{\expandafter\advance\csname 
             #1:counter\endcsname 1\relax}
\stepcounter{foo}
\end{verbatim}
此处使用的\cs{expandafter}使得\cs{csname}得以构建完毕
\cs{foo:counter}控制序列;
在\cs{expandafter}展开后,
该声明被缩减为
\begin{verbatim}
\advance\foo:counter 1\relax
\end{verbatim}
越过不属于控制序列的记号也是可能的,在
\begin{verbatim}
\uppercase\expandafter{\romannumeral \year}
\end{verbatim}
中,它先行展开左花括号右侧的\cs{romannumeral}宏。

%You can expand after two control sequences:
%\begin{verbatim}
%\def\globalstepcounter
%    #1{\expandafter\global\expandafter\advance
%             \csname #1:counter\endcsname 1\relax}
%\end{verbatim}
%If you think of \cs{expandafter} as reversing the evaluation
%order of {\sl two\/} control sequences, you can reverse
%{\sl three\/} by
%\begin{verbatim}
%\expandafter\expandafter\expandafter\a\expandafter\b\c 
%\end{verbatim}
%which reaches across the three control sequences
%\begin{verbatim}
%            \expandafter            \a            \b 
%\end{verbatim} 
%to expand \cs{c} first.
自然,你也可以越过两个控制序列展开:
\begin{verbatim}
\def\globalstepcounter
    #1{\expandafter\global\expandafter\advance
             \csname #1:counter\endcsname 1\relax}
\end{verbatim}
若你觉得只能使用\cs{expandafter}颠倒
{\sl 两个\/}控制序列的执行顺序,
你还可以颠倒{\sl 三个的\/},使用
\begin{verbatim}
\expandafter\expandafter\expandafter\a\expandafter\b\c 
\end{verbatim}
即可最先展开\cs{c}而得到
\begin{verbatim}
            \expandafter            \a            \b 
\end{verbatim} 
三个控制序列作为结果。

%There is even an unexpected use for \cs{expandafter} in
%conditionals;
%with
%\begin{verbatim}
%\def\bold#1{{\bf #1}}
%\end{verbatim}
%the sequence
%\begin{verbatim}
%\ifnum1>0 \bold \fi {word}
%\end{verbatim}
%will not give a boldface `word', but
%\begin{verbatim}
%\ifnum1>0 \expandafter\bold \fi {word}
%\end{verbatim}
%will.
%The \cs{expandafter} lets \TeX\ see the \cs{fi} and remove it
%before it tackles the macro \cs{bold}
%(see also page~\pageref{after:cond}).
在条件判断中,\cs{expandafter}还有更预料不到的作用;如下
\begin{verbatim}
\def\bold#1{{\bf #1}}
\end{verbatim}
代码中,序列
\begin{verbatim}
\ifnum1>0 \bold \fi {word}
\end{verbatim}
并不会让「word」被加粗,而
\begin{verbatim}
\ifnum1>0 \expandafter\bold \fi {word}
\end{verbatim}
可以(使其被加粗)。
控制序列\cs{expandafter}会让{\TeX}先一步看到\cs{fi}
并在它构建宏\cs{bold}的参数之前移除之
(亦见第~\pageref{after:cond}页)。

%%\spoint Defining inside an \cs{edef}
%\subsection{Defining inside an \cs{edef}}
%\spoint Defining inside an \cs{edef}
\subsection{\cs{edef}内部的定义}

%There is one \TeX\ command that is executed instead of
%expanded that is worth pointing out explicitly:
%the primitive command \cs{def} (and all other \gr{def} commands)
%is not expanded.
有一个以执行代替展开的{\TeX}命令值得特别指出:
元命令\cs{def}(以及所有其他的\gr{def}命令)
不会被展开。

%Thus the call
%\begin{verbatim}
%\edef\next{\def\thing{text}}
%\end{verbatim}
%will give an `undefined
%control sequence' for \cs{thing}, even though after
%\cs{def} expansion is ordinarily inhibited (see page~\pageref{noexp:list}).
%After
%\begin{verbatim}
%\edef\next{\def\noexpand\thing{text}}
%\end{verbatim}
%the `meaning' of \cs{next} will be
%\begin{verbatim}
%macro: \def \thing {text}
%\end{verbatim}
%The definition
%\begin{verbatim}
%\edef\next{\def\noexpand\thing{text}\thing}
%\end{verbatim}
%will again give an `undefined control sequence' for \cs{thing}
%(this time on its second occurrence),
%as it will only be defined when \cs{next} is called,
%not when \cs{next} is defined.
所以以下写法
\begin{verbatim}
\edef\next{\def\thing{text}}
\end{verbatim}
会给出一个宏\cs{thing}未定义的错误,
即使一般情况下\cs{def}后的展开被抑制
(参考第~\pageref{noexp:list}页)。
而若使用这种写法
\begin{verbatim}
\edef\next{\def\noexpand\thing{text}}
\end{verbatim}
\cs{next}的意义会是
\begin{verbatim}
macro: \def \thing {text}
\end{verbatim}
而以下定义
\begin{verbatim}
\edef\next{\def\noexpand\thing{text}\thing}
\end{verbatim}
又会给出宏\cs{thing}未定义的错误
(但这次是由于它的第二次出现),
因为\cs{thing}只有在\cs{next}被实际调用时才会被定义
(在定义\cs{next}宏时是不会定义\cs{thing}的)。


%%\spoint[expand:write] Expansion and \cs{write}
%\subsection{Expansion and \cs{write}}
%\label{expand:write}
%\spoint[expand:write] Expansion and \cs{write}
\subsection{展开与\cs{write}}
\label{expand:write}

%The argument token list of \csidx{write} is treated in much
%the same way as the replacement text of an \cs{edef};
%that is, expandable control sequences and active characters
%are completely expanded. Unexpandable control sequences
%are treated by \cs{write} as if they are prefixed
%by \cs{string}.
\csidx{write}中的作为参量的记号列表
几乎被以同\cs{edef}的替换文本相同的方法被处理;
即,可展开的控制序列和活动字符被完全展开。
在\cs{write}中不可展开的控制序列,
被像有\cs{string}作前缀那样被处理。

%Because of the expansion performed by \cs{write},
%some care has to be taken when outputting control
%sequences with \cs{write}.
%Even more complications arise from the fact that 
%the expansion of the argument of \cs{write} is only performed
%when it is shipped out. Here follows a worked-out
%example.
在使用\cs{write}输出控制序列时,
需要注意某些情况。
其更复杂的原因来自于\cs{write}参量的展开
仅在它被输出的时候发生。
后面则是一个有启发的例子。
%Suppose \cs{somecs} is a macro, and you
%want to write the string 
%\begin{disp}\verb-\def\othercs-\lb {\italic the expansion of \cs{somecs}}\rb
%\end{disp}
%to a file.
假设有一个宏\cs{somecs},
同时你希望将以下序列
\begin{disp}\verb-\def\othercs-\lb {\italic \cs{somecs}的展开}\rb
\end{disp}
写入到文件中。

%The first attempt is
%\begin{verbatim}
%\write\myfile{\def\othercs{\somecs}}
%\end{verbatim}
%This gives an error `undefined control sequence' for \cs{othercs},
%\altt
%because the \cs{write} will try to expand that token. 
%Note that the \cs{somecs} is also  expanded, 
%so that part is right.
首先想到的方法是
\begin{verbatim}
\write\myfile{\def\othercs{\somecs}}
\end{verbatim}
然而,这会提示\cs{othercs}未定义之错误,
\altt
因为\cs{write}会尝试展开其。
好事是\cs{somecs}也被展开,所以该部分目的被达到了。

%The next attempt is
%\begin{verbatim}
%\write\myfile{\def\noexpand\othercs{\somecs}}
%\end{verbatim}
%This is almost right, but not quite. The
%statement written is
%\begin{disp}\verb>\def\othercs>\lb{\italic expansion of \cs{somecs}}\rb\end{disp}
%which looks right.
随后的实现是
\begin{verbatim}
\write\myfile{\def\noexpand\othercs{\somecs}}
\end{verbatim}
它几乎就对了,但不完全对。
我们写出的定义是
\begin{disp}\verb>\def\othercs>\lb{\italic expansion of \cs{somecs}}\rb\end{disp}
它看起来无可挑剔。

%However, writes \ldash and the expansion of their argument \rdash 
%are not executed
%on the spot, but saved until the part of the page on which
%they occur is shipped out (see Chapter~\ref{io}).
%So, in the meantime, the value of \cs{somecs} may have
%changed. In other words, the value written may not be the
%value at the time the \cs{write} command was given.
%Somehow, therefore, the current expansion must be
%inserted in the write command.
但是,写入——当然也包括写入参量的展开——
并非即时进行的,
它是在(它出现的)该部分页面输出时才进行的(参考第\ref{io}章)。
所以,那时\cs{somecs}的值可能会发生变化。
这也就意味着写入到文件中的值不一定是\cs{write}命令被给出时的值。
所以,必须以某种方式使用当前的展开结果作为\cs{write}的参量。

%The following is an attempt at repair:
%\begin{verbatim}
%\edef\act{\write\myfile{\def\noexpand\othercs{\somecs}}}
%\act
%\end{verbatim} 
%Now the  write command will be
%\begin{disp}\verb>\write\myfile{\def\othercs{>\italic value of\/
%     \verb>\somecs}}>\end{disp}
%The \cs{noexpand} prevented the \cs{edef} from expanding
%the \cs{othercs}, but after the definition it has disappeared,
%so that execution of the write will again give an undefined control
%sequence. The final solution is
%\begin{verbatim}
%\edef\act{\write\myfile
%          {\def \noexpand\noexpand \noexpand\othercs{\somecs}}}
%\act
%\end{verbatim} 
%In this case the write command caused by the expansion of \cs{act}
%will be
%\begin{disp}\verb>\write\myfile{\def\noexpand\othercs>\lb
%     {\italic current value of \cs{somecs}}\rb\end{disp}
%and the string actually written is
%\begin{disp}\verb>\def\othercs>\lb
%     {\italic current value of \cs{somecs}}\rb\end{disp}
%This mechanism is the basis for cross-referencing
%macros in several macro packages.
于是我们尝试
\begin{verbatim}
\edef\act{\write\myfile{\def\noexpand\othercs{\somecs}}}
\act
\end{verbatim} 
这时写入的命令会是
\begin{disp}\verb>\write\myfile{\def\othercs{>\italic
     \verb>\somecs}}>的值\end{disp}
命令\cs{noexpand}抑制了\cs{edef}对\cs{othercs}的展开,
但在对\cs{act}定义后它就消失了,
这就导致在写入时还将出现未定义控制序列的错误。
于是我们最终的解决方案是如下的
\begin{verbatim}
\edef\act{\write\myfile
          {\def \noexpand\noexpand \noexpand\othercs{\somecs}}}
\act
\end{verbatim} 
此时\cs{act}开始的写入命令实际上是
\begin{disp}\verb>\write\myfile{\def\noexpand\othercs>\lb
     {\italic \cs{somecs}的当前值}\rb\end{disp}
而实际写入的序列是
\begin{disp}\verb>\def\othercs>\lb
     {\italic \cs{somecs}的当前值}\rb\end{disp}
以上的机制即为一些宏集交叉索引宏的底层原理。


%%\spoint Controlled expansion inside an \cs{edef}
%\subsection{Controlled expansion inside an \cs{edef}}
%\spoint Controlled expansion inside an \cs{edef}
\subsection{Controlled expansion inside an \cs{edef}}

%Sometimes you may need an \cs{edef} to evaluate current
%\howto Control expansion inside an \cs{edef}\par
%conditions, but you want to expand something in the replacement
%text only to a certain level. Suppose that
%\begin{verbatim}
%\def\a{\b} \def\b{c} \def\d{\e} \def\e{f}
%\end{verbatim}
%is given, and you want to define \cs{g} as \cs{a} expanded
%one step, followed by \cs{d} fully expanded. The following
%works:
%\begin{verbatim}
%\edef\g{\expandafter\noexpand\a \d}
%\end{verbatim}
%Explanation: the \cs{expandafter} reaches over the \cs{noexpand}
%to expand \cs{a} one step, after which the 
%sequence \verb-\noexpand\b- is left.
Sometimes you may need an \cs{edef} to evaluate current
\howto Control expansion inside an \cs{edef}\par
conditions, but you want to expand something in the replacement
text only to a certain level. Suppose that
\begin{verbatim}
\def\a{\b} \def\b{c} \def\d{\e} \def\e{f}
\end{verbatim}
is given, and you want to define \cs{g} as \cs{a} expanded
one step, followed by \cs{d} fully expanded. The following
works:
\begin{verbatim}
\edef\g{\expandafter\noexpand\a \d}
\end{verbatim}
Explanation: the \cs{expandafter} reaches over the \cs{noexpand}
to expand \cs{a} one step, after which the 
sequence \verb-\noexpand\b- is left.

%This trick comes in handy when you need to
%construct a control sequence with \cs{csname} inside 
%an \cs{edef}. The following sequence inside an \cs{edef}
%\begin{verbatim}
%\expandafter\noexpand\csname name\endcsname
%\end{verbatim}
%will expand exactly to \cs{name}, but not further.
%As an example, suppose
%\begin{verbatim}
%\def\condition{true}
%\end{verbatim}
%has been given, then
%\begin{verbatim}
%\edef\setmycondition{\expandafter\noexpand
%           \csname mytest\condition\endcsname}
%\end{verbatim}
%will let \cs{setmycondition} expand to \cs{mytesttrue}.
This trick comes in handy when you need to
construct a control sequence with \cs{csname} inside 
an \cs{edef}. The following sequence inside an \cs{edef}
\begin{verbatim}
\expandafter\noexpand\csname name\endcsname
\end{verbatim}
will expand exactly to \cs{name}, but not further.
As an example, suppose
\begin{verbatim}
\def\condition{true}
\end{verbatim}
has been given, then
\begin{verbatim}
\edef\setmycondition{\expandafter\noexpand
           \csname mytest\condition\endcsname}
\end{verbatim}
will let \cs{setmycondition} expand to \cs{mytesttrue}.

%%\spoint Multiple prevention of expansion
%\subsection{Multiple prevention of expansion}
%\spoint Multiple prevention of expansion
\subsection{Multiple prevention of expansion}

%As   was pointed out above, prefixing  a command with
%\cs{noexpand} prevents its expansion in commands
%such as \cs{edef} and~\cs{write}. However, if a sequence of tokens
%passes through more than one expanding command
%stronger measures are needed.
As   was pointed out above, prefixing  a command with
\cs{noexpand} prevents its expansion in commands
such as \cs{edef} and~\cs{write}. However, if a sequence of tokens
passes through more than one expanding command
stronger measures are needed.

%The following trick can be used:
%in order to protect a command against expansion
%it can be prefixed with \csidx{protect}.
%During the stages of processing where expansion is
%not desired the definition of \cs{protect} is
%\begin{verbatim}
%\def\protect{\noexpand\protect\noexpand}
%\end{verbatim}
%Later on, when the command is actually needed,
%\cs{protect} is defined as 
%\begin{verbatim}
%\def\protect{}
%\end{verbatim}
The following trick can be used:
in order to protect a command against expansion
it can be prefixed with \csidx{protect}.
During the stages of processing where expansion is
not desired the definition of \cs{protect} is
\begin{verbatim}
\def\protect{\noexpand\protect\noexpand}
\end{verbatim}
Later on, when the command is actually needed,
\cs{protect} is defined as 
\begin{verbatim}
\def\protect{}
\end{verbatim}

%Why does this work? The expansion of
%\begin{verbatim}
%\protect\somecs
%\end{verbatim}
%is at first
%\begin{verbatim}
%\noexpand\protect\noexpand\somecs
%\end{verbatim}
%Inside an \cs{edef} this sequence is expanded further,
%and the subsequent expansion is
%\begin{verbatim}
%\protect\somecs
%\end{verbatim}
%That is, the expansion is equal to the original sequence.
Why does this work? The expansion of
\begin{verbatim}
\protect\somecs
\end{verbatim}
is at first
\begin{verbatim}
\noexpand\protect\noexpand\somecs
\end{verbatim}
Inside an \cs{edef} this sequence is expanded further,
and the subsequent expansion is
\begin{verbatim}
\protect\somecs
\end{verbatim}
That is, the expansion is equal to the original sequence.


%%\spoint More examples with \cs{relax}
%\subsection{More examples with \cs{relax}}
%\spoint More examples with \cs{relax}
\subsection{More examples with \cs{relax}}

%Above, a first example was given in which \cs{relax} served
%to prevent \TeX\ from scanning too far.
%Here are some more examples, using \cs{relax} to bound
%numbers.
Above, a first example was given in which \cs{relax} served
to prevent \TeX\ from scanning too far.
Here are some more examples, using \cs{relax} to bound
numbers.

%After
%\begin{verbatim}
%\countdef\pageno=0 \pageno=1
%\def\Par{\par\penalty200}
%\end{verbatim}
%the sequence
%\begin{verbatim}
%\Par\number\pageno
%\end{verbatim}
%is misunderstood as
%\begin{verbatim}
%\par\penalty2001
%\end{verbatim}
%In this case it is sufficient to define
%\begin{verbatim}
%\def\Par{\par\penalty200 }
%\end{verbatim}
%as an \gr{optional space} is allowed to follow a number.
After
\begin{verbatim}
\countdef\pageno=0 \pageno=1
\def\Par{\par\penalty200}
\end{verbatim}
the sequence
\begin{verbatim}
\Par\number\pageno
\end{verbatim}
is misunderstood as
\begin{verbatim}
\par\penalty2001
\end{verbatim}
In this case it is sufficient to define
\begin{verbatim}
\def\Par{\par\penalty200 }
\end{verbatim}
as an \gr{optional space} is allowed to follow a number.

%Sometimes, however, such a simple escape is not possible.
%Consider the definition
%\begin{verbatim}
%\def\ifequal#1#2{\ifnum#1=#2 1\else 0\fi}
%\end{verbatim}
%The question is whether the space after \verb-#2-
%is necessary, superfluous, or simply wrong.
%Calls such as \verb-\ifequal{27}{28}- that compare two
%numbers (denotations) will correctly give \n1 or~\n0,
%and the space is necessary to prevent misinterpretation.
Sometimes, however, such a simple escape is not possible.
Consider the definition
\begin{verbatim}
\def\ifequal#1#2{\ifnum#1=#2 1\else 0\fi}
\end{verbatim}
The question is whether the space after \verb-#2-
is necessary, superfluous, or simply wrong.
Calls such as \verb-\ifequal{27}{28}- that compare two
numbers (denotations) will correctly give \n1 or~\n0,
and the space is necessary to prevent misinterpretation.

%However, \verb-\ifequal\somecounter\othercounter- will
%give \n{\char 32 1} if the counters are equal; in this
%case the space could have been dispensed with.
%The solution that works in both cases is
%\begin{verbatim}
%\def\ifequal#1#2{\ifnum#1=#2\relax 1\else 0\fi}
%\end{verbatim}
%Note that \cs{relax} is not expanded, so
%\begin{verbatim}
%\edef\foo{1\ifequal\counta\countb}
%\end{verbatim}
%will define \cs{foo} as either \verb-1\relax1- or~\n{10}.
However, \verb-\ifequal\somecounter\othercounter- will
give \n{\char 32 1} if the counters are equal; in this
case the space could have been dispensed with.
The solution that works in both cases is
\begin{verbatim}
\def\ifequal#1#2{\ifnum#1=#2\relax 1\else 0\fi}
\end{verbatim}
Note that \cs{relax} is not expanded, so
\begin{verbatim}
\edef\foo{1\ifequal\counta\countb}
\end{verbatim}
will define \cs{foo} as either \verb-1\relax1- or~\n{10}.

%\subsection{Example: category code saving and restoring}
%\label{store:cat}
\subsection{Example: category code saving and restoring}
\label{store:cat}

%In many applications it is necessary to change
%\howto Save and restore category codes\par
%the category code of a certain character during the
%execution of some piece of code. If the writer of
%that code is also the writer of the surrounding code,
%s/he can simply change the category code back and forth.
%However, if the surrounding code is by another author,
%the value of the category code will have to be stored
%and restored.
In many applications it is necessary to change
\howto Save and restore category codes\par
the category code of a certain character during the
execution of some piece of code. If the writer of
that code is also the writer of the surrounding code,
s/he can simply change the category code back and forth.
However, if the surrounding code is by another author,
the value of the category code will have to be stored
and restored.

%Thus one would like to write
%\begin{verbatim}
%\storecat@
%... some code ...
%\restorecat@
%\end{verbatim}
%or maybe
%\begin{verbatim}
%\storecat\%
%\end{verbatim}
%for characters that
%are possibly a comment character (or ignored or invalid).
%\alt
%The basic idea is to define
%\begin{verbatim}
%\def\storecat#1{%
%    \expandafter\edef\csname restorecat#1\endcsname
%        {\catcode`#1=\the\catcode`#1}}
%\end{verbatim}
%so that, for instance, \verb>\storecat$> will define
%the single control sequence `\verb>\restorecat$>'
%(one control sequence) as
%\begin{verbatim}
%\catcode`$=3
%\end{verbatim}
%The macro \cs{restorecat} can then be implemented as
%\begin{verbatim}
%\def\restorecat#1{%
%    \csname restorecat#1\endcsname}
%\end{verbatim}
%Unfortunately, things are not so simple.
Thus one would like to write
\begin{verbatim}
\storecat@
... some code ...
\restorecat@
\end{verbatim}
or maybe
\begin{verbatim}
\storecat\%
\end{verbatim}
for characters that
are possibly a comment character (or ignored or invalid).
\alt
The basic idea is to define
\begin{verbatim}
\def\storecat#1{%
    \expandafter\edef\csname restorecat#1\endcsname
        {\catcode`#1=\the\catcode`#1}}
\end{verbatim}
so that, for instance, \verb>\storecat$> will define
the single control sequence `\verb>\restorecat$>'
(one control sequence) as
\begin{verbatim}
\catcode`$=3
\end{verbatim}
The macro \cs{restorecat} can then be implemented as
\begin{verbatim}
\def\restorecat#1{%
    \csname restorecat#1\endcsname}
\end{verbatim}
Unfortunately, things are not so simple.

%The problems occur with active characters, because these
%are expanded inside the \cs{csname} \verb>...> \cs{endcsname} pairs.
%One might be tempted to write \verb>\noexpand#1> everywhere,
%but this is wrong. As was explained above, this is essentially
%equal to \cs{relax}, which is unexpandable, and will therefore
%lead to an error message when it appears between
%\cs{csname} and \cs{endcsname}. The proper solution is then
%to use \verb>\string#1>. For the case where the argument
%was given as a control symbol (for example~\verb>\%>),
%the escape character has to be switched off for a while.
% 
%Here are the complete macros. The \cs{storecat} macro
%gives its argument a default category code of~12.
%\begin{verbatim}
%\newcount\tempcounta % just a temporary
%\def\csarg#1#2{\expandafter#1\csname#2\endcsname}
%\def\storecat#1%
%   {\tempcounta\escapechar \escapechar=-1
%    \csarg\edef{restorecat\string#1}%
%          {\catcode`\string#1=
%                \the\catcode\expandafter`\string#1}%
%    \catcode\expandafter`\string#1=12\relax
%    \escapechar\tempcounta}
%\def\restorecat#1%
%   {\tempcounta\escapechar \escapechar=-1
%    \csname restorecat\string#1\endcsname
%    \escapechar\tempcounta}
%\end{verbatim}
The problems occur with active characters, because these
are expanded inside the \cs{csname} \verb>...> \cs{endcsname} pairs.
One might be tempted to write \verb>\noexpand#1> everywhere,
but this is wrong. As was explained above, this is essentially
equal to \cs{relax}, which is unexpandable, and will therefore
lead to an error message when it appears between
\cs{csname} and \cs{endcsname}. The proper solution is then
to use \verb>\string#1>. For the case where the argument
was given as a control symbol (for example~\verb>\%>),
the escape character has to be switched off for a while.
 
Here are the complete macros. The \cs{storecat} macro
gives its argument a default category code of~12.
\begin{verbatim}
\newcount\tempcounta % just a temporary
\def\csarg#1#2{\expandafter#1\csname#2\endcsname}
\def\storecat#1%
   {\tempcounta\escapechar \escapechar=-1
    \csarg\edef{restorecat\string#1}%
          {\catcode`\string#1=
                \the\catcode\expandafter`\string#1}%
    \catcode\expandafter`\string#1=12\relax
    \escapechar\tempcounta}
\def\restorecat#1%
   {\tempcounta\escapechar \escapechar=-1
    \csname restorecat\string#1\endcsname
    \escapechar\tempcounta}
\end{verbatim}

%%\spoint Combining \cs{aftergroup} and boxes
%\subsection{Combining \cs{aftergroup} and boxes}
%\spoint Combining \cs{aftergroup} and boxes
\subsection{Combining \cs{aftergroup} and boxes}

%At times, one wants to construct a box and immediately
%after it has been constructed to
%do something with it. The \cs{aftergroup} command
%can be used to put both the commands creating the box,
%and the ones handling it, in one macro.
At times, one wants to construct a box and immediately
after it has been constructed to
do something with it. The \cs{aftergroup} command
can be used to put both the commands creating the box,
and the ones handling it, in one macro.

%As an example, here is a macro 
%\cs{textvcenter}\label{text:vcenter}\
%which defines a variant of the \cs{vcenter} box
%\howto \cs{vcenter} outside math mode\par
%(see page~\pageref{vcenter}\label{tvcenter})
%that can be used outside math mode.
%\begin{verbatim}
%\def\textvcenter
%   {\hbox \bgroup$\everyvbox{\everyvbox{}%
%    \aftergroup$\aftergroup\egroup}\vcenter}
%\end{verbatim}
%The idea is that the macro inserts \verb>\hbox {$>,
%and that the matching \verb>$}> gets inserted
%by the \cs{aftergroup} commands. In order to get the 
%\cs{aftergroup} commands inside the box, an
%\cs{everyvbox} command is  used.
As an example, here is a macro 
\cs{textvcenter}\label{text:vcenter}\
which defines a variant of the \cs{vcenter} box
\howto \cs{vcenter} outside math mode\par
(see page~\pageref{vcenter}\label{tvcenter})
that can be used outside math mode.
\begin{verbatim}
\def\textvcenter
   {\hbox \bgroup$\everyvbox{\everyvbox{}%
    \aftergroup$\aftergroup\egroup}\vcenter}
\end{verbatim}
The idea is that the macro inserts \verb>\hbox {$>,
and that the matching \verb>$}> gets inserted
by the \cs{aftergroup} commands. In order to get the 
\cs{aftergroup} commands inside the box, an
\cs{everyvbox} command is  used.

%This macro can even be used with a \gr{box specification}
%(see page~\pageref{box:spec}), for example
%\begin{verbatim}
%\textvcenter spread 8pt{\hbox{a}\vfil\hbox{b}}
%\end{verbatim}
%and because it  is really just an \cs{hbox}, it can also
%be used in a \cs{setbox} assignment.
This macro can even be used with a \gr{box specification}
(see page~\pageref{box:spec}), for example
\begin{verbatim}
\textvcenter spread 8pt{\hbox{a}\vfil\hbox{b}}
\end{verbatim}
and because it  is really just an \cs{hbox}, it can also
be used in a \cs{setbox} assignment.

%\subsection{More expansion}
\subsection{More expansion}

%There is a particular charm to macros that work
%purely by expansion. See the articles by
%\cite{E2}, \cite{Jeffrey:lists}, and~\cite{Maus2}.
There is a particular charm to macros that work
purely by expansion. See the articles by
\cite{E2}, \cite{Jeffrey:lists}, and~\cite{Maus2}.

%\endofchapter
%%%%% end of input file [expand]
\endofchapter
%%%% end of input file [expand]

\end{document}
