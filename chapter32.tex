% -*- coding: utf-8 -*-
\documentclass{book}

% -*- coding: utf-8 -*-

\usepackage[b5paper,text={5in,8in},centering]{geometry}
\usepackage{amsmath}
\usepackage{amssymb}
\usepackage[heading = false, scheme = plain]{ctex}
% \usepackage[CJKchecksingle]{xeCJK}
\setmainfont[Mapping=tex-text]{TeX Gyre Schola}
%\setsansfont{URW Gothic L Book}
%\setmonofont{Nimbus Mono L}
% \setCJKmainfont[BoldFont=FandolHei,ItalicFont=FandolKai]{FandolSong}
% \setCJKsansfont{FandolHei}
% \setCJKmonofont{FandolFang}
\xeCJKsetup{PunctStyle = kaiming}

\linespread{1.25}
\setlength{\parindent}{2em}
\setlength{\parskip}{0.5ex}
\usepackage{indentfirst}

\usepackage{xcolor}
\definecolor{myblue}{rgb}{0,0.2,0.6}

\usepackage{titlesec}
\titleformat{\chapter}
    {\normalfont\Huge\sffamily\color{myblue}}
    {第\thechapter 章}
    {1em}
    {}
%\titlespacing{\chapter}{0pt}{50pt}{40pt}
\titleformat{\section}
    {\normalfont\Large\sffamily\color{myblue}}
    {\thesection}
    {1em}
    {}
%\titlespacing{\section}{0pt}{3.5ex plus 1ex minus .2ex}{2.3ex plus .2ex}
\titleformat{\subsection}
    {\normalfont\large\sffamily\color{myblue}}
    {\thesubsection}
    {1em}
    {}
%\titlespacing{\subsection}{0pt}{3.25ex plus 1ex minus .2ex}{1.5ex plus .2ex}
%
\newpagestyle{special}[\small\sffamily]{
  \headrule
  \sethead[\usepage][][\chaptertitle]
  {\chaptertitle}{}{\usepage}}
\newpagestyle{main}[\small\sffamily]{
  \headrule
  \sethead[\usepage][][第\thechapter 章\quad\chaptertitle]
  {\thesection\quad\sectiontitle}{}{\usepage}}

\usepackage{titletoc}
%\setcounter{tocdepth}{1}
%\titlecontents{标题层次}[左间距]{上间距和整体格式}{标题序号}{标题内容}{指引线和页码}[下间距]
\titlecontents{chapter}[1.5em]{\vspace{.5em}\bfseries\sffamily}{\color{myblue}\contentslabel{1.5em}}{}
    {\titlerule*[20pt]{$\cdot$}\contentspage}[]
\titlecontents{section}[4.5em]{\sffamily}{\color{myblue}\contentslabel{3em}}{}
    {\titlerule*[20pt]{$\cdot$}\contentspage}[]
%\titlecontents{subsection}[8.5em]{\sffamily}{\contentslabel{4em}}{}
%    {\titlerule*[20pt]{$\cdot$}\contentspage}

\usepackage{enumitem}
\setlist{topsep=2pt,itemsep=2pt,parsep=1pt,leftmargin=\parindent}

\usepackage{fancyvrb}
\DefineVerbatimEnvironment{verbatim}{Verbatim}
  {xleftmargin=2em,baselinestretch=1,formatcom=\color{teal}\upshape}

\usepackage{etoolbox}
\makeatletter
\preto{\FV@ListVSpace}{\topsep=2pt \partopsep=0pt }
\makeatother

\usepackage[colorlinks,plainpages,pagebackref]{hyperref}
\hypersetup{
   pdfstartview={FitH},
   citecolor=teal,
   linkcolor=myblue,
   urlcolor=black,
   bookmarksnumbered
}

\usepackage{comment,makeidx,multicol}

%\usepackage{german}
%% german
%\righthyphenmin=3
%\mdqoff
%\captionsenglish
\usepackage[english]{babel}
{\catcode`"=13 \gdef"#1{\ifx#1"\discretionary{}{}{}\fi\relax}}
\def\mdqon{\catcode`"=13\relax}
\def\mdqoff{\catcode`"=12\relax}
\makeindex
\hyphenation{ex-em-pli-fies}

\newdimen\tempdima \newdimen\tempdimb

% these are fine
\def\handbreak{\\ \message{^^JManual break!!!!^^J}}
\def\nl{\protect\\}\def\n#1{{\tt #1}}
\protected\def\cs#1{\texttt{\textbackslash#1}}
\pdfstringdefDisableCommands{\def\cs#1{\textbackslash#1}}
\let\csc\cs
\def\lb{{\tt\char`\{}}\def\rb{{\tt\char`\}}}
\def\gr#1{\texorpdfstring{$\langle$#1$\rangle$}{<#1>}} %\def\gr#1{$\langle$#1$\rangle$}
\def\marg#1{{\tt \{}#1{\tt \}}}
\def\oarg#1{{\tt [}#1{\tt }}
\def\key#1{{\tt#1}}
\def\alt{}\def\altt{}%this way in manstijl
\def\ldash{\unskip\ ——\nobreak\ \ignorespaces}
\def\rdash{\unskip\nobreak\ ——\ \ignorespaces}
% check these
\def\hex{{\tt"}}
\def\ascii{{\sc ascii}}
\def\ebcdic{{\sc ebcdic}}
\def\IniTeX{Ini\TeX}\def\LamsTeX{LAMS\TeX}\def\VirTeX{Vir\TeX}
\def\AmsTeX{Ams\TeX}
\def\TeXbook{the \TeX\ book}\def\web{{\sc web}}
% needs major thinking
\newenvironment{myquote}{\list{}{%
    \topsep=2pt \partopsep=0pt%
    \leftmargin=\parindent \rightmargin=\parindent
    }\item[]}{\endlist}
\newenvironment{disp}{\begin{myquote}}{\end{myquote}}
\newenvironment{Disp}{\begin{myquote}}{\end{myquote}}
\newenvironment{tdisp}{\begin{myquote}}{\end{myquote}}
\newenvironment{example}{\begin{myquote}\noindent\itshape 例子:}{\end{myquote}}
\newenvironment{inventory}{\begin{description}\raggedright}{\end{description}}
\newenvironment{glossinventory}{\begin{description}}{\end{description}}
\def\gram#1{\gr{#1}}%???
\def\meta{\gr}% alias
%
% index
%
\def\indexterm#1{\emph{#1}\index{#1}}
\def\indextermsub#1#2{\emph{#1 #2}\index{#1!#2}}
\def\indextermbus#1#2{\emph{#1 #2}\index{#2!#1}}
\def\cindextermsub#1#2{\emph{#1#2}\index{#1!#1#2}}
\def\cindextermbus#1#2{\emph{#1#2}\index{#2!#1#2}}
\def\term#1\par{\index{#1}}
\def\howto#1\par{}
\def\cstoidx#1\par{\index{#1@\cs{#1}@}}
\def\thecstoidx#1\par{\index{#1@\protect\csname #1\endcsname}}
\def\thecstoidxsub#1#2{\index{#1, #2@\protect\csname #1\endcsname, #2}\ignorespaces}
\def\csterm#1\par{\cstoidx #1\par\cs{#1}}
\def\csidx#1{\cstoidx #1\par\cs{#1}}

\def\tmc{\tracingmacros=2 \tracingcommands\tracingmacros}

%%%%%%%%%%%%%%%%%%%
\makeatletter
\def\snugbox{\hbox\bgroup\setbox\z@\vbox\bgroup
    \leftskip\z@
    \bgroup\aftergroup\make@snug
    \let\next=}
\def\make@snug{\par\sn@gify\egroup \box\z@\egroup}
\def\sn@gify
   {\skip\z@=\lastskip \unskip
    \advance\skip\z@\lastskip \unskip
    \unpenalty
    \setbox\z@\lastbox
    \ifvoid\z@ \nointerlineskip \else {\sn@gify} \fi
    \hbox{\unhbox\z@}\nointerlineskip
    \vskip\skip\z@
    }

\newdimen\fbh \fbh=60pt % dimension for easy scaling:
\newdimen\fbw \fbw=60pt % height and width of character box

\newdimen\dh \newdimen\dw % height and width of current character box
\newdimen\lh % height of previous character box
\newdimen\lw \lw=.4pt % line weight, instead of default .4pt

\def\hdotfill{\noindent
    \leaders\hbox{\vrule width 1pt height\lw
                  \kern4pt
                  \vrule width.5pt height\lw}\hfill\hbox{}
    \par}
\def\hlinefill{\noindent
    \leaders\hbox{\vrule width 5.5pt height\lw         }\hfill\hbox{}
    \par}
\def\stippel{$\qquad\qquad\qquad\qquad$}
\makeatother
%%%%%%%%%%%%%%%%%%%

%\def\SansSerif{\Typeface:macHelvetica }
%\def\SerifFont{\Typeface:macTimes }
%\def\SansSerif{\Typeface:bsGillSans }
%\def\SerifFont{\Typeface:bsBaskerville }
\let\SansSerif\relax \def\italic{\it}
\let\SerifFont\relax \def\MainFont{\rm}
\let\SansSerif\relax
\let\SerifFont\relax
\let\PopIndentLevel\relax \let\PushIndentLevel\relax
\let\ToVerso\relax \let\ToRecto\relax

%\def\stop@command@suffix{stop}
%\let\PopListLevel\PopIndentLevel
%\let\FlushRight\relax
%\let\flushright\FlushRight
%\let\SetListIndent\LevelIndent
%\def\awp{\ifhmode\vadjust{\penalty-10000 }\else
%    \penalty-10000 \fi}
\let\awp\relax
\let\PopIndentLevel\relax \let\PopListLevel\relax

\showboxdepth=-1

%\input figs
\def\endofchapter{\vfill\noindent}

\newcommand{\liamfnote}[1]{\footnote{译注(Liam0205):#1}}
\newcommand{\cstate}[1]{状态 \textit{#1}}

\setcounter{chapter}{31}

\begin{document}

%\chapter{Running \TeX}\label{run}
\chapter{运行 \TeX}\label{run}

%This chapter treats the run modes of \TeX, and some
%other commands associated with the job being processed.
本章主要讨论 \TeX 的运行模式以及一些和任务(job)处理有关的命令。

%\label{cschap:everyjob}\label{cschap:jobname}\label{cschap:end}\label{cschap:bye}\label{cschap:pausing}\label{cschap:errorstopmode}\label{cschap:scrollmode}\label{cschap:nonstopmode}\label{cschap:batchmode}
%\begin{inventory}
%\item [\cs{everyjob}] 
%      Token list that is inserted at the start of each new job.
\label{cschap:everyjob}\label{cschap:jobname}\label{cschap:end}\label{cschap:bye}\label{cschap:pausing}\label{cschap:errorstopmode}\label{cschap:scrollmode}\label{cschap:nonstopmode}\label{cschap:batchmode}
\begin{inventory}
\item [\cs{everyjob}] 
      每个新任务开始时都会插入的记号(Token)列表。

%\item [\cs{jobname}] 
%      Name of the main \TeX\ file being processed.
\item [\cs{jobname}] 
      正在处理的 \TeX\ 文件的名称。

%\item [\cs{end}] 
%      Command to finish off a run of \TeX.
\item [\cs{end}] 
      中止 \TeX 运行的命令。

%\item [\cs{bye}]
%      Plain \TeX\ macro to force the final output.
\item [\cs{bye}]
      强制最终输出的 Plain \TeX\ 宏。

%\item [\cs{pausing}] 
%      Specify that \TeX\ should pause after each line that is 
%      read from a file.
\item [\cs{pausing}] 
      让 \TeX\ 每读完文件的一行,就暂停一下。

%\item [\cs{errorstopmode}]
%      \TeX\ will ask for user input on the occurrence of an error.
\item [\cs{errorstopmode}]
      让 \TeX\ 在每次出错时都暂停下来,要求用户输入。 

%\item [\cs{scrollmode}] 
%      \TeX\ fixes errors itself,
%      but will ask the user for missing files.
\item [\cs{scrollmode}] 
      让 \TeX\ 在出错时自行修复错误,但要求用户提供缺失的文件。

%\item [\cs{nonstopmode}] 
%      \TeX\ fixes errors itself, 
%      and performs an emergency stop on serious errors 
%      such as missing input files.
\item [\cs{nonstopmode}] 
      让 \TeX\ 在出错时自行修复错误,并且只在发生类似“输入文件缺失”等致命错误时才停止。

%\item [\cs{batchmode}] 
%      \TeX\ fixes errors itself 
%      and performs an emergency stop on serious errors 
%      such as missing input files,
%      but no terminal output is generated.
\item [\cs{batchmode}] 
      功能与上一个命令 \cs{nonstopmode} 相同,但不产生终端输出。

%\end{inventory}
\end{inventory}

%%\point Jobs
%\section{Jobs}
%\index{job|(}
%\point Jobs
\section{任务}
\index{job|(}

%\TeX\ associates with each run a name for the file
%being processed: the \csidx{jobname}. If \TeX\ is run
%interactively
%\ldash meaning that it has been invoked without a file argument,
%and the user types commands \rdash
%the jobname is \n{texput}.
\TeX\ 为每个待处理的文件指定了一个名字:\csidx{jobname}。
如果交互式的运行 \TeX\ 
\ldash 这意味着并不存在一个真实的文件,
用户直接输入命令运行 \rdash 
这时将用 \n{texput} 作为任务名。

%The \cs{jobname} can be used to generate
%the names of auxiliary files to be read or
%written during the run. For instance, for a file \n{story.tex}
%the \cs{jobname} is \n{story}, and writing
%\begin{verbatim}
%\openout\Auxiliary=\jobname.aux
%\openout\TableOfContents=\jobname.toc
%\end{verbatim}
%will create the files \n{story.aux} and \n{story.toc}.
\cs{jobname} 可以用于生成临时文件名。
例如,对于一个叫 \n{story.tex} 的文件,
\cs{jobname} 就是 \n{story},以下代码
\begin{verbatim}
\openout\Auxiliary=\jobname.aux
\openout\TableOfContents=\jobname.toc
\end{verbatim}
会创建 \n{story.aux} 和 \n{story.toc} 两个文件。

%%\spoint Start of the job
%\subsection{Start of the job}
%\spoint Start of the job
\subsection{任务的开始}

%\TeX\ starts each job by inserting the \csidx{everyjob} token
%list into the command stream.
%Setting this variable during a run of \TeX\ has no use,
%but a format can use it to identify itself to the user.
%If a
%format fills the token list, the commands therein are automatically
%executed when \TeX\ is run using that format.
\TeX\ 通过将 \csidx{everyjob} 记号列表插入命令流来启动每个任务。
在 \TeX\ 运行期间设置这个变量是没有用的,
但文件格式可以使用它来向用户标识自己。
如果一种格式填充了记号列表,当 \TeX\ 使用这种格式运行时,
列表里面的命令会自动执行。

%%\spoint End of the job
%\subsection{End of the job}
%\spoint End of the job
\subsection{任务的结束}

%A \TeX\ job is terminated by the \csidx{end} command. This
%may involve first forcing the output routine to process any
%remaining material (see Chapter~\ref{page:break}).
%If the end of job occurs inside a group
%\TeX\ will give a diagnostic
%message. The \cs{end} command is not allowed in internal
%vertical mode, because this would be inside a vertical box.
一个 \TeX\ 任务通过 \csidx{end} 命令中止。
这可能包括首先强制输出程序去处理任何剩余的材料(参第~\ref{page:break} 章)。
如果任务的终止发生在一个组(group)中,\TeX\ 会输出一个诊断信息。
\cs{end} 命令会插入一个竖直盒子,因此内部竖直模式中不允许使用该命令。

%Usually some sugar coating of the \cs{end} command is necessary.
%For instance the plain \TeX\ macro \csidx{bye} is defined
%as
%\begin{verbatim}
%\def\bye{\par\vfill\supereject\end}
%\end{verbatim}
%where the \cs{supereject} takes care of any leftover insertions.
通常来说,一些 \cs{end} 命令的语法糖是十分必要的。
例如 plain \TeX\ 宏 \csidx{bye} 是这样定义的:
\begin{verbatim}
\def\bye{\par\vfill\supereject\end}
\end{verbatim}
其中 \cs{supereject} 负责处理任何剩余的插入项。

%%\spoint The log file
%\subsection{The log file}
%\spoint The log file
\subsection{日志文件}

%For each run \TeX\ creates a \indexterm{log file}. Usually this will be
%a file with as name the value of \cs{jobname}, and the
%extension \n{.log}. Other extensions such as \n{.lis}
%are used by some implementations.
%This log file contains all information that
%is displayed on the screen during the run of \TeX, but
%it will display some information more elaborately, and it
%can contain statistics that are usually not displayed on
%the screen. If the parameter \cs{tracingonline}
%has a positive value, all the log file information will be
%shown on the screen.
每次运行 \TeX\ 都会创建一个日志文件(\indexterm{log file})。
通常这个文件名字为 \cs{jobname} 的值,后缀为 \n{.log}。
有些发行版会使用其他的扩展名如 \n{.lis}。
日志文件包含 \TeX\ 运行时显示在屏幕上的所有信息,
但它会更加详细的显示一些信息,并且日志通常会包含不显示在屏幕上的统计信息。
如果 \cs{tracingonline} 参数为正值,所有的日志信息都会输出到屏幕上。

%Overfull and underfull boxes are reported on the terminal
%screen, and they are dumped using the parameters
%\cs{showboxdepth} and \cs{showboxbreadth} in the log file
%(see Chapter~\ref{trace}). These parameters are also used
%for box dumps caused by the \cs{showbox} command, and
%for the dump of boxes written by \cs{shipout} 
%if \cs{tracingoutput} is set to a positive value.
盒子上溢出和下溢出会在终端中报告,使用 \cs{showboxdepth} 和
\cs{showboxbreadth} 可以转储它们到日志文件中(参见第~\ref{trace}章)。
这些参数也会影响由 \cs{showbox} 命令生成的盒子转储,
如果 \cs{tracingoutput} 为正数,
参数也会影响 \cs{shipout} 命令写入的盒子转储。

%Statistics generated by commands such as \cs{tracingparagraphs}
%will be written to the log file; if \cs{tracingonline} is positive
%they will also be shown on the screen.
由像 \cs{tracingparagraphs} 这样的命令生成的统计信息会写入日志中;
如果 \cs{tracingonline} 为正数,统计信息还会显示在屏幕上。

%Output operations to a stream that is not open, or to a
%stream with a number that is not in the range 0--15,
%go to the log file. If the stream number is positive,
%they also go to the terminal.
输出到未打开的流中、或指定流的编号不在 0--15 这个范围内,会输出到日志里。
如果流的编号为正数,会同时输出到终端。

%\index{job|)}
\index{job|)}

%\section{Run modes}
%\index{run modes|(}
\section{运行模式}
\index{run modes|(}

%By default, \TeX\ goes into \cs{errorstopmode} if an error occurs:
%\cstoidx errorstopmode\par
%it stops and asks for input from the user. Some implementations 
% have a way of forcing \TeX\ into errorstopmode
%when the user interrupts \TeX, so that
%the internal state of \TeX\ can be inspected (and altered). 
%See page~\pageref{interaction} for ways to switch the run
%mode when \TeX\ has been interrupted.
默认情况下 \TeX\ 会进入 \cs{errorstopmode} 模式,
\cstoidx errorstopmode\par
发生错误时,\TeX\ 会停下来等待用户的输入。
一些发行版会在用户中断 \TeX\ 运行时,强制进入 errorstopmode 模式,
这样就可以检查或修改 \TeX\ 的内部状态。
更多关于 \TeX\ 中断运行时的模式切换,请参见~\pageref{interaction}页。

%Often, \TeX\ can
%fix an error itself if the user asks \TeX\ just to continue
%(usually by hitting the return key),
%but sometimes (for instance in alignments)
%it may take a while before \TeX\ is on the
%right track again (and sometimes it never is).
%In such cases the user may want to
%turn on \csidx{scrollmode},
%which instructs \TeX\ to fix as best it can any
%occurring error without confirmation from the user.
%This is usually done by typing `s' when \TeX\ asks
%for input.
通常情况下,当用户要求 \TeX\ 继续运行时(通常是按回车健),
\TeX\ 能自己修复错误,
但是有时(如在处理对齐(alignments)时)\TeX\ 需要很长时间才能回到正确的道路上。
这时,用户可能希望进入 \csidx{scrollmode} 模式。在这种模式下,
发生错误时,用户无需任何确认操作,\TeX\ 就能尽力去修复错误。
在 \TeX\ 等待用户输入时,输入 “s”,就能进入这种模式。

%In \cs{scrollmode}, \TeX\ also does not ask for input
%after \cs{show...} commands.
%\alt
%However, some errors, such as a file that could not be
%found for \cs{input}, are not so easily remedied, so
%the user will still be asked for input.
在 \cs{scrollmode} 模式中,碰到 \cs{show...} 命令,
\TeX\ 也不会要求用户输入。
\alt
但是在发生一些难以自动修复的错误时,如找不到 \cs{input} 所需的文件,
\TeX\ 依旧会等待用户的输入。

%With \csidx{nonstopmode} \TeX\ will scroll through errors and,
%in the case of the kind of error that cannot be recovered from,
%it will make an emergency stop, aborting the run.
%Also \TeX\ will abort the run if a \cs{read} is attempted
%from the terminal.
%The \csidx{batchmode} differs only from nonstopmode in that
%it gives messages only to the log file, not to the terminal.
在 \csidx{nonstopmode} 模式中,\TeX\ 会忽略错误。
当发生无法恢复的错误时,\TeX\ 会紧急停止、中断运行。
碰到 \cs{read} 命令试图从终端读取输入时,\TeX\ 也会中断运行。
\csidx{batchmode} 与 nonstopmode 唯一的区别是:前者只将输出信息写入日志,而不会输出到终端。

%\index{run modes|)}
%\endofchapter
\index{run modes|)}
\endofchapter

\end{document}
