% -*- coding: utf-8 -*-
\documentclass{book}

% -*- coding: utf-8 -*-

\usepackage[b5paper,text={5in,8in},centering]{geometry}
\usepackage{amsmath}
\usepackage{amssymb}
\usepackage[heading = false, scheme = plain]{ctex}
% \usepackage[CJKchecksingle]{xeCJK}
\setmainfont[Mapping=tex-text]{TeX Gyre Schola}
%\setsansfont{URW Gothic L Book}
%\setmonofont{Nimbus Mono L}
% \setCJKmainfont[BoldFont=FandolHei,ItalicFont=FandolKai]{FandolSong}
% \setCJKsansfont{FandolHei}
% \setCJKmonofont{FandolFang}
\xeCJKsetup{PunctStyle = kaiming}

\linespread{1.25}
\setlength{\parindent}{2em}
\setlength{\parskip}{0.5ex}
\usepackage{indentfirst}

\usepackage{xcolor}
\definecolor{myblue}{rgb}{0,0.2,0.6}

\usepackage{titlesec}
\titleformat{\chapter}
    {\normalfont\Huge\sffamily\color{myblue}}
    {第\thechapter 章}
    {1em}
    {}
%\titlespacing{\chapter}{0pt}{50pt}{40pt}
\titleformat{\section}
    {\normalfont\Large\sffamily\color{myblue}}
    {\thesection}
    {1em}
    {}
%\titlespacing{\section}{0pt}{3.5ex plus 1ex minus .2ex}{2.3ex plus .2ex}
\titleformat{\subsection}
    {\normalfont\large\sffamily\color{myblue}}
    {\thesubsection}
    {1em}
    {}
%\titlespacing{\subsection}{0pt}{3.25ex plus 1ex minus .2ex}{1.5ex plus .2ex}
%
\newpagestyle{special}[\small\sffamily]{
  \headrule
  \sethead[\usepage][][\chaptertitle]
  {\chaptertitle}{}{\usepage}}
\newpagestyle{main}[\small\sffamily]{
  \headrule
  \sethead[\usepage][][第\thechapter 章\quad\chaptertitle]
  {\thesection\quad\sectiontitle}{}{\usepage}}

\usepackage{titletoc}
%\setcounter{tocdepth}{1}
%\titlecontents{标题层次}[左间距]{上间距和整体格式}{标题序号}{标题内容}{指引线和页码}[下间距]
\titlecontents{chapter}[1.5em]{\vspace{.5em}\bfseries\sffamily}{\color{myblue}\contentslabel{1.5em}}{}
    {\titlerule*[20pt]{$\cdot$}\contentspage}[]
\titlecontents{section}[4.5em]{\sffamily}{\color{myblue}\contentslabel{3em}}{}
    {\titlerule*[20pt]{$\cdot$}\contentspage}[]
%\titlecontents{subsection}[8.5em]{\sffamily}{\contentslabel{4em}}{}
%    {\titlerule*[20pt]{$\cdot$}\contentspage}

\usepackage{enumitem}
\setlist{topsep=2pt,itemsep=2pt,parsep=1pt,leftmargin=\parindent}

\usepackage{fancyvrb}
\DefineVerbatimEnvironment{verbatim}{Verbatim}
  {xleftmargin=2em,baselinestretch=1,formatcom=\color{teal}\upshape}

\usepackage{etoolbox}
\makeatletter
\preto{\FV@ListVSpace}{\topsep=2pt \partopsep=0pt }
\makeatother

\usepackage[colorlinks,plainpages,pagebackref]{hyperref}
\hypersetup{
   pdfstartview={FitH},
   citecolor=teal,
   linkcolor=myblue,
   urlcolor=black,
   bookmarksnumbered
}

\usepackage{comment,makeidx,multicol}

%\usepackage{german}
%% german
%\righthyphenmin=3
%\mdqoff
%\captionsenglish
\usepackage[english]{babel}
{\catcode`"=13 \gdef"#1{\ifx#1"\discretionary{}{}{}\fi\relax}}
\def\mdqon{\catcode`"=13\relax}
\def\mdqoff{\catcode`"=12\relax}
\makeindex
\hyphenation{ex-em-pli-fies}

\newdimen\tempdima \newdimen\tempdimb

% these are fine
\def\handbreak{\\ \message{^^JManual break!!!!^^J}}
\def\nl{\protect\\}\def\n#1{{\tt #1}}
\protected\def\cs#1{\texttt{\textbackslash#1}}
\pdfstringdefDisableCommands{\def\cs#1{\textbackslash#1}}
\let\csc\cs
\def\lb{{\tt\char`\{}}\def\rb{{\tt\char`\}}}
\def\gr#1{\texorpdfstring{$\langle$#1$\rangle$}{<#1>}} %\def\gr#1{$\langle$#1$\rangle$}
\def\marg#1{{\tt \{}#1{\tt \}}}
\def\oarg#1{{\tt [}#1{\tt }}
\def\key#1{{\tt#1}}
\def\alt{}\def\altt{}%this way in manstijl
\def\ldash{\unskip\ ——\nobreak\ \ignorespaces}
\def\rdash{\unskip\nobreak\ ——\ \ignorespaces}
% check these
\def\hex{{\tt"}}
\def\ascii{{\sc ascii}}
\def\ebcdic{{\sc ebcdic}}
\def\IniTeX{Ini\TeX}\def\LamsTeX{LAMS\TeX}\def\VirTeX{Vir\TeX}
\def\AmsTeX{Ams\TeX}
\def\TeXbook{the \TeX\ book}\def\web{{\sc web}}
% needs major thinking
\newenvironment{myquote}{\list{}{%
    \topsep=2pt \partopsep=0pt%
    \leftmargin=\parindent \rightmargin=\parindent
    }\item[]}{\endlist}
\newenvironment{disp}{\begin{myquote}}{\end{myquote}}
\newenvironment{Disp}{\begin{myquote}}{\end{myquote}}
\newenvironment{tdisp}{\begin{myquote}}{\end{myquote}}
\newenvironment{example}{\begin{myquote}\noindent\itshape 例子:}{\end{myquote}}
\newenvironment{inventory}{\begin{description}\raggedright}{\end{description}}
\newenvironment{glossinventory}{\begin{description}}{\end{description}}
\def\gram#1{\gr{#1}}%???
\def\meta{\gr}% alias
%
% index
%
\def\indexterm#1{\emph{#1}\index{#1}}
\def\indextermsub#1#2{\emph{#1 #2}\index{#1!#2}}
\def\indextermbus#1#2{\emph{#1 #2}\index{#2!#1}}
\def\cindextermsub#1#2{\emph{#1#2}\index{#1!#1#2}}
\def\cindextermbus#1#2{\emph{#1#2}\index{#2!#1#2}}
\def\term#1\par{\index{#1}}
\def\howto#1\par{}
\def\cstoidx#1\par{\index{#1@\cs{#1}@}}
\def\thecstoidx#1\par{\index{#1@\protect\csname #1\endcsname}}
\def\thecstoidxsub#1#2{\index{#1, #2@\protect\csname #1\endcsname, #2}\ignorespaces}
\def\csterm#1\par{\cstoidx #1\par\cs{#1}}
\def\csidx#1{\cstoidx #1\par\cs{#1}}

\def\tmc{\tracingmacros=2 \tracingcommands\tracingmacros}

%%%%%%%%%%%%%%%%%%%
\makeatletter
\def\snugbox{\hbox\bgroup\setbox\z@\vbox\bgroup
    \leftskip\z@
    \bgroup\aftergroup\make@snug
    \let\next=}
\def\make@snug{\par\sn@gify\egroup \box\z@\egroup}
\def\sn@gify
   {\skip\z@=\lastskip \unskip
    \advance\skip\z@\lastskip \unskip
    \unpenalty
    \setbox\z@\lastbox
    \ifvoid\z@ \nointerlineskip \else {\sn@gify} \fi
    \hbox{\unhbox\z@}\nointerlineskip
    \vskip\skip\z@
    }

\newdimen\fbh \fbh=60pt % dimension for easy scaling:
\newdimen\fbw \fbw=60pt % height and width of character box

\newdimen\dh \newdimen\dw % height and width of current character box
\newdimen\lh % height of previous character box
\newdimen\lw \lw=.4pt % line weight, instead of default .4pt

\def\hdotfill{\noindent
    \leaders\hbox{\vrule width 1pt height\lw
                  \kern4pt
                  \vrule width.5pt height\lw}\hfill\hbox{}
    \par}
\def\hlinefill{\noindent
    \leaders\hbox{\vrule width 5.5pt height\lw         }\hfill\hbox{}
    \par}
\def\stippel{$\qquad\qquad\qquad\qquad$}
\makeatother
%%%%%%%%%%%%%%%%%%%

%\def\SansSerif{\Typeface:macHelvetica }
%\def\SerifFont{\Typeface:macTimes }
%\def\SansSerif{\Typeface:bsGillSans }
%\def\SerifFont{\Typeface:bsBaskerville }
\let\SansSerif\relax \def\italic{\it}
\let\SerifFont\relax \def\MainFont{\rm}
\let\SansSerif\relax
\let\SerifFont\relax
\let\PopIndentLevel\relax \let\PushIndentLevel\relax
\let\ToVerso\relax \let\ToRecto\relax

%\def\stop@command@suffix{stop}
%\let\PopListLevel\PopIndentLevel
%\let\FlushRight\relax
%\let\flushright\FlushRight
%\let\SetListIndent\LevelIndent
%\def\awp{\ifhmode\vadjust{\penalty-10000 }\else
%    \penalty-10000 \fi}
\let\awp\relax
\let\PopIndentLevel\relax \let\PopListLevel\relax

\showboxdepth=-1

%\input figs
\def\endofchapter{\vfill\noindent}

\newcommand{\liamfnote}[1]{\footnote{译注(Liam0205):#1}}
\newcommand{\cstate}[1]{状态 \textit{#1}}

\setcounter{chapter}{30}

\begin{document}

%\chapter{Allocation}\label{alloc}
\chapter{寄存器分配}\label{alloc}

%\TeX\ has registers of a number of types. For some of these,
%explicit commands exist to define a synonym for a certain register;
%for all of them macros exist in the plain format
%to allocate an unused register. This chapter treats
%the synonym and allocation commands, and discusses
%some guidelines for macro writers regarding allocation.
\TeX\ 的寄存器是数字类型的。对于其中的一些寄存器,
我们可以使用显式的命令定义某个特定寄存器的别名;
在 plain \TeX\ 中,
所有类型的寄存器都可以通过对应的宏来分配一个未使用的寄存器。
本章主要关注别名定义和寄存器分配命令。
对于宏编写者,我们讨论了一些关于寄存器分配的准则。

%\begin{inventory}
%\item [\cs{countdef}] 
%      Define a synonym for a \cs{count} register.
%\item [\cs{dimendef}]
%      Define a synonym for a \cs{dimen} register.
%\item [\cs{muskipdef}]
%      Define a synonym for a \cs{muskip} register.
%\item [\cs{skipdef}] 
%      Define a synonym for a \cs{skip} register.
%\item [\cs{toksdef}] 
%      Define a synonym for a \cs{toks} register.
%\item [\cs{newbox}]
%      Allocate an unused \cs{box} register.
%\item [\cs{newcount}]
%      Allocate an unused \cs{count} register.
%\item [\cs{newdimen}]
%      Allocate an unused \cs{dimen} register.
%\item [\cs{newfam}]
%      Allocate an unused math family.
%\item [\cs{newinsert}]
%      Allocate an unused insertion class.
%\item [\cs{newlanguage}]
%      (\TeX3 only)
%      Allocate a new language number.
%\item [\cs{newmuskip}]
%      Allocate an unused \cs{muskip} register.
%\item [\cs{newskip}]
%      Allocate an unused \cs{skip} register.
%\item [\cs{newtoks}]
%      Allocate an unused \cs{toks} register.
%\item [\cs{newread}]
%      Allocate an unused input stream.
%\item [\cs{newwrite}]
%      Allocate an unused output stream.
%\end{inventory}
\begin{inventory}
\item [\cs{countdef}] 
      定义了一个 \cs{count} 寄存器的别名。
\item [\cs{dimendef}]
      定义了一个 \cs{dimen} 寄存器的别名。
\item [\cs{muskipdef}]
      定义了一个 \cs{muskip} 寄存器的别名。
\item [\cs{skipdef}] 
      定义了一个 \cs{skip} 寄存器的别名。
\item [\cs{toksdef}] 
      定义了一个 \cs{toks} 寄存器的别名。
\item [\cs{newbox}]
      分配了一个未使用的 \cs{box} 寄存器。
\item [\cs{newcount}]
      分配了一个未使用的 \cs{count} 寄存器。
\item [\cs{newdimen}]
      分配了一个未使用的 \cs{dimen} 寄存器。
\item [\cs{newfam}]
      分配了一个未使用的数学字体族。
\item [\cs{newinsert}]
      分配了一个未使用的插入类。
\item [\cs{newlanguage}]
      (仅 \TeX3)
      分配了一个新的语言编号。
\item [\cs{newmuskip}]
      分配了一个未使用的 \cs{muskip} 寄存器。
\item [\cs{newskip}]
      分配了一个未使用的 \cs{skip} 寄存器。
\item [\cs{newtoks}]
      分配了一个未使用的 \cs{toks} 寄存器。
\item [\cs{newread}]
      分配了一个未使用的输入流。
\item [\cs{newwrite}]
      分配了一个未使用的输出流。
\end{inventory}

%%\point Allocation commands
%\section{Allocation commands}
%\index{registers!allocation of|(}
%\point Allocation commands
\section{寄存器分配命令}
\index{registers!allocation of|(}

%In plain \TeX, \cs{new...} macros are defined for
%allocation of registers.
%The registers of \TeX\ fall into two classes that are 
%allocated in different ways. This is treated below.
在 plain \TeX 中, 定义了宏 \cs{new...} 用于分配寄存器。
\TeX\ 的寄存器可以根据分配方式的不同而分为两类。这一点将会在下面谈到。

%The \csidx{newlanguage} macro of plain \TeX\ 
%does not allocate any register. Instead it merely assigns
%a number, starting from~0.
%\TeX\ (version~3) can have at most 256 different
%sets of hyphenation patterns.
plain \TeX\ 的 \csidx{newlanguage} 宏并不会分配任何的寄存器。
相反的,它只是从0开始,分配了一个数字。
\TeX3 最多有256种断字组合模式。

%The \cs{new...} macros of plain \TeX\ are defined to be
%\cs{outer} (see Chapter~\ref{macro} for a precise explanation),
%which precludes use of the allocation macros in other macros.
%Therefore the \LaTeX\ format redefines these macros
%without the \cs{outer} prefix.
plain \TeX\ 的 \cs{new...} 宏被定义为 \cs{outer}
(精确定义详见 \ref{macro} 章),
\cs{outer} 避免了在其他宏中使用分配宏。
因此,\LaTeX\ 在重定义这些宏时没有带上 \cs{outer} 前缀。

%%\spoint \cs{count}, \cs{dimen}, \cs{skip}, \cs{muskip}, \cs{toks}
%\subsection{\cs{count}, \cs{dimen}, \cs{skip}, \cs{muskip}, \cs{toks}}
%\spoint \cs{count}, \cs{dimen}, \cs{skip}, \cs{muskip}, \cs{toks}
\subsection{\cs{count}, \cs{dimen}, \cs{skip}, \cs{muskip}, \cs{toks}}

%For these registers there exists a \gr{registerdef} command,
%for instance \cs{countdef}, to couple a specific register
%to a control sequence:
%\begin{Disp}\gr{registerdef}\gr{control 
%    sequence}\gr{equals}\gr{8-bit number}\end{Disp}
对于这些寄存器,存在一个 \gr{registerdef} 命令。
举个例子:\cs{countdef}, 将寄存器与特定的控制序列组合在一起:
\begin{Disp}\gr{registerdef}\gr{control 
    sequence}\gr{equals}\gr{8-bit number}\end{Disp}

%After the definition
%\begin{verbatim}
%\countdef\MyCount=42
%\end{verbatim}
%the allocated register can be used as
%\begin{verbatim}
%\MyCount=314
%\end{verbatim}
%or
%\begin{verbatim}
%\vskip\MyCount\baselineskip
%\end{verbatim}
在定义之后
\begin{verbatim}
\countdef\MyCount=42
\end{verbatim}
分配的寄存器可以这样使用:
\begin{verbatim}
\MyCount=314
\end{verbatim}
或者
\begin{verbatim}
\vskip\MyCount\baselineskip
\end{verbatim}

%The \gr{registerdef} commands are used in plain \TeX\ macros
%\cs{newcount} et cetera that allocate an unused register;
%after
%\begin{verbatim}
%\newcount\MyCount
%\end{verbatim}
%\cs{MyCount} can be used
%exactly as in the above two examples.
\gr{registerdef} 命令用在 \cs{newcount} 等 plain \TeX\ 宏中,
它能分配一个未使用的寄存器;在以下命令之后
\begin{verbatim}
\newcount\MyCount
\end{verbatim}
\cs{MyCount} 就可以完全按照上述两个例子那样使用。

%%\spoint \cs{box}, \cs{fam}, \cs{write}, \cs{read}, \cs{insert}
%\subsection{\cs{box}, \cs{fam}, \cs{write}, \cs{read}, \cs{insert}}
%\spoint \cs{box}, \cs{fam}, \cs{write}, \cs{read}, \cs{insert}
\subsection{\cs{box}, \cs{fam}, \cs{write}, \cs{read}, \cs{insert}}

%For these registers there exists no  \gr{registerdef} command in \TeX,
%so \cs{chardef} is used to allocate box registers
%in the corresponding plain \TeX\ macros \cs{newbox}, for instance.
对于 \TeX 中的这些寄存器,不存在 \gr{registerdef} 命令,所以 \cs{chardef} 命令用于在对应的 \TeX 宏 \cs{newbox} 中分配盒子寄存器,举个例子。

%The fact that \cs{chardef} is used implies that the
%defined control sequence does not stand for the register itself,
%but only for its number. Thus after
%\begin{verbatim}
%\newbox\MyBox
%\end{verbatim}
%it is necessary to write
%\begin{verbatim}
%\box\MyBox
%\end{verbatim} 
%Leaving out the \cs{box} means that the character
%in the current font with number
%\cs{MyBox} is typeset. The \cs{chardef} command
%is treated further in Chapter~\ref{char}.
使用 \cs{chardef} 意味着定义的控制序列不代表寄存器本身,而仅代表它的编号。因此在定义了
\begin{verbatim}
\newbox\MyBox
\end{verbatim}
后,有必要继续写
\begin{verbatim}
\box\MyBox
\end{verbatim}
省略 \cs{box} 意味着当前字体编号为 \cs{MyBox} 的字符正在被排版。
\cs{chardef} 在第\ref{char}章中会详细叙述。

%\index{registers!allocation of|)}
\index{registers!allocation of|)}

%\section{Ground rules for macro writers}
\section{给宏编写者的基本准则}

%The \cs{new...} macros of plain \TeX\ have been designed
%to form a foundation for macro packages, such that
%several of such packages can operate without collisions
%in the same run of \TeX. In appendix~B of \TeXbook\
%Knuth formulates some ground rules that macro writers should
%adhere to.
%\begin{enumerate}
%\item The \cs{new...} macros do not allocate registers
%with numbers~0--9. These can therefore be used as `scratch'
%registers. However, as any macro family can use them,
%no assumption can be made about the permanency of their
%contents. Results that are to be passed from one call to
%another should reside in specifically allocated registers.
plain \TeX\ 的 \cs{new...} 宏被设计为用来构建宏包的基础,
这样在同一个 \TeX\ 实例中,不同的宏包之间也不会冲突。
在Knuth的\TeXbook\ 附录~B 中规定了一些宏编写者应当遵守的基本准则。
\begin{enumerate}
\item 不要使用 \cs{new...} 宏分配编号~0--9的寄存器。
这些寄存器会被用作临时寄存器。
任意一个宏族都可以使用它们,因此无法保证它们的值是永久不变的。
需要从一个调用传递到另一个调用的结果应放在特定的寄存器中。

%Note that count registers 0--9 are used for page identification
%in the \n{dvi} file (see Chapter~\ref{TeXcomm}), so no global assignments
%to these should be made.
注意到计数用(count) 0--9号寄存器用于 \n{dvi} 文件的页面标记
(见~\ref{TeXcomm}章),因此对这些寄存器不应有全局的赋值。

%\item \cs{count255}, \cs{dimen255}, and \cs{skip255} are
%also available. This is because inserts are
%allocated from 254 downward  and, together with an insertion box,
%a count, dimen, and skip register, 
%all with the same number, are allocated.
%Since \cs{box255} is used by the output routine 
%(see Chapter~\ref{output}),
%the count, dimen, and skip with number~255 are freely available.
\item \cs{count255}, \cs{dimen255}, 和 \cs{skip255} 是可用的寄存器。
这是因为插入盒子用的寄存器从254开始向下分配,
每一个插入盒子都会分配一组同编号的 count, dimen, 和 skip 寄存器。
另外box255由输出程序使用(见~\ref{output}章),
故剩下的编号为255的 count, dimen, 和 skip寄存器可以自由使用。

%\item Assignments to scratch registers~0, 2, 4, 6, 8, and~255
%should be local; assignments to registers~1, 3, 5, 7,~9
%should be \cs{global} (with the exception of the \cs{count}
%registers). This guideline prevents `save
%stack build-up' (see Chapter~\ref{error}).
\item 给 0, 2, 4, 6, 8, 和 255 号临时寄存器的赋值应当是局部的。
给 1, 3, 5, 7, 9号寄存器的赋值应当是全局的。(除\cs{count}寄存器外)。
这条准则避免了‘保存栈的建立’(见~\ref{error}章)。

%\item Any register can be used inside a group, as \TeX's
%grouping mechanism will restore its value outside
%the group. There are two conditions on this use of
%a register:
%no global assignments should be made to it, and 
%it must not be possible that other macros may be
%activated in that group that perform global assignments
%to that register.
\item 任何寄存器都应在组内使用,\TeX\ 的分组机制会在组外保存它的值。
使用这种寄存器有两个条件:
没有对它进行全局赋值,并且组内没有执行全局分配该寄存器的其他宏。

%\item Registers that are used over longer periods of time,
%or that have to survive in between calls of different
%macros, should be allocated by \cs{new...}.
%\end{enumerate}
\item 长期使用的寄存器,或在不同宏的调用中需要保留的寄存器应使用 \cs{new...} 来分配。
\end{enumerate}

%\endofchapter
%%%%% end of input file [alloc]
\endofchapter
%%%% end of input file [alloc]

\end{document}
