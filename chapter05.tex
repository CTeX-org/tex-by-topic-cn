% -*- coding: utf-8 -*-
% Translated by Liam0205@bbs.ctex.org
% Date of translated: 2018-05-12
\documentclass{book}

% -*- coding: utf-8 -*-

\usepackage[b5paper,text={5in,8in},centering]{geometry}
\usepackage{amsmath}
\usepackage{amssymb}
\usepackage[heading = false, scheme = plain]{ctex}
% \usepackage[CJKchecksingle]{xeCJK}
\setmainfont[Mapping=tex-text]{TeX Gyre Schola}
%\setsansfont{URW Gothic L Book}
%\setmonofont{Nimbus Mono L}
% \setCJKmainfont[BoldFont=FandolHei,ItalicFont=FandolKai]{FandolSong}
% \setCJKsansfont{FandolHei}
% \setCJKmonofont{FandolFang}
\xeCJKsetup{PunctStyle = kaiming}

\linespread{1.25}
\setlength{\parindent}{2em}
\setlength{\parskip}{0.5ex}
\usepackage{indentfirst}

\usepackage{xcolor}
\definecolor{myblue}{rgb}{0,0.2,0.6}

\usepackage{titlesec}
\titleformat{\chapter}
    {\normalfont\Huge\sffamily\color{myblue}}
    {第\thechapter 章}
    {1em}
    {}
%\titlespacing{\chapter}{0pt}{50pt}{40pt}
\titleformat{\section}
    {\normalfont\Large\sffamily\color{myblue}}
    {\thesection}
    {1em}
    {}
%\titlespacing{\section}{0pt}{3.5ex plus 1ex minus .2ex}{2.3ex plus .2ex}
\titleformat{\subsection}
    {\normalfont\large\sffamily\color{myblue}}
    {\thesubsection}
    {1em}
    {}
%\titlespacing{\subsection}{0pt}{3.25ex plus 1ex minus .2ex}{1.5ex plus .2ex}
%
\newpagestyle{special}[\small\sffamily]{
  \headrule
  \sethead[\usepage][][\chaptertitle]
  {\chaptertitle}{}{\usepage}}
\newpagestyle{main}[\small\sffamily]{
  \headrule
  \sethead[\usepage][][第\thechapter 章\quad\chaptertitle]
  {\thesection\quad\sectiontitle}{}{\usepage}}

\usepackage{titletoc}
%\setcounter{tocdepth}{1}
%\titlecontents{标题层次}[左间距]{上间距和整体格式}{标题序号}{标题内容}{指引线和页码}[下间距]
\titlecontents{chapter}[1.5em]{\vspace{.5em}\bfseries\sffamily}{\color{myblue}\contentslabel{1.5em}}{}
    {\titlerule*[20pt]{$\cdot$}\contentspage}[]
\titlecontents{section}[4.5em]{\sffamily}{\color{myblue}\contentslabel{3em}}{}
    {\titlerule*[20pt]{$\cdot$}\contentspage}[]
%\titlecontents{subsection}[8.5em]{\sffamily}{\contentslabel{4em}}{}
%    {\titlerule*[20pt]{$\cdot$}\contentspage}

\usepackage{enumitem}
\setlist{topsep=2pt,itemsep=2pt,parsep=1pt,leftmargin=\parindent}

\usepackage{fancyvrb}
\DefineVerbatimEnvironment{verbatim}{Verbatim}
  {xleftmargin=2em,baselinestretch=1,formatcom=\color{teal}\upshape}

\usepackage{etoolbox}
\makeatletter
\preto{\FV@ListVSpace}{\topsep=2pt \partopsep=0pt }
\makeatother

\usepackage[colorlinks,plainpages,pagebackref]{hyperref}
\hypersetup{
   pdfstartview={FitH},
   citecolor=teal,
   linkcolor=myblue,
   urlcolor=black,
   bookmarksnumbered
}

\usepackage{comment,makeidx,multicol}

%\usepackage{german}
%% german
%\righthyphenmin=3
%\mdqoff
%\captionsenglish
\usepackage[english]{babel}
{\catcode`"=13 \gdef"#1{\ifx#1"\discretionary{}{}{}\fi\relax}}
\def\mdqon{\catcode`"=13\relax}
\def\mdqoff{\catcode`"=12\relax}
\makeindex
\hyphenation{ex-em-pli-fies}

\newdimen\tempdima \newdimen\tempdimb

% these are fine
\def\handbreak{\\ \message{^^JManual break!!!!^^J}}
\def\nl{\protect\\}\def\n#1{{\tt #1}}
\protected\def\cs#1{\texttt{\textbackslash#1}}
\pdfstringdefDisableCommands{\def\cs#1{\textbackslash#1}}
\let\csc\cs
\def\lb{{\tt\char`\{}}\def\rb{{\tt\char`\}}}
\def\gr#1{\texorpdfstring{$\langle$#1$\rangle$}{<#1>}} %\def\gr#1{$\langle$#1$\rangle$}
\def\marg#1{{\tt \{}#1{\tt \}}}
\def\oarg#1{{\tt [}#1{\tt }}
\def\key#1{{\tt#1}}
\def\alt{}\def\altt{}%this way in manstijl
\def\ldash{\unskip\ ——\nobreak\ \ignorespaces}
\def\rdash{\unskip\nobreak\ ——\ \ignorespaces}
% check these
\def\hex{{\tt"}}
\def\ascii{{\sc ascii}}
\def\ebcdic{{\sc ebcdic}}
\def\IniTeX{Ini\TeX}\def\LamsTeX{LAMS\TeX}\def\VirTeX{Vir\TeX}
\def\AmsTeX{Ams\TeX}
\def\TeXbook{the \TeX\ book}\def\web{{\sc web}}
% needs major thinking
\newenvironment{myquote}{\list{}{%
    \topsep=2pt \partopsep=0pt%
    \leftmargin=\parindent \rightmargin=\parindent
    }\item[]}{\endlist}
\newenvironment{disp}{\begin{myquote}}{\end{myquote}}
\newenvironment{Disp}{\begin{myquote}}{\end{myquote}}
\newenvironment{tdisp}{\begin{myquote}}{\end{myquote}}
\newenvironment{example}{\begin{myquote}\noindent\itshape 例子:}{\end{myquote}}
\newenvironment{inventory}{\begin{description}\raggedright}{\end{description}}
\newenvironment{glossinventory}{\begin{description}}{\end{description}}
\def\gram#1{\gr{#1}}%???
\def\meta{\gr}% alias
%
% index
%
\def\indexterm#1{\emph{#1}\index{#1}}
\def\indextermsub#1#2{\emph{#1 #2}\index{#1!#2}}
\def\indextermbus#1#2{\emph{#1 #2}\index{#2!#1}}
\def\cindextermsub#1#2{\emph{#1#2}\index{#1!#1#2}}
\def\cindextermbus#1#2{\emph{#1#2}\index{#2!#1#2}}
\def\term#1\par{\index{#1}}
\def\howto#1\par{}
\def\cstoidx#1\par{\index{#1@\cs{#1}@}}
\def\thecstoidx#1\par{\index{#1@\protect\csname #1\endcsname}}
\def\thecstoidxsub#1#2{\index{#1, #2@\protect\csname #1\endcsname, #2}\ignorespaces}
\def\csterm#1\par{\cstoidx #1\par\cs{#1}}
\def\csidx#1{\cstoidx #1\par\cs{#1}}

\def\tmc{\tracingmacros=2 \tracingcommands\tracingmacros}

%%%%%%%%%%%%%%%%%%%
\makeatletter
\def\snugbox{\hbox\bgroup\setbox\z@\vbox\bgroup
    \leftskip\z@
    \bgroup\aftergroup\make@snug
    \let\next=}
\def\make@snug{\par\sn@gify\egroup \box\z@\egroup}
\def\sn@gify
   {\skip\z@=\lastskip \unskip
    \advance\skip\z@\lastskip \unskip
    \unpenalty
    \setbox\z@\lastbox
    \ifvoid\z@ \nointerlineskip \else {\sn@gify} \fi
    \hbox{\unhbox\z@}\nointerlineskip
    \vskip\skip\z@
    }

\newdimen\fbh \fbh=60pt % dimension for easy scaling:
\newdimen\fbw \fbw=60pt % height and width of character box

\newdimen\dh \newdimen\dw % height and width of current character box
\newdimen\lh % height of previous character box
\newdimen\lw \lw=.4pt % line weight, instead of default .4pt

\def\hdotfill{\noindent
    \leaders\hbox{\vrule width 1pt height\lw
                  \kern4pt
                  \vrule width.5pt height\lw}\hfill\hbox{}
    \par}
\def\hlinefill{\noindent
    \leaders\hbox{\vrule width 5.5pt height\lw         }\hfill\hbox{}
    \par}
\def\stippel{$\qquad\qquad\qquad\qquad$}
\makeatother
%%%%%%%%%%%%%%%%%%%

%\def\SansSerif{\Typeface:macHelvetica }
%\def\SerifFont{\Typeface:macTimes }
%\def\SansSerif{\Typeface:bsGillSans }
%\def\SerifFont{\Typeface:bsBaskerville }
\let\SansSerif\relax \def\italic{\it}
\let\SerifFont\relax \def\MainFont{\rm}
\let\SansSerif\relax
\let\SerifFont\relax
\let\PopIndentLevel\relax \let\PushIndentLevel\relax
\let\ToVerso\relax \let\ToRecto\relax

%\def\stop@command@suffix{stop}
%\let\PopListLevel\PopIndentLevel
%\let\FlushRight\relax
%\let\flushright\FlushRight
%\let\SetListIndent\LevelIndent
%\def\awp{\ifhmode\vadjust{\penalty-10000 }\else
%    \penalty-10000 \fi}
\let\awp\relax
\let\PopIndentLevel\relax \let\PopListLevel\relax

\showboxdepth=-1

%\input figs
\def\endofchapter{\vfill\noindent}

\newcommand{\liamfnote}[1]{\footnote{译注(Liam0205):#1}}
\newcommand{\cstate}[1]{状态 \textit{#1}}

\setcounter{chapter}{4}

\begin{document}

%\chapter{Boxes}\label{boxes}\index{boxes|(}
\chapter{盒子}\label{boxes}\index{盒子|(}

%The horizontal and vertical boxes of \TeX\ are containers for
%pieces of horizontal and vertical lists.
%Boxes can be stored in box registers.
%This chapter treats box registers and such
%aspects of boxes as their dimensions, and the way their components
%are placed relative to each other.
\TeX 中的水平与竖直盒子分别是水平与竖直列表内容的容器。盒子可保存在盒子寄存器当中。本章讨论盒子寄存器,以及关于盒子尺寸、盒子内容相互位置的问题。

\label{cschap:hbox}\label{cschap:vbox}\label{cschap:vtop}\label{cschap:vsplit}\label{cschap:box}\label{cschap:setbox}\label{cschap:copy}\label{cschap:ifhbox}\label{cschap:ifvbox}\label{cschap:ifvoid2}\label{cschap:newbox}\label{cschap:unhbox}\label{cschap:unvbox}\label{cschap:unhcopy}\label{cschap:unvcopy}\label{cschap:ht}\label{cschap:dp}\label{cschap:wd}\label{cschap:boxmaxdepth}\label{cschap:splitmaxdepth}\label{cschap:badness}\label{cschap:hfuzz}\label{cschap:vfuzz}\label{cschap:hbadness}\label{cschap:vbadness}\label{cschap:overfullrule}\label{cschap:hsize}\label{cschap:vsize}\label{cschap:lastbox}\label{cschap:raise}\label{cschap:lower}\label{cschap:moveleft}\label{cschap:moveright}\label{cschap:everyhbox}\label{cschap:everyvbox}
\begin{inventory}
%\item [\cs{hbox}]
%      Construct a horizontal box.
\item [\cs{hbox}] 构造一个水平盒子。
%\item [\cs{vbox}]
%      Construct a vertical box with reference point of the last item.
\item [\cs{vbox}] 构造一个竖直盒子,其参考点(reference point)是最后一个元素。
%\item [\cs{vtop}]
%      Construct a vertical box with reference point of the first item.
\item [\cs{vtop}] 构造一个竖直盒子,其参考点是第一个元素。
%\item [\cs{vcenter}]
%      Construct a vertical box vertically centred
%      on the math axis; this command can only be used in math mode.
\item [\cs{vcenter}] 构造一个在竖直方向于数学轴(math axis)居中的竖直盒子;该命令仅在数学模式下使用。
%\item [\cs{vsplit}]
%      Split off the top part of a vertical box.
\item [\cs{vsplit}] 将竖直盒子顶部部分分离。% TODO(Liam0205): 这里翻译需校正。
%\item [\cs{box}]
%      Use a box register, emptying it.
\item [\cs{box}] 使用盒子寄存器,而后清空它。
%\item [\cs{setbox}]
%      Assign a box to a box register.
\item [\cs{setbox}] 将盒子赋值给一个盒子寄存器。
%\item [\cs{copy}]
%      Use a box register, but retain the contents.
\item [\cs{copy}] 使用盒子寄存器,但保留其中内容而不清空。
%\item [\cs{ifhbox} \cs{ifvbox}]
%\mdqon
%      Test whether a box register contains a horizontal/""vertical box.
%\mdqoff
\item [\cs{ifhbox} \cs{ifvbox}] 判断盒子寄存器中保存的是水平盒子还是竖直盒子。
%\item [\cs{ifvoid}]
%      Test whether a box register is empty.
\item [\cs{ifvoid}] 判断盒子寄存器是否为空。
%\item [\cs{newbox}]
%      Allocate a new box register.
\item [\cs{newbox}] 分配新的盒子寄存器。
%\item [\cs{unhbox} \cs{unvbox}]
%      Unpack a box register containing a horizontal/vertical box,
%      adding the contents to the current horizontal/vertical list,
%      and emptying the register.
\item [\cs{unhbox} \cs{unvbox}] 将保存有水平盒子或竖直盒子的寄存器解包,并将其中内容追加到当前的水平列表或竖直列表中,而后清空寄存器。
%\item [\cs{unhcopy} \cs{unvcopy}]
%      The same as \cs{unhbox}$\,$/$\,$\cs{unvbox},
%      but do not empty the register.
\item [\cs{unhcopy} \cs{unvcopy}] 与 \cs{unhbox}、\cs{unvbox} 的行为相同,但不清空寄存器。
%\item [\cs{ht} \cs{dp} \cs{wd}]
%      Height/depth/width of the box in a box register.
\item [\cs{ht} \cs{dp} \cs{wd}] 返回盒子寄存器中盒子的高度、深度、宽度。
%\item [\cs{boxmaxdepth}]
%      Maximum allowed depth of boxes.
%      Plain \TeX\ default:~\cs{maxdimen}.
\item [\cs{boxmaxdepth}] 该命令表示 \TeX 所允许的盒子的最大深度。在 plain \TeX 中,它的默认值是~\cs{maxdimen}。
%\item [\cs{splitmaxdepth}]
%      Maximum allowed depth of boxes generated by \cs{vsplit}.
\item [\cs{splitmaxdepth}] 该命令表示 \TeX 所允许的由 \cs{vsplit} 生成盒子的最大深度。
%\item [\cs{badness}]
%      Badness of the most recently constructed box.
\item [\cs{badness}] 刚刚构造的盒子的劣度。
%\item [\cs{hfuzz} \cs{vfuzz}]
%      Excess size that \TeX\ tolerates before it considers
%\mdqon
%      a horizontal/""vertical box overfull.
%\mdqoff
\item [\cs{hfuzz} \cs{vfuzz}] 该命令表示一个水平盒子或竖直盒子超出外层盒子尺寸的阈值;当超过该阈值时,\TeX 会考虑将其认定为溢出的(overfull)盒子。
%\item [\cs{hbadness} \cs{vbadness}]
%      Amount of tolerance before \TeX\ reports an underfull
%\mdqon
%      or overfull  horizontal/""vertical box.
%\mdqoff
\item [\cs{hbadness} \cs{vbadness}] 该命令表示一个阈值;当超过该阈值时,\TeX 会将相应盒子认定为欠满的(underfull)或者溢出的。
%\item [\cs{overfullrule}]
%      Width of the rule that is printed to indicate
%      overfull horizontal boxes.
\item [\cs{overfullrule}] 当水平盒子溢出时,\TeX 会打印一条标尺,以表明有溢出的水平盒子;该命令表示上述标尺的宽度。
%\item [\cs{hsize}]
%      Line width used for text typesetting inside a vertical box.
\item [\cs{hsize}] 该命令表示在水平盒子内文本排版可用的行宽。
%\item [\cs{vsize}]
%      Height of the page box.
\item [\cs{vsize}] 该命令表示页盒子(page box)的高度。
%\item [\cs{lastbox}]
%      Register containing the last item added to the current list,
%      if this was a box.
\item [\cs{lastbox}] 若最后追加进当前列表的内容是一个盒子,则该寄存器会保存上述盒子。
%\item [\cs{raise} \cs{lower}]
%      Adjust vertical positioning of a box in horizontal mode.
\item [\cs{raise} \cs{lower}] 这两个命令用于在水平模式中调整盒子的垂直位置。
%\item [\cs{moveleft} \cs{moveright}]
%      Adjust horizontal positioning of a box in vertical mode.
\item [\cs{moveleft} \cs{moveright}] 这两个命令用于在竖直模式中调整盒子的水平位置。
%\item [\cs{everyhbox} \cs{everyvbox}]
%\mdqon
%      Token list inserted at the start of a horizontal/""vertical box.
%\mdqoff
\item [\cs{everyhbox} \cs{everyvbox}] 这两个命令分别用于在每个水平盒子或竖直盒子开头处插入的记号列表。
\end{inventory}

%\section{Boxes}
\section{Boxes}

%In this chapter we shall look at boxes. Boxes are containers
%for pieces of horizontal or vertical lists.
%Boxes that are needed more than once can be stored in box registers.
In this chapter we shall look at boxes. Boxes are containers
for pieces of horizontal or vertical lists.
Boxes that are needed more than once can be stored in box registers.

%When \TeX\ expects a \gr{box}, any of the following forms
%is admissible:
%\begin{itemize}
%\item \cs{hbox}\gr{box specification}\lb\gr{horizontal material}\rb
%\item \cs{vbox}\gr{box specification}\lb\gr{vertical material}\rb
%\item \cs{vtop}\gr{box specification}\lb\gr{vertical material}\rb
%\item \cs{box}\gr{8-bit number}
%\item \cs{copy}\gr{8-bit number}
%\item \cs{vsplit}\gr{8-bit number}\n{to}\gr{dimen}
%\item \cs{lastbox}
%\end{itemize}
%A \gr{box specification} is defined as\label{box:spec}
%\begin{disp}\gr{box specification} $\longrightarrow$ \gr{filler}
%\nl\indent$|$ \n{to} \gr{dimen}\gr{filler}
%          $|$ \n{spread} \gr{dimen}\gr{filler}
%\end{disp}
%An \gr{8-bit number} is a number in the range~0--255.
When \TeX\ expects a \gr{box}, any of the following forms
is admissible:
\begin{itemize}
\item \cs{hbox}\gr{box specification}\lb\gr{horizontal material}\rb
\item \cs{vbox}\gr{box specification}\lb\gr{vertical material}\rb
\item \cs{vtop}\gr{box specification}\lb\gr{vertical material}\rb
\item \cs{box}\gr{8-bit number}
\item \cs{copy}\gr{8-bit number}
\item \cs{vsplit}\gr{8-bit number}\n{to}\gr{dimen}
\item \cs{lastbox}
\end{itemize}
A \gr{box specification} is defined as\label{box:spec}
\begin{disp}\gr{box specification} $\longrightarrow$ \gr{filler}
\nl\indent$|$ \n{to} \gr{dimen}\gr{filler}
          $|$ \n{spread} \gr{dimen}\gr{filler}
\end{disp}
An \gr{8-bit number} is a number in the range~0--255.

%The braces surrounding box material define a group;
%they can be explicit characters
%of categories 1 and~2 respectively,
%or control sequences \cs{let} to such characters;
%see also below.
The braces surrounding box material define a group;
they can be explicit characters
of categories 1 and~2 respectively,
or control sequences \cs{let} to such characters;
see also below.


%A \gr{box} can in general be used in horizontal, vertical,
%and math mode, but see below for the \cs{lastbox}.
%The connection between
%boxes and modes is explored further in Chapter~\ref{hvmode}.
A \gr{box} can in general be used in horizontal, vertical,
and math mode, but see below for the \cs{lastbox}.
The connection between
boxes and modes is explored further in Chapter~\ref{hvmode}.

%The box produced by \cs{vcenter} \ldash a command that is allowed only in
%math mode \rdash  is not a \gr{box}. For instance,
%it can not be assigned with \verb=\setbox=; see further
%Chapter~\ref{math}.
The box produced by \cs{vcenter} \ldash a command that is allowed only in
math mode \rdash  is not a \gr{box}. For instance,
it can not be assigned with \verb=\setbox=; see further
Chapter~\ref{math}.

%The \cs{vsplit} operation is treated in Chapter~\ref{page:break}.
The \cs{vsplit} operation is treated in Chapter~\ref{page:break}.

%\section{Box registers}
\section{Box registers}

%There are 256 box registers, numbered 0--255.
%Either a box register is  empty (`void'), or it contains a horizontal
%or vertical box.
%This section discusses specifically box {\em registers};
%the sizes of boxes, and the way material is arranged inside them,
%is treated below.
There are 256 box registers, numbered 0--255.
Either a box register is  empty (`void'), or it contains a horizontal
or vertical box.
This section discusses specifically box {\em registers};
the sizes of boxes, and the way material is arranged inside them,
is treated below.

%%\spoint Allocation: \cs{newbox}
%\subsection{Allocation: \cs{newbox}}
%\spoint Allocation: \cs{newbox}
\subsection{Allocation: \cs{newbox}}

%The plain \TeX\ \csidx{newbox} macro allocates an unused
%box register:
%\begin{verbatim}
%\newbox\MyBox
%\end{verbatim}
%after which one can say
%\begin{verbatim}
%\setbox\MyBox=...
%\end{verbatim}
%or
%\begin{verbatim}
%\box\MyBox
%\end{verbatim}
%and so on.
%Subsequent calls to this macro give subsequent box numbers;
%this way macro collections can allocate their own boxes
%without fear of collision with other macros.
The plain \TeX\ \csidx{newbox} macro allocates an unused
box register:
\begin{verbatim}
\newbox\MyBox
\end{verbatim}
after which one can say
\begin{verbatim}
\setbox\MyBox=...
\end{verbatim}
or
\begin{verbatim}
\box\MyBox
\end{verbatim}
and so on.
Subsequent calls to this macro give subsequent box numbers;
this way macro collections can allocate their own boxes
without fear of collision with other macros.

%The number of the box is assigned by \cs{chardef}
%(see Chapter~\ref{alloc}).
%This implies that \cs{MyBox} is equivalent to,
%and can be used as, a~\gr{number}.
%The control sequence
%\altt
%\cs{newbox} is an \cs{outer} macro.
%Newly allocated box registers are initially empty.
The number of the box is assigned by \cs{chardef}
(see Chapter~\ref{alloc}).
This implies that \cs{MyBox} is equivalent to,
and can be used as, a~\gr{number}.
The control sequence
\altt
\cs{newbox} is an \cs{outer} macro.
Newly allocated box registers are initially empty.


%\subsection{Usage: \cs{setbox}, \cs{box}, \cs{copy}}
\subsection{Usage: \cs{setbox}, \cs{box}, \cs{copy}}

%A~register is filled by assigning a \gr{box}
%\cstoidx setbox\par
%to it:
%\begin{Disp}\verb>\setbox>\gr{number}\gr{equals}\gr{box}\end{Disp}
%For example, the \gr{box} can be explicit
%\begin{Disp}\verb>\setbox37=\hbox{...}>\quad or\quad \verb>\setbox37=\vbox{...}>
%\end{Disp}
%or it can be a box register:
%\begin{verbatim}
%\setbox37=\box38
%\end{verbatim}
%Usually, box numbers will have been assigned by a \cs{newbox}
%command.
A~register is filled by assigning a \gr{box}
\cstoidx setbox\par
to it:
\begin{Disp}\verb>\setbox>\gr{number}\gr{equals}\gr{box}\end{Disp}
For example, the \gr{box} can be explicit
\begin{Disp}\verb>\setbox37=\hbox{...}>\quad or\quad \verb>\setbox37=\vbox{...}>
\end{Disp}
or it can be a box register:
\begin{verbatim}
\setbox37=\box38
\end{verbatim}
Usually, box numbers will have been assigned by a \cs{newbox}
command.

%The box in a box register is appended
%by the commands \cs{box} and~\cs{copy}
%to whatever list \TeX\ is building: the call
%\begin{verbatim}
%\box38
%\end{verbatim}
%appends box~38.
%To save memory space, box registers become empty by using them:
%\TeX\ assumes that after you have inserted a box by
%calling \csidx{box}$nn$ in some mode, you do not need the
%contents of that register any more and empties it.
%In case you {\em do\/} need the contents of
%a box register more than once,
%you can \csidx{copy} it. Calling \cs{copy}$nn$ is
%equivalent to \cs{box}$nn$ in all respects except that
%the register is not cleared.
The box in a box register is appended
by the commands \cs{box} and~\cs{copy}
to whatever list \TeX\ is building: the call
\begin{verbatim}
\box38
\end{verbatim}
appends box~38.
To save memory space, box registers become empty by using them:
\TeX\ assumes that after you have inserted a box by
calling \csidx{box}$nn$ in some mode, you do not need the
contents of that register any more and empties it.
In case you {\em do\/} need the contents of
a box register more than once,
you can \csidx{copy} it. Calling \cs{copy}$nn$ is
equivalent to \cs{box}$nn$ in all respects except that
the register is not cleared.

%It is possible to unwrap the contents of a box register
%by `unboxing' it using the commands \cs{unhbox} and \cs{unvbox},
%and their copying versions \cs{unhcopy} and \cs{unvcopy}.
%Whereas a box can be used in any mode, the
%unboxing operations can only be used in the appropriate mode,
%since in effect they contribute a partial
%horizontal or vertical list (see also Chapter~\ref{hvmode}).
%See below for more information on unboxing registers.
It is possible to unwrap the contents of a box register
by `unboxing' it using the commands \cs{unhbox} and \cs{unvbox},
and their copying versions \cs{unhcopy} and \cs{unvcopy}.
Whereas a box can be used in any mode, the
unboxing operations can only be used in the appropriate mode,
since in effect they contribute a partial
horizontal or vertical list (see also Chapter~\ref{hvmode}).
See below for more information on unboxing registers.

%%\spoint Testing: \cs{ifvoid}, \cs{ifhbox}, \cs{ifvbox}
%\subsection{Testing: \cs{ifvoid}, \cs{ifhbox}, \cs{ifvbox}}
%\spoint Testing: \cs{ifvoid}, \cs{ifhbox}, \cs{ifvbox}
\subsection{Testing: \cs{ifvoid}, \cs{ifhbox}, \cs{ifvbox}}

%Box
%registers can be tested for their contents:
%\begin{disp}\cs{ifvoid}\gr{number}\end{disp}
%is true if the box register is empty.
%Note that an empty, or `void',
%box register is not the same as a register containing an empty box.
%An empty box is still either a horizontal or a vertical box;
%a~void register can be used as both.
Box
registers can be tested for their contents:
\begin{disp}\cs{ifvoid}\gr{number}\end{disp}
is true if the box register is empty.
Note that an empty, or `void',
box register is not the same as a register containing an empty box.
An empty box is still either a horizontal or a vertical box;
a~void register can be used as both.

%The test
%\begin{disp}\cs{ifhbox}\gr{number}\end{disp}
%is true if the box register contains a horizontal box;
%\begin{disp}\cs{ifvbox}\gr{number}\end{disp}
%is true if the box register contains a vertical box.
%Both tests are false for void registers.
The test
\begin{disp}\cs{ifhbox}\gr{number}\end{disp}
is true if the box register contains a horizontal box;
\begin{disp}\cs{ifvbox}\gr{number}\end{disp}
is true if the box register contains a vertical box.
Both tests are false for void registers.

%%\spoint[lastbox] The \cs{lastbox}
%\subsection{The \cs{lastbox}}
%\label{lastbox}
%\spoint[lastbox] The \cs{lastbox}
\subsection{The \cs{lastbox}}
\label{lastbox}

%When \TeX\ has built a partial list, the last box in this
%list is accessible as the \csidx{lastbox}. This behaves
%like a box register, so you can remove the last box from  the
%list by assigning the \cs{lastbox} to some  box register.
%If the last item on the current list is not a box,
%the \cs{lastbox} acts like a void box register.
%It is not possible to get hold of the last box
%in the case of the main vertical list.
%The \cs{lastbox} is then always void.
When \TeX\ has built a partial list, the last box in this
list is accessible as the \csidx{lastbox}. This behaves
like a box register, so you can remove the last box from  the
list by assigning the \cs{lastbox} to some  box register.
If the last item on the current list is not a box,
the \cs{lastbox} acts like a void box register.
It is not possible to get hold of the last box
in the case of the main vertical list.
The \cs{lastbox} is then always void.

%As an example, the statement
%\begin{verbatim}
%{\setbox0=\lastbox}
%\end{verbatim}
%removes
%the last box from the current list, assigning it to box
%register~0. Since this assignment occurs inside a group,
%the register is cleared at the end of the group.
%At the start of a paragraph this can be used to remove the
%indentation box (see Chapter~\ref{par:start}).
%Another example of \cs{lastbox} can be found on page~\pageref{varioset}.
As an example, the statement
\begin{verbatim}
{\setbox0=\lastbox}
\end{verbatim}
removes
the last box from the current list, assigning it to box
register~0. Since this assignment occurs inside a group,
the register is cleared at the end of the group.
At the start of a paragraph this can be used to remove the
indentation box (see Chapter~\ref{par:start}).
Another example of \cs{lastbox} can be found on page~\pageref{varioset}.

%Because the \verb-\lastbox- is always empty in external vertical mode,
%it is not possible to get hold of boxes that have been
%added to the page. However, it is possible to dissect
%the page once it is in \cs{box255}, for instance doing
%\begin{verbatim}
%\vbox{\unvbox255{\setbox0=\lastbox}}
%\end{verbatim}
%inside the output routine.
Because the \verb-\lastbox- is always empty in external vertical mode,
it is not possible to get hold of boxes that have been
added to the page. However, it is possible to dissect
the page once it is in \cs{box255}, for instance doing
\begin{verbatim}
\vbox{\unvbox255{\setbox0=\lastbox}}
\end{verbatim}
inside the output routine.

%If boxes in vertical mode have been shifted by \cs{moveright}
%or \cs{moveleft}, or if boxes in horizontal mode  have
%been raised by \cs{raise} or lowered by \cs{lower},
%any information about this
%displacement due to such a command is lost when
%the \cs{lastbox} is taken from the list.
If boxes in vertical mode have been shifted by \cs{moveright}
or \cs{moveleft}, or if boxes in horizontal mode  have
been raised by \cs{raise} or lowered by \cs{lower},
any information about this
displacement due to such a command is lost when
the \cs{lastbox} is taken from the list.

%%\point Natural dimensions of boxes
%\section{Natural dimensions of boxes}
%\point Natural dimensions of boxes
\section{Natural dimensions of boxes}

%%\spoint Dimensions of created horizontal boxes
%\subsection{Dimensions of created horizontal boxes}
%\spoint Dimensions of created horizontal boxes
\subsection{Dimensions of created horizontal boxes}

%Inside an \csidx{hbox} all constituents are lined up next to each other,
%with their reference points on the baseline of the box,
%unless they are moved explicitly in the vertical direction
%by \cs{lower} or~\cs{raise}.
Inside an \csidx{hbox} all constituents are lined up next to each other,
with their reference points on the baseline of the box,
unless they are moved explicitly in the vertical direction
by \cs{lower} or~\cs{raise}.

%The resulting width of the box is the sum of the widths
%of the components. Thus the width of
%\begin{verbatim}
%\hbox{\hskip1cm}
%\end{verbatim}
%is positive, and the width of
%\begin{verbatim}
%\hbox{\hskip-1cm}
%\end{verbatim}
%is negative. By way of example,
%\begin{disp}\verb>a\hbox{\kern-1em b}-->\end{disp}
%gives as output
%\begin{disp}\leavevmode\hphantom{b}a\hbox{\kern-1em b}--\end{disp}
%\message{check align input/output}
%which shows that a horizontal box can have negative
%width.
The resulting width of the box is the sum of the widths
of the components. Thus the width of
\begin{verbatim}
\hbox{\hskip1cm}
\end{verbatim}
is positive, and the width of
\begin{verbatim}
\hbox{\hskip-1cm}
\end{verbatim}
is negative. By way of example,
\begin{disp}\verb>a\hbox{\kern-1em b}-->\end{disp}
gives as output
\begin{disp}\leavevmode\hphantom{b}a\hbox{\kern-1em b}--\end{disp}
\message{check align input/output}
which shows that a horizontal box can have negative
width.

%The height and depth of an \cs{hbox} are the
%maximum amount that constituent boxes project above and
%below the baseline of the box. They are non-negative when the
%box is created.
The height and depth of an \cs{hbox} are the
maximum amount that constituent boxes project above and
below the baseline of the box. They are non-negative when the
box is created.

%The commands \cs{lower} and \cs{raise} are the only possibilities
%for vertical movement inside an \cs{hbox} (other than
%including a \cs{vbox} inside the \cs{hbox}, of course);
%a~\gr{vertical command} \ldash such as \cs{vskip} \rdash
%is not allowed in a horizontal box, and
%\cs{par}, although allowed,
%does not do anything inside a horizontal box.
The commands \cs{lower} and \cs{raise} are the only possibilities
for vertical movement inside an \cs{hbox} (other than
including a \cs{vbox} inside the \cs{hbox}, of course);
a~\gr{vertical command} \ldash such as \cs{vskip} \rdash
is not allowed in a horizontal box, and
\cs{par}, although allowed,
does not do anything inside a horizontal box.

%%\spoint Dimensions of created vertical boxes
%\subsection{Dimensions of created vertical boxes}
%\spoint Dimensions of created vertical boxes
\subsection{Dimensions of created vertical boxes}

%Inside a \csidx{vbox} vertical material is lined up with the
%\cstoidx vtop\par
%reference points on the vertical line through the reference
%point of the box,
%unless components are moved explicitly in the horizontal direction
%by \csidx{moveleft} or~\csidx{moveright}.
Inside a \csidx{vbox} vertical material is lined up with the
\cstoidx vtop\par
reference points on the vertical line through the reference
point of the box,
unless components are moved explicitly in the horizontal direction
by \csidx{moveleft} or~\csidx{moveright}.

%The  reference point of a vertical box
%is always located at the left boundary of the box.
%The width of a vertical box
%is then the maximal amount that any material in the
%box sticks to the right of the reference point.
%Material to the left of the reference point is
%not taken into account in the width.
%Thus the result of
%\begin{disp}\verb>a\vbox{\hbox{\kern-1em b}}-->\end{disp}
%is
%\begin{disp}\leavevmode\hphantom{b}a\vbox{\hbox{\kern-1em b}}--\end{disp}
%This should be contrasted with the above example.
The  reference point of a vertical box
is always located at the left boundary of the box.
The width of a vertical box
is then the maximal amount that any material in the
box sticks to the right of the reference point.
Material to the left of the reference point is
not taken into account in the width.
Thus the result of
\begin{disp}\verb>a\vbox{\hbox{\kern-1em b}}-->\end{disp}
is
\begin{disp}\leavevmode\hphantom{b}a\vbox{\hbox{\kern-1em b}}--\end{disp}
This should be contrasted with the above example.


%The calculation of height and depth is different
%for vertical boxes constructed by \cs{vbox} and \cs{vtop}.
%The ground rule is that
%a \cs{vbox} has a reference point that lies on
%the baseline of its last component,
%and a \cs{vtop} has its reference point on the baseline of the
%first component.
%In general, the depth (height) of a \cs{vbox} (\cs{vtop})
%\alt
%can be non-zero if the last (first) item is a box or rule.
The calculation of height and depth is different
for vertical boxes constructed by \cs{vbox} and \cs{vtop}.
The ground rule is that
a \cs{vbox} has a reference point that lies on
the baseline of its last component,
and a \cs{vtop} has its reference point on the baseline of the
first component.
In general, the depth (height) of a \cs{vbox} (\cs{vtop})
\alt
can be non-zero if the last (first) item is a box or rule.

%The height of a \cs{vbox} is then the sum of the heights and
%depths of all components except the last, plus the height
%of that last component; the depth of the \cs{vbox} is the
%depth of its last component.
%The depth of a \cs{vtop}
%is the sum of the depth of the first component and the heights
%and depths of all subsequent material; its height is the
%height of the first component.
The height of a \cs{vbox} is then the sum of the heights and
depths of all components except the last, plus the height
of that last component; the depth of the \cs{vbox} is the
depth of its last component.
The depth of a \cs{vtop}
is the sum of the depth of the first component and the heights
and depths of all subsequent material; its height is the
height of the first component.

%However, the actual rules are a bit
%more complicated when the first component of a \cs{vtop}
%or the last component of a \cs{vbox} is not a box or rule.
%If the last component of a \cs{vbox} is a kern or a glue,
%the depth of that box is zero; a \cs{vtop}'s
%height is zero
%unless its first component is a box or rule.
%\altt
%(Note the asymmetry in these definitions; see below for
%an example illustrating this.)
%The depth of a \cs{vtop}, then, is equal to the total
%height plus depth of all enclosed material minus
%the height of the \cs{vtop}.
However, the actual rules are a bit
more complicated when the first component of a \cs{vtop}
or the last component of a \cs{vbox} is not a box or rule.
If the last component of a \cs{vbox} is a kern or a glue,
the depth of that box is zero; a \cs{vtop}'s
height is zero
unless its first component is a box or rule.
\altt
(Note the asymmetry in these definitions; see below for
an example illustrating this.)
The depth of a \cs{vtop}, then, is equal to the total
height plus depth of all enclosed material minus
the height of the \cs{vtop}.

%There is a limit on the depth of vertical boxes:
%if the depth of a \cs{vbox} or \cs{vtop}
%calculated by the above rules would exceed
%\cstoidx boxmaxdepth\par,
%the reference point of the box
%is moved down by the excess amount.
%More precisely, the excess depth is added to the
%natural height of the box. If the box had a \n{to} or
%\n{spread} specification, any glue is set anew to take
%the new height into account.
There is a limit on the depth of vertical boxes:
if the depth of a \cs{vbox} or \cs{vtop}
calculated by the above rules would exceed
\cstoidx boxmaxdepth\par,
the reference point of the box
is moved down by the excess amount.
More precisely, the excess depth is added to the
natural height of the box. If the box had a \n{to} or
\n{spread} specification, any glue is set anew to take
the new height into account.

%Ordinarily,
%\cs{boxmaxdepth} is set to the maximum dimension
%possible in \TeX. It is for instance reduced during some of
%the calculations  in the plain \TeX\ output routine;
%see Chapter~\ref{output}.
Ordinarily,
\cs{boxmaxdepth} is set to the maximum dimension
possible in \TeX. It is for instance reduced during some of
the calculations  in the plain \TeX\ output routine;
see Chapter~\ref{output}.

%\subsection{Examples}
\subsection{Examples}

%Horizontal boxes are relatively straightforward. Their width is the
%distance between the `beginning' and the `end' of the
%box,
%and consequently the width is not necessarily positive.
%With
%\begin{verbatim}
%\setbox0=\hbox{aa} \setbox1=\hbox{\copy0 \hskip-\wd0}
%\end{verbatim}
%the \cs{box1} has width zero;
%\begin{Disp} \verb-/\box1/-\quad gives\quad
%`{\setbox0=\hbox{aa}\setbox1=\hbox{\copy0 \hskip-\wd0}/\box1/}\kern.75em'
%\end{Disp}
%The height and depth of a horizontal box cannot be negative: in
%\begin{verbatim}
%\setbox0=\hbox{\vrule height 5pt depth 5pt}
%\setbox1=\hbox{\raise 10pt \box0}
%\end{verbatim}
%the \cs{box1} has depth \n{0pt} and height~\n{15pt}
Horizontal boxes are relatively straightforward. Their width is the
distance between the `beginning' and the `end' of the
box,
and consequently the width is not necessarily positive.
With
\begin{verbatim}
\setbox0=\hbox{aa} \setbox1=\hbox{\copy0 \hskip-\wd0}
\end{verbatim}
the \cs{box1} has width zero;
\begin{Disp} \verb-/\box1/-\quad gives\quad
`{\setbox0=\hbox{aa}\setbox1=\hbox{\copy0 \hskip-\wd0}/\box1/}\kern.75em'
\end{Disp}
The height and depth of a horizontal box cannot be negative: in
\begin{verbatim}
\setbox0=\hbox{\vrule height 5pt depth 5pt}
\setbox1=\hbox{\raise 10pt \box0}
\end{verbatim}
the \cs{box1} has depth \n{0pt} and height~\n{15pt}

%Vertical boxes are more troublesome than horizontal boxes.
%Let us first treat their width.
%After
%\begin{verbatim}
%\setbox0=\hbox{\hskip 10pt}
%\end{verbatim}
%the box in the
%\cs{box0} register has a width of \n{10pt}. Defining
%\begin{verbatim}
%\setbox1=\vbox{\moveleft 5pt \copy0}
%\end{verbatim}
%the \cs{box1} will have width \n{5pt}; material to the
%left of the reference point is not accounted for in the
%width of a vertical box. With
%\begin{verbatim}
%\setbox2=\vbox{\moveright 5pt \copy0}
%\end{verbatim}
%the \cs{box2} will have width \n{15pt}.
Vertical boxes are more troublesome than horizontal boxes.
Let us first treat their width.
After
\begin{verbatim}
\setbox0=\hbox{\hskip 10pt}
\end{verbatim}
the box in the
\cs{box0} register has a width of \n{10pt}. Defining
\begin{verbatim}
\setbox1=\vbox{\moveleft 5pt \copy0}
\end{verbatim}
the \cs{box1} will have width \n{5pt}; material to the
left of the reference point is not accounted for in the
width of a vertical box. With
\begin{verbatim}
\setbox2=\vbox{\moveright 5pt \copy0}
\end{verbatim}
the \cs{box2} will have width \n{15pt}.

%The depth of a \cs{vbox} is the depth of the last item if
%that is a box, so
%\begin{verbatim}
%\vbox{\vskip 5pt \hbox{\vrule height 5pt depth 5pt}}
%\end{verbatim}
%has height \n{10pt} and depth \n{5pt},
%and
%\begin{verbatim}
%\vbox{\vskip -5pt \hbox{\vrule height 5pt depth 5pt}}
%\end{verbatim}
%has height \n{0pt} and depth~\n{5pt}.
%With a glue or kern as the last item in the box, the resulting depth
%is zero, so
%\begin{verbatim}
%\vbox{\hbox{\vrule height 5pt depth 5pt}\vskip 5pt}
%\end{verbatim}
%has height \n{15pt} and depth~\n{0pt};
%\begin{verbatim}
%\vbox{\hbox{\vrule height 5pt depth 5pt}\vskip -5pt}
%\end{verbatim}
%has height \n{5pt} and depth~\n{0pt}.
The depth of a \cs{vbox} is the depth of the last item if
that is a box, so
\begin{verbatim}
\vbox{\vskip 5pt \hbox{\vrule height 5pt depth 5pt}}
\end{verbatim}
has height \n{10pt} and depth \n{5pt},
and
\begin{verbatim}
\vbox{\vskip -5pt \hbox{\vrule height 5pt depth 5pt}}
\end{verbatim}
has height \n{0pt} and depth~\n{5pt}.
With a glue or kern as the last item in the box, the resulting depth
is zero, so
\begin{verbatim}
\vbox{\hbox{\vrule height 5pt depth 5pt}\vskip 5pt}
\end{verbatim}
has height \n{15pt} and depth~\n{0pt};
\begin{verbatim}
\vbox{\hbox{\vrule height 5pt depth 5pt}\vskip -5pt}
\end{verbatim}
has height \n{5pt} and depth~\n{0pt}.

%The height of a \cs{vtop} behaves (almost) the same with respect to
%the first item of the box, as the depth of a \cs{vbox} does
%with respect to the last item. Repeating the above examples with
%a \cs{vtop} gives the following:
%\begin{verbatim}
%\vtop{\vskip 5pt \hbox{\vrule height 5pt depth 5pt}}
%\end{verbatim}
%has height \n{0pt} and depth \n{15pt},
%and
%\begin{verbatim}
%\vtop{\vskip -5pt \hbox{\vrule height 5pt depth 5pt}}
%\end{verbatim}
%has height \n{0pt} and depth~\n{5pt};
%\begin{verbatim}
%\vtop{\hbox{\vrule height 5pt depth 5pt} \vskip 5pt}
%\end{verbatim}
%has height \n{5pt} and depth~\n{10pt}, and
%\begin{verbatim}
%\vtop{\hbox{\vrule height 5pt depth 5pt} \vskip -5pt}
%\end{verbatim}
%has height \n{5pt} and depth~\n{0pt}.
The height of a \cs{vtop} behaves (almost) the same with respect to
the first item of the box, as the depth of a \cs{vbox} does
with respect to the last item. Repeating the above examples with
a \cs{vtop} gives the following:
\begin{verbatim}
\vtop{\vskip 5pt \hbox{\vrule height 5pt depth 5pt}}
\end{verbatim}
has height \n{0pt} and depth \n{15pt},
and
\begin{verbatim}
\vtop{\vskip -5pt \hbox{\vrule height 5pt depth 5pt}}
\end{verbatim}
has height \n{0pt} and depth~\n{5pt};
\begin{verbatim}
\vtop{\hbox{\vrule height 5pt depth 5pt} \vskip 5pt}
\end{verbatim}
has height \n{5pt} and depth~\n{10pt}, and
\begin{verbatim}
\vtop{\hbox{\vrule height 5pt depth 5pt} \vskip -5pt}
\end{verbatim}
has height \n{5pt} and depth~\n{0pt}.

%%\point More about box dimensions
%\section{More about box dimensions}
%\point More about box dimensions
\section{More about box dimensions}

%%\spoint Predetermined dimensions
%\subsection{Predetermined dimensions}
%\spoint Predetermined dimensions
\subsection{Predetermined dimensions}

%The size of a box can be specified in advance
%with a \gr{box specification}; see above for the syntax.
%Any glue
%in the box is then set in order to reach the required size.
%Prescribing the size of the box is done by
%\begin{disp}\cs{hbox} \n{to} \gr{dimen} \n{\lb...\rb},
%     \cs{vbox} \n{to} \gr{dimen} \n{\lb...\rb}\end{disp}
%If stretchable or shrinkable glue is present in the box,
%it is stretched or shrunk in order to give the box the
%specified size. Associated with this glue setting is a badness value
%(see Chapter~\ref{glue}). If no stretch or shrink \ldash whichever
%is necessary \rdash  is present, the resulting box will be underfull
%or overfull respectively. Error reporting for over/underfull
%boxes is treated below.
The size of a box can be specified in advance
with a \gr{box specification}; see above for the syntax.
Any glue
in the box is then set in order to reach the required size.
Prescribing the size of the box is done by
\begin{disp}\cs{hbox} \n{to} \gr{dimen} \n{\lb...\rb},
     \cs{vbox} \n{to} \gr{dimen} \n{\lb...\rb}\end{disp}
If stretchable or shrinkable glue is present in the box,
it is stretched or shrunk in order to give the box the
specified size. Associated with this glue setting is a badness value
(see Chapter~\ref{glue}). If no stretch or shrink \ldash whichever
is necessary \rdash  is present, the resulting box will be underfull
or overfull respectively. Error reporting for over/underfull
boxes is treated below.

%Another command to let a box have a size other than
%the natural size is
%\begin{disp}\cs{hbox} \n{spread} \gr{dimen} \n{\lb...\rb},
%     \cs{vbox} \n{spread} \gr{dimen} \n{\lb...\rb}\end{disp}
%which tells \TeX\ to set the glue in such a way that
%the size of the box is a  specified amount more than the
%natural size.
Another command to let a box have a size other than
the natural size is
\begin{disp}\cs{hbox} \n{spread} \gr{dimen} \n{\lb...\rb},
     \cs{vbox} \n{spread} \gr{dimen} \n{\lb...\rb}\end{disp}
which tells \TeX\ to set the glue in such a way that
the size of the box is a  specified amount more than the
natural size.

%Box specifications for \cs{vtop} vertical boxes are
%somewhat difficult to interpret. \TeX\ constructs a \cs{vtop}
%by first making a \cs{vbox}, including
%glue settings induced by a \gr{box specification};
%then it computes the height and depth by the above rules.
%Glue setting is described in Chapter~\ref{glue}.
Box specifications for \cs{vtop} vertical boxes are
somewhat difficult to interpret. \TeX\ constructs a \cs{vtop}
by first making a \cs{vbox}, including
glue settings induced by a \gr{box specification};
then it computes the height and depth by the above rules.
Glue setting is described in Chapter~\ref{glue}.

%%\spoint Changes to box dimensions
%\subsection{Changes to box dimensions}
%\spoint Changes to box dimensions
\subsection{Changes to box dimensions}

%The dimensions of a box register are accessible by the
%commands \csidx{ht}, \csidx{dp}, and~\csidx{wd};
%for instance \cs{dp13} gives the depth of box~13.
%However, not only can boxes be measured this way;
%by assigning values to these
%dimensions \TeX\ can even be fooled into thinking that
%a box has a  size different from its actual.
%However, changing the dimensions of a box does not change
%anything about the contents; in particular it does not
%change the way the glue is set.
The dimensions of a box register are accessible by the
commands \csidx{ht}, \csidx{dp}, and~\csidx{wd};
for instance \cs{dp13} gives the depth of box~13.
However, not only can boxes be measured this way;
by assigning values to these
dimensions \TeX\ can even be fooled into thinking that
a box has a  size different from its actual.
However, changing the dimensions of a box does not change
anything about the contents; in particular it does not
change the way the glue is set.


%Various formats use this in `smash' macros: the macro defined by
%\cstoidx smash\par
%\begin{verbatim}
%\def\smash#1{{\setbox0=\hbox{#1}\dp0=0pt \ht0=0pt \box0\relax}}
%\end{verbatim}
%places its argument but annihilates its height and depth;
%\altt
%that is, the output does show the whole box, but further calculations
%by \TeX\ act as if the height and depth were zero.
Various formats use this in `smash' macros: the macro defined by
\cstoidx smash\par
\begin{verbatim}
\def\smash#1{{\setbox0=\hbox{#1}\dp0=0pt \ht0=0pt \box0\relax}}
\end{verbatim}
places its argument but annihilates its height and depth;
\altt
that is, the output does show the whole box, but further calculations
by \TeX\ act as if the height and depth were zero.

%Box dimensions can be changed only by setting them.
%They are \gr{box dimen}s, which can only be set
%in a \gr{box size assignment}, and not, for instance
%changed with \cs{advance}.
Box dimensions can be changed only by setting them.
They are \gr{box dimen}s, which can only be set
in a \gr{box size assignment}, and not, for instance
changed with \cs{advance}.

%Note that a \gr{box size assignment}\index{assignment!box size} is a
%\gr{global assignment}\index{assignment!global} its effect transcends
%  any groups in which it occurs (see Chapter~\ref{group}).  Thus the
%  output of
%\begin{verbatim}
%\setbox0=\hbox{---} {\wd0=0pt} a\box0b
%\end{verbatim}
%is `{\setbox0=\hbox{---}{\wd0=0pt}a\box0b}\kern.5em'.
Note that a \gr{box size assignment}\index{assignment!box size} is a
\gr{global assignment}\index{assignment!global} its effect transcends
  any groups in which it occurs (see Chapter~\ref{group}).  Thus the
  output of
\begin{verbatim}
\setbox0=\hbox{---} {\wd0=0pt} a\box0b
\end{verbatim}
is `{\setbox0=\hbox{---}{\wd0=0pt}a\box0b}\kern.5em'.

%The limits that hold on the dimensions with which a
%box can be created (see above) do not hold for explicit changes to the
%\mdqon
%size of a box: the assignment \cs{dp0=}""\n{-2pt} for a
%\mdqoff
%horizontal box is perfectly admissible.
The limits that hold on the dimensions with which a
box can be created (see above) do not hold for explicit changes to the
\mdqon
size of a box: the assignment \cs{dp0=}""\n{-2pt} for a
\mdqoff
horizontal box is perfectly admissible.

%\subsection{Moving boxes around}
\subsection{Moving boxes around}

%In a horizontal box all constituent elements are lined up
%\cstoidx raise\par\cstoidx lower\par
%with their reference points at the same height as the
%reference point of the box. Any box inside a horizontal
%box can be lifted or dropped using the macros
%\cs{raise} and~\cs{lower}.
In a horizontal box all constituent elements are lined up
\cstoidx raise\par\cstoidx lower\par
with their reference points at the same height as the
reference point of the box. Any box inside a horizontal
box can be lifted or dropped using the macros
\cs{raise} and~\cs{lower}.

%Similarly, in a vertical box all constituent elements
%are lined up with their reference points underneath one another,
%in line with the reference point of the box.
%Boxes can now be moved sideways by the macros
%\csidx{moveleft} and~\csidx{moveright}.
Similarly, in a vertical box all constituent elements
are lined up with their reference points underneath one another,
in line with the reference point of the box.
Boxes can now be moved sideways by the macros
\csidx{moveleft} and~\csidx{moveright}.

%Only boxes can be shifted thus; these operations cannot
%be applied to, for instance, characters or rules.
Only boxes can be shifted thus; these operations cannot
be applied to, for instance, characters or rules.

%\subsection{Box dimensions and box placement}
\subsection{Box dimensions and box placement}

%\TeX\ places the components of horizontal and
%vertical lists by maintaining a reference line and a
%current position on that line. For horizontal lists
%the reference line is the baseline of the surrounding
%\cs{hbox}; for vertical lists it is the vertical line
%through the reference point of the surrounding \cs{vbox}.
\TeX\ places the components of horizontal and
vertical lists by maintaining a reference line and a
current position on that line. For horizontal lists
the reference line is the baseline of the surrounding
\cs{hbox}; for vertical lists it is the vertical line
through the reference point of the surrounding \cs{vbox}.

%In horizontal mode a component is placed as follows.
%The current position coincides initially
%with the reference point of the surrounding box. After that,
%the following actions are carried out.
%\begin{enumerate} \item If the component has been shifted by
%\cs{raise} or \cs{lower}, shift the current
%position correspondingly.
%\item If the component is a horizontal box, use
%this algorithm recursively for its contents;
%if it is a vertical box, go up  by the height of this box,
%putting  a new current position for the enclosed vertical list there,
%and place its components using the algorithm for vertical
%lists below.
%\item Move the current position (on the reference line)
%to the right by the width of the component.
%\end{enumerate}
In horizontal mode a component is placed as follows.
The current position coincides initially
with the reference point of the surrounding box. After that,
the following actions are carried out.
\begin{enumerate} \item If the component has been shifted by
\cs{raise} or \cs{lower}, shift the current
position correspondingly.
\item If the component is a horizontal box, use
this algorithm recursively for its contents;
if it is a vertical box, go up  by the height of this box,
putting  a new current position for the enclosed vertical list there,
and place its components using the algorithm for vertical
lists below.
\item Move the current position (on the reference line)
to the right by the width of the component.
\end{enumerate}

%For the list in a vertical box \TeX's  current position is
%initially at the upper left corner of that box, as explained above,
%and the reference line is the vertical line through that point;
%it also runs through the reference point of the box.
%Enclosed components are then placed as follows.
%\begin{enumerate} \item If a component has been shifted using
%\cs{moveleft} or \cs{moveright}, shift the current position
%accordingly.
%\item Put the component with its upper left corner at the
%current position.
%\item If the component is a vertical box, use this algorithm
%recursively for its contents; if it is a horizontal box,
%its reference point can be found below  the current position
%by the height of the box. Put the current position for that
%box there, and use the above algorithm for horizontal lists.
%\item Go down by the height plus depth of the box
%(that is, starting at the upper left corner of the box)
%on the  reference line,
%and continue processing vertically.
%\end{enumerate}
%Note that the above processes do not describe the construction
%of boxes. That would (for instance)
%involve for vertical boxes the insertion
%of baselineskip glue. Rather, it describes the way the components
%of a finished box are arranged in the output.
For the list in a vertical box \TeX's  current position is
initially at the upper left corner of that box, as explained above,
and the reference line is the vertical line through that point;
it also runs through the reference point of the box.
Enclosed components are then placed as follows.
\begin{enumerate} \item If a component has been shifted using
\cs{moveleft} or \cs{moveright}, shift the current position
accordingly.
\item Put the component with its upper left corner at the
current position.
\item If the component is a vertical box, use this algorithm
recursively for its contents; if it is a horizontal box,
its reference point can be found below  the current position
by the height of the box. Put the current position for that
box there, and use the above algorithm for horizontal lists.
\item Go down by the height plus depth of the box
(that is, starting at the upper left corner of the box)
on the  reference line,
and continue processing vertically.
\end{enumerate}
Note that the above processes do not describe the construction
of boxes. That would (for instance)
involve for vertical boxes the insertion
of baselineskip glue. Rather, it describes the way the components
of a finished box are arranged in the output.

%\subsection{Boxes and negative glue}
\subsection{Boxes and negative glue}

%Sometimes it is useful to have boxes overlapping instead of
%line up. An easy way to do this is to use negative glue.
%In horizontal mode
%\begin{verbatim}
%{\dimen0=\wd8 \box8 \kern-\dimen0}
%\end{verbatim}
%places box 8 without moving the current location.
Sometimes it is useful to have boxes overlapping instead of
line up. An easy way to do this is to use negative glue.
In horizontal mode
\begin{verbatim}
{\dimen0=\wd8 \box8 \kern-\dimen0}
\end{verbatim}
places box 8 without moving the current location.

%More versatile are the macros \csidx{llap} and \csidx{rlap}\label{rlap},
%defined as
%\begin{verbatim}
%\def\llap#1{\hbox to 0pt{\hss #1}}
%\end{verbatim}
%and
%\begin{verbatim}
%\def\rlap#1{\hbox to 0pt{#1\hss}}
%\end{verbatim}
%that allow material to protrude left or right from the
%current location.
%The \cs{hss} glue is equivalent to \verb>\hskip 0pt plus 1fil minus 1fil>,
%which absorbs any positive or negative width
%of the argument of \cs{llap} or \cs{rlap}.
More versatile are the macros \csidx{llap} and \csidx{rlap}\label{rlap},
defined as
\begin{verbatim}
\def\llap#1{\hbox to 0pt{\hss #1}}
\end{verbatim}
and
\begin{verbatim}
\def\rlap#1{\hbox to 0pt{#1\hss}}
\end{verbatim}
that allow material to protrude left or right from the
current location.
The \cs{hss} glue is equivalent to \verb>\hskip 0pt plus 1fil minus 1fil>,
which absorbs any positive or negative width
of the argument of \cs{llap} or \cs{rlap}.

%\begin{example} The sequence
%\begin{verbatim}
%\llap{\hbox to 10pt{a\hfil}}
%\end{verbatim}
%is effectively the same as
%\begin{verbatim}
%\hbox{\hskip-10pt \hbox to 10pt{a\hfil}}
%\end{verbatim}
%which has a total width of~\n{0pt}.
%\end{example}
\begin{example} The sequence
\begin{verbatim}
\llap{\hbox to 10pt{a\hfil}}
\end{verbatim}
is effectively the same as
\begin{verbatim}
\hbox{\hskip-10pt \hbox to 10pt{a\hfil}}
\end{verbatim}
which has a total width of~\n{0pt}.
\end{example}

%\section{Overfull and underfull boxes}
%\label{over/underfull}
\section{Overfull and underfull boxes}
\label{over/underfull}

%If a box has a  size specification \TeX\ will
%stretch or shrink glue in the box. For glue with
%only finite stretch or shrink components the {\em badness\/}
%(see Chapter~\ref{line:break}) of stretching or shrinking
%is computed.
%In \TeX\ version~3 the badness
%\cstoidx badness\par
%\thecstoidxsub{TeX}{version 3}
%of the box most recently
%constructed is available for inspection
%by the user through the \cs{badness} parameter. Values for
%badness range 0--$10\,000$, but if the box is overfull
%it is~$1\,000\,000$.
If a box has a  size specification \TeX\ will
stretch or shrink glue in the box. For glue with
only finite stretch or shrink components the {\em badness\/}
(see Chapter~\ref{line:break}) of stretching or shrinking
is computed.
In \TeX\ version~3 the badness
\cstoidx badness\par
\thecstoidxsub{TeX}{version 3}
of the box most recently
constructed is available for inspection
by the user through the \cs{badness} parameter. Values for
badness range 0--$10\,000$, but if the box is overfull
it is~$1\,000\,000$.

%When \TeX\ considers the badness too large,
%it gives a diagnostic message. Let us first consider error reporting
%for horizontal boxes.
When \TeX\ considers the badness too large,
it gives a diagnostic message. Let us first consider error reporting
for horizontal boxes.

%Horizontal boxes of which the glue has to stretch are never reported if
%\cstoidx hbadness\par\cstoidx vbadness\par
%\cs{hbadness}${}\geq10\,000$; otherwise \TeX\ reports them
%as `underfull' if their badness is more than \cs{hbadness}.
Horizontal boxes of which the glue has to stretch are never reported if
\cstoidx hbadness\par\cstoidx vbadness\par
\cs{hbadness}${}\geq10\,000$; otherwise \TeX\ reports them
as `underfull' if their badness is more than \cs{hbadness}.

%Glue shrinking can lead to `overfull' boxes: a box is called
%\cstoidx hfuzz\par\cstoidx vfuzz\par
%overfull if the available shrink is less than the shrink
%necessary to meet the box specification. An overfull box
%is only reported if the difference in shrink is more than
%\cs{hfuzz}, or if \cs{hbadness}${}<100$ (and it turns out that
%using all available shrinkability has badness~$100$).
Glue shrinking can lead to `overfull' boxes: a box is called
\cstoidx hfuzz\par\cstoidx vfuzz\par
overfull if the available shrink is less than the shrink
necessary to meet the box specification. An overfull box
is only reported if the difference in shrink is more than
\cs{hfuzz}, or if \cs{hbadness}${}<100$ (and it turns out that
using all available shrinkability has badness~$100$).

%\begin{example} Setting \verb>\hfuzz=1pt> will let \TeX\ ignore
%boxes that can not shrink enough if they lack less than~\n{1pt}.
%In
%\begin{verbatim}
%\hbox to 1pt{\hskip3pt minus .5pt}
%\end{verbatim}
%\begin{verbatim}
%\hbox to 1pt{\hskip3pt minus 1.5pt}
%\end{verbatim}
%only the first box will give an error message:
%it is \n{1.5pt} too big, whereas the second lacks
%\n{.5pt} which is less than \cs{hfuzz}.
%\end{example}
\begin{example} Setting \verb>\hfuzz=1pt> will let \TeX\ ignore
boxes that can not shrink enough if they lack less than~\n{1pt}.
In
\begin{verbatim}
\hbox to 1pt{\hskip3pt minus .5pt}
\end{verbatim}
\begin{verbatim}
\hbox to 1pt{\hskip3pt minus 1.5pt}
\end{verbatim}
only the first box will give an error message:
it is \n{1.5pt} too big, whereas the second lacks
\n{.5pt} which is less than \cs{hfuzz}.
\end{example}

%Also, boxes that shrink but that are not overfull can be reported:
%if a box is `tight', that is, if it uses at least half its
%shrinkability, \TeX\ reports this fact if the
%computed badness (which is between 13 and~100) is more than
%\cs{hbadness}.
Also, boxes that shrink but that are not overfull can be reported:
if a box is `tight', that is, if it uses at least half its
shrinkability, \TeX\ reports this fact if the
computed badness (which is between 13 and~100) is more than
\cs{hbadness}.

%For horizontal and vertical boxes this error reporting is almost
%\cstoidx overfullrule\par
%the same, with parameters \cs{vbadness} and \cs{vfuzz}.
%The difference is that for horizontal overfull boxes
%\TeX\ will draw a rule to the right of the box that has the
%same height as the box, and width \cs{overfullrule}.
%No overfull rule ensues if
%the \cs{tabskip} glue in an \cs{halign} cannot be
%shrunk enough.
For horizontal and vertical boxes this error reporting is almost
\cstoidx overfullrule\par
the same, with parameters \cs{vbadness} and \cs{vfuzz}.
The difference is that for horizontal overfull boxes
\TeX\ will draw a rule to the right of the box that has the
same height as the box, and width \cs{overfullrule}.
No overfull rule ensues if
the \cs{tabskip} glue in an \cs{halign} cannot be
shrunk enough.

%\section{Opening and closing boxes}
\section{Opening and closing boxes}

%The opening and closing braces of a box can be either explicit, that
%is, character tokens of category 1\index{category!1}
%and~2\index{category!2}, or implicit, a control sequence \verb=\let=
%to such a character.  After the opening brace the \csidx{everyhbox} or
%\csidx{everyvbox} tokens are inserted.  If this box appeared in a
%\csidx{setbox} assignment any \csidx{afterassignment} token is
%inserted even before the `everybox' tokens.
The opening and closing braces of a box can be either explicit, that
is, character tokens of category 1\index{category!1}
and~2\index{category!2}, or implicit, a control sequence \verb=\let=
to such a character.  After the opening brace the \csidx{everyhbox} or
\csidx{everyvbox} tokens are inserted.  If this box appeared in a
\csidx{setbox} assignment any \csidx{afterassignment} token is
inserted even before the `everybox' tokens.

%\begin{example} \label{every:box:assign}
%\begin{verbatim}
%\everyhbox{b}
%\afterassignment a
%\setbox0=\hbox{c}
%\showbox0
%\end{verbatim}
%gives
%\begin{verbatim}
%> \box0=
%\hbox(6.94444+0.0)x15.27782
%.\tenrm a
%.\tenrm b
%.\kern0.27779
%.\tenrm c
%\end{verbatim}
%\end{example}
\begin{example} \label{every:box:assign}
\begin{verbatim}
\everyhbox{b}
\afterassignment a
\setbox0=\hbox{c}
\showbox0
\end{verbatim}
gives
\begin{verbatim}
> \box0=
\hbox(6.94444+0.0)x15.27782
.\tenrm a
.\tenrm b
.\kern0.27779
.\tenrm c
\end{verbatim}
\end{example}

%Implicit braces can be used to let a box be opened or closed
%by a macro, for example:
%\begin{verbatim}
%\def\openbox#1{\setbox#1=\hbox\bgroup}
%\def\closebox#1{\egroup\DoSomethingWithBox#1}
%\openbox0 ... \closebox0
%\end{verbatim}
%This mechanism can be used to scoop up paragraphs:
%\begin{verbatim}
%\everypar{\setbox\parbox=
%    \vbox\bgroup
%         \everypar{}
%         \def\par{\egroup\UseBox\parbox}}
%\end{verbatim}
%Here the \cs{everypar} opens the box and lets the text be
%set in the box: starting for instance
%\begin{verbatim}
%Begin a text ...
%\end{verbatim}
%gives the equivalent of
%\begin{verbatim}
%\setbox\parbox=\vbox{Begin a text ...
%\end{verbatim}
%Inside the box \cs{par} has been redefined, so
%\begin{verbatim}
%... a text ends.\par
%\end{verbatim}
%is equivalent to
%\begin{verbatim}
%... a text ends.}\Usebox\parbox
%\end{verbatim}
Implicit braces can be used to let a box be opened or closed
by a macro, for example:
\begin{verbatim}
\def\openbox#1{\setbox#1=\hbox\bgroup}
\def\closebox#1{\egroup\DoSomethingWithBox#1}
\openbox0 ... \closebox0
\end{verbatim}
This mechanism can be used to scoop up paragraphs:
\begin{verbatim}
\everypar{\setbox\parbox=
    \vbox\bgroup
         \everypar{}
         \def\par{\egroup\UseBox\parbox}}
\end{verbatim}
Here the \cs{everypar} opens the box and lets the text be
set in the box: starting for instance
\begin{verbatim}
Begin a text ...
\end{verbatim}
gives the equivalent of
\begin{verbatim}
\setbox\parbox=\vbox{Begin a text ...
\end{verbatim}
Inside the box \cs{par} has been redefined, so
\begin{verbatim}
... a text ends.\par
\end{verbatim}
is equivalent to
\begin{verbatim}
... a text ends.}\Usebox\parbox
\end{verbatim}

%In this example, the \cs{UseBox} command can only treat the
%box as a whole; if the elements of the box should somehow
%be treated separately another approach is necessary.
%In
%\begin{verbatim}
%\everypar{\setbox\parbox=
%  \vbox\bgroup\everypar{}%
%       \def\par{\endgraf\HandleLines
%                \egroup\box\parbox}}
%\def\HandleLines{ ... \lastbox ... }
%\end{verbatim}
%the macro \cs{HandleLines} can have access to successive
%elements from the vertical list of the paragraph.
%See also the example on page~\pageref{varioset}.
In this example, the \cs{UseBox} command can only treat the
box as a whole; if the elements of the box should somehow
be treated separately another approach is necessary.
In
\begin{verbatim}
\everypar{\setbox\parbox=
  \vbox\bgroup\everypar{}%
       \def\par{\endgraf\HandleLines
                \egroup\box\parbox}}
\def\HandleLines{ ... \lastbox ... }
\end{verbatim}
the macro \cs{HandleLines} can have access to successive
elements from the vertical list of the paragraph.
See also the example on page~\pageref{varioset}.

%%\point Unboxing
%\section{Unboxing}
%\point Unboxing
\section{Unboxing}

%Boxes can be unwrapped by the commands \csidx{unhbox} and
%\csidx{unvbox}, and by their copying versions
%\csidx{unhcopy} and \csidx{unvcopy}.
%These are horizontal and vertical commands
%(see Chapter~\ref{hvmode}), considering that in effect
%they contribute a partial horizontal or vertical list.
%It is not possible to \cs{unhbox} a register
%containing a \cs{vbox} or vice versa,
%but a void box register can both be \cs{unhbox}ed and
%\cs{unvbox}ed.
Boxes can be unwrapped by the commands \csidx{unhbox} and
\csidx{unvbox}, and by their copying versions
\csidx{unhcopy} and \csidx{unvcopy}.
These are horizontal and vertical commands
(see Chapter~\ref{hvmode}), considering that in effect
they contribute a partial horizontal or vertical list.
It is not possible to \cs{unhbox} a register
containing a \cs{vbox} or vice versa,
but a void box register can both be \cs{unhbox}ed and
\cs{unvbox}ed.

%Unboxing takes the contents of a box in a box register and appends
%them to the surrounding list; any glue can then
%be set anew. Thus
%\begin{verbatim}
%\setbox0=\hbox to 1cm{\hfil} \hbox to 2cm{\unhbox0}
%\end{verbatim}
%is completely equivalent to
%\begin{verbatim}
%\hbox to 2cm{\hfil}
%\end{verbatim}
%and not to
%\begin{verbatim}
%\hbox to 2cm{\kern1cm}
%\end{verbatim}
Unboxing takes the contents of a box in a box register and appends
them to the surrounding list; any glue can then
be set anew. Thus
\begin{verbatim}
\setbox0=\hbox to 1cm{\hfil} \hbox to 2cm{\unhbox0}
\end{verbatim}
is completely equivalent to
\begin{verbatim}
\hbox to 2cm{\hfil}
\end{verbatim}
and not to
\begin{verbatim}
\hbox to 2cm{\kern1cm}
\end{verbatim}

%The intrinsically horizontal nature of \cs{unhbox} is
%\cstoidx leavevmode\par
%used to define
%\begin{verbatim}
%\def\leavevmode{\unhbox\voidb@x}
%\end{verbatim}
%This command switches from vertical mode to horizontal without
%adding anything to the horizontal list.
%However, the subsequent \cs{indent} caused by this transition
%adds an indentation box.
%In horizontal mode the \cs{leavevmode} command has no effect.
%Note that here it is not necessary to use \cs{unhcopy},
%because the register is empty anyhow.
The intrinsically horizontal nature of \cs{unhbox} is
\cstoidx leavevmode\par
used to define
\begin{verbatim}
\def\leavevmode{\unhbox\voidb@x}
\end{verbatim}
This command switches from vertical mode to horizontal without
adding anything to the horizontal list.
However, the subsequent \cs{indent} caused by this transition
adds an indentation box.
In horizontal mode the \cs{leavevmode} command has no effect.
Note that here it is not necessary to use \cs{unhcopy},
because the register is empty anyhow.

%Beware of the following subtlety: unboxing in vertical
%mode does not add interline glue between the box contents and
%any preceding item.
%Also, the value of \cs{prevdepth} is not
%changed, so glue between the box contents and any following
%item will  occur only if there was something preceding the box;
%interline glue will be based on the depth of that preceding item.
%Similarly, unboxing in horizontal mode does not influence
%the \cs{spacefactor}.
Beware of the following subtlety: unboxing in vertical
mode does not add interline glue between the box contents and
any preceding item.
Also, the value of \cs{prevdepth} is not
changed, so glue between the box contents and any following
item will  occur only if there was something preceding the box;
interline glue will be based on the depth of that preceding item.
Similarly, unboxing in horizontal mode does not influence
the \cs{spacefactor}.

%%\point Text in boxes
%\section{Text in boxes}
%\point Text in boxes
\section{Text in boxes}

%Both horizontal and vertical boxes can contain text. However,
%\index{boxes !text in}
%the way text is treated differs.
%In horizontal boxes
%the text is placed in one straight line, and the width of
%the box is in principle the natural width of the text
%(and other items) contained in it. No \gram{vertical command}s
%are allowed inside a horizontal box, and \cs{par} does
%nothing in this case.
Both horizontal and vertical boxes can contain text. However,
\index{boxes !text in}
the way text is treated differs.
In horizontal boxes
the text is placed in one straight line, and the width of
the box is in principle the natural width of the text
(and other items) contained in it. No \gram{vertical command}s
are allowed inside a horizontal box, and \cs{par} does
nothing in this case.

%For vertical boxes the situation is radically different.
%As soon as a character, or any other \gram{horizontal command}
%(see page~\pageref{h:com:list}),
%is encountered in a vertical box, \TeX\ starts building a paragraph
%in unrestricted horizontal mode, that is, just as if the paragraph
%were directly part of the page.
%At the occurrence of a \gram{vertical command}
%(see page~\pageref{v:com:list}), or at the end
%of the box, the paragraph is broken into lines using the
%current values of parameters such as~\cs{hsize}.
For vertical boxes the situation is radically different.
As soon as a character, or any other \gram{horizontal command}
(see page~\pageref{h:com:list}),
is encountered in a vertical box, \TeX\ starts building a paragraph
in unrestricted horizontal mode, that is, just as if the paragraph
were directly part of the page.
At the occurrence of a \gram{vertical command}
(see page~\pageref{v:com:list}), or at the end
of the box, the paragraph is broken into lines using the
current values of parameters such as~\cs{hsize}.

%Thus
%\begin{verbatim}
%\hbox to 3cm{\vbox{some reasonably long text}}
%\end{verbatim}
%will {\sl not\/} give a paragraph of width 3 centimetres
%(it gives an overfull horizontal box if \cs{hsize}${}>{}$\n{3cm}).
%However,
%\begin{verbatim}
%\vbox{\hsize=3cm some reasonably long text}
%\end{verbatim}
%will be 3 centimetres wide.
Thus
\begin{verbatim}
\hbox to 3cm{\vbox{some reasonably long text}}
\end{verbatim}
will {\sl not\/} give a paragraph of width 3 centimetres
(it gives an overfull horizontal box if \cs{hsize}${}>{}$\n{3cm}).
However,
\begin{verbatim}
\vbox{\hsize=3cm some reasonably long text}
\end{verbatim}
will be 3 centimetres wide.

%A paragraph of text inside a vertical box is broken into
%lines, which are packed in horizontal boxes.
%These boxes  are then stacked
%in internal vertical mode, possibly with
%\cs{baselineskip} and \cs{lineskip} separating them
%(this is treated in Chapter~\ref{baseline}).
%This process is also used for text on the page; the boxes
%are then stacked in outer vertical mode.
A paragraph of text inside a vertical box is broken into
lines, which are packed in horizontal boxes.
These boxes  are then stacked
in internal vertical mode, possibly with
\cs{baselineskip} and \cs{lineskip} separating them
(this is treated in Chapter~\ref{baseline}).
This process is also used for text on the page; the boxes
are then stacked in outer vertical mode.

%If the internal vertical list is empty, no \cs{parskip}
%glue is added at the start of a paragraph.
If the internal vertical list is empty, no \cs{parskip}
glue is added at the start of a paragraph.

%Because text in a horizontal box is not
%\label{wide:vbox}%
%broken into lines, there is a further
%difference between text in restricted and unrestricted
%horizontal mode. In restricted horizontal mode no
%discretionary nodes and whatsit items changing the
%value of the current language are inserted.
%This may give problems if the text is subsequently
%unboxed to form part of a paragraph.
Because text in a horizontal box is not
\label{wide:vbox}%
broken into lines, there is a further
difference between text in restricted and unrestricted
horizontal mode. In restricted horizontal mode no
discretionary nodes and whatsit items changing the
value of the current language are inserted.
This may give problems if the text is subsequently
unboxed to form part of a paragraph.

%See Chapter~\ref{line:break} for an explanation of these
%items, and \cite{Downs} for a way around this problem.
See Chapter~\ref{line:break} for an explanation of these
items, and \cite{Downs} for a way around this problem.

%%\point Assorted remarks
%\section{Assorted remarks}
%\point Assorted remarks
\section{Assorted remarks}

%%\spoint Forgetting the \cs{box}
%\subsection{Forgetting the \cs{box}}
%\spoint Forgetting the \cs{box}
\subsection{Forgetting the \cs{box}}

%After \verb.\newcount\foo., one can use \cs{foo} on its own
%to get the \cs{foo} counter.
%For  boxes, however, one has to use \verb.\box\foo. to get
%the \cs{foo} box.
%The reason for this is that there exists
%no separate \cs{boxdef} command, so \cs{chardef} is
%used (see Chapter~\ref{alloc}).
After \verb.\newcount\foo., one can use \cs{foo} on its own
to get the \cs{foo} counter.
For  boxes, however, one has to use \verb.\box\foo. to get
the \cs{foo} box.
The reason for this is that there exists
no separate \cs{boxdef} command, so \cs{chardef} is
used (see Chapter~\ref{alloc}).

%\begin{example}
%Suppose \verb.\newbox\foo. allocates box register~25; then
%typing \cs{foo} is equivalent  to typing
%\verb.\char25..
%\end{example}
\begin{example}
Suppose \verb.\newbox\foo. allocates box register~25; then
typing \cs{foo} is equivalent  to typing
\verb.\char25..
\end{example}

%%\spoint Special-purpose boxes
%\subsection{Special-purpose boxes}
%\spoint Special-purpose boxes
\subsection{Special-purpose boxes}

%Some   box registers
%have a special
%purpose:
%\begin{itemize}
%\item \cs{box255} is by used \TeX\ internally
% to give the page to the output routine.
%\item \cs{voidb@x} is the number of
% a box register allocated in
% \n{plain.tex}; it is supposed to be empty always.
% It is used in the macro \cs{leavevmode} and others.
%\item when a new \cs{insert} is created with the plain \TeX\
% \cs{newinsert} macro, a \cs{count},
% \cs{dimen}, \cs{skip}, and \cs{box} all with the same number
% are reserved for that insert.
% The numbers for these registers count down from~254.
%\end{itemize}
Some   box registers
have a special
purpose:
\begin{itemize}
\item \cs{box255} is by used \TeX\ internally
 to give the page to the output routine.
\item \cs{voidb@x} is the number of
 a box register allocated in
 \n{plain.tex}; it is supposed to be empty always.
 It is used in the macro \cs{leavevmode} and others.
\item when a new \cs{insert} is created with the plain \TeX\
 \cs{newinsert} macro, a \cs{count},
 \cs{dimen}, \cs{skip}, and \cs{box} all with the same number
 are reserved for that insert.
 The numbers for these registers count down from~254.
\end{itemize}


%%\spoint The height of a vertical box in horizontal mode
%\subsection{The height of a vertical box in horizontal mode}
%\spoint The height of a vertical box in horizontal mode
\subsection{The height of a vertical box in horizontal mode}

%In horizontal mode a vertical box is placed with its
%reference point aligned vertically with the reference
%point of the surrounding box.
%\TeX\ then traverses its contents starting at the left
%upper corner; that is, the point that lies above the reference
%point by a distance of the height of the box.
%Changing the height of the box  implies then that the
%contents of the box are placed at a different height.
In horizontal mode a vertical box is placed with its
reference point aligned vertically with the reference
point of the surrounding box.
\TeX\ then traverses its contents starting at the left
upper corner; that is, the point that lies above the reference
point by a distance of the height of the box.
Changing the height of the box  implies then that the
contents of the box are placed at a different height.

%Consider as an example
%\begin{verbatim}
%\hbox{a\setbox0=\vbox{\hbox{b}}\box0 c}
%\end{verbatim}
%which gives
%\begin{disp}\leavevmode\hbox{a\setbox0=\vbox{\hbox{b}}\box0 c}\end{disp}
%and
%\begin{verbatim}
%\hbox{a\setbox0=\vbox{\hbox{b}}\ht0=0cm \box0 c}
%\end{verbatim}
%which gives
%\begin{disp}\leavevmode\hbox{a\setbox0=\vbox{\hbox{b}}\ht0=0cm \box0 c}\end{disp}
Consider as an example
\begin{verbatim}
\hbox{a\setbox0=\vbox{\hbox{b}}\box0 c}
\end{verbatim}
which gives
\begin{disp}\leavevmode\hbox{a\setbox0=\vbox{\hbox{b}}\box0 c}\end{disp}
and
\begin{verbatim}
\hbox{a\setbox0=\vbox{\hbox{b}}\ht0=0cm \box0 c}
\end{verbatim}
which gives
\begin{disp}\leavevmode\hbox{a\setbox0=\vbox{\hbox{b}}\ht0=0cm \box0 c}\end{disp}

%By contrast, changing the width of a box placed in vertical
%mode has no effect on its placement.
By contrast, changing the width of a box placed in vertical
mode has no effect on its placement.

%%\spoint More subtleties with vertical boxes
%\subsection{More subtleties with vertical boxes}
%\spoint More subtleties with vertical boxes
\subsection{More subtleties with vertical boxes}

%Since there are two kinds of vertical boxes, the \cs{vbox} and
%the \cs{vtop}, using these two kinds nested may lead to
%confusing results. For instance,
%\begin{verbatim}
%\vtop{\vbox{...}}
%\end{verbatim}
%is completely equivalent to just
%\begin{verbatim}
%\vbox{...}
%\end{verbatim}
Since there are two kinds of vertical boxes, the \cs{vbox} and
the \cs{vtop}, using these two kinds nested may lead to
confusing results. For instance,
\begin{verbatim}
\vtop{\vbox{...}}
\end{verbatim}
is completely equivalent to just
\begin{verbatim}
\vbox{...}
\end{verbatim}

%It was stated above that
%the depth of a \cs{vbox} is zero if the last item
%is a kern or  glue, and the height of a \cs{vtop} is
%zero unless the first item in it is a box.
%The above examples used a kern for that first or last item,
%but if, in the case of a \cs{vtop},
%this item is not a glue or kern, one is apt to
%overlook the effect that it has on the surrounding box.
%For instance,
%\begin{verbatim}
%\vtop{\write16{...}...}
%\end{verbatim}
%has zero height,
%because the write instruction
%is packed into a `whatsit' item that is placed on the current,
%that is, the vertical, list.
%The remedy here is
%\begin{verbatim}
%\vtop{\leavevmode\write16{...}...}
%\end{verbatim}
%which puts the whatsit in the beginning of the paragraph,
%instead of above it.
It was stated above that
the depth of a \cs{vbox} is zero if the last item
is a kern or  glue, and the height of a \cs{vtop} is
zero unless the first item in it is a box.
The above examples used a kern for that first or last item,
but if, in the case of a \cs{vtop},
this item is not a glue or kern, one is apt to
overlook the effect that it has on the surrounding box.
For instance,
\begin{verbatim}
\vtop{\write16{...}...}
\end{verbatim}
has zero height,
because the write instruction
is packed into a `whatsit' item that is placed on the current,
that is, the vertical, list.
The remedy here is
\begin{verbatim}
\vtop{\leavevmode\write16{...}...}
\end{verbatim}
which puts the whatsit in the beginning of the paragraph,
instead of above it.

%Placement of items in a vertical list is sometimes
%a bit tricky. There is for instance a difference between
%how vertical and horizontal boxes are treated in a
%vertical list. Consider the following examples.
%After \cs{offinterlineskip} the first example\begin{verbatim}
%\vbox{\hbox{a}
%      \setbox0=\vbox{\hbox{(}}
%      \ht0=0pt \dp0=0pt \box0
%      \hbox{ b}}
%\end{verbatim}
%gives \begin{disp}\offinterlineskip\leavevmode\vbox{\hbox{a}
%      \setbox0=\vbox{\hbox{(}}
%      \ht0=0pt \dp0=0pt \box0
%      \hbox{ b}}
%\end{disp}
%while a slight variant\begin{verbatim}
%\vbox{\hbox{a}
%      \setbox0=\hbox{(}
%      \ht0=0pt \dp0=0pt \box0
%      \hbox{ b}}
%\end{verbatim}
%gives
%\begin{disp}\offinterlineskip\leavevmode\vbox{\hbox{a}
%      \setbox0=\hbox{(}
%      \ht0=0pt \dp0=0pt
%      \box0
%      \hbox{ b}}
%\end{disp}
%The difference is caused by the fact that horizontal boxes
%are placed with respect to their reference point, but vertical
%boxes with respect to their upper left corner.
Placement of items in a vertical list is sometimes
a bit tricky. There is for instance a difference between
how vertical and horizontal boxes are treated in a
vertical list. Consider the following examples.
After \cs{offinterlineskip} the first example\begin{verbatim}
\vbox{\hbox{a}
      \setbox0=\vbox{\hbox{(}}
      \ht0=0pt \dp0=0pt \box0
      \hbox{ b}}
\end{verbatim}
gives \begin{disp}\offinterlineskip\leavevmode\vbox{\hbox{a}
      \setbox0=\vbox{\hbox{(}}
      \ht0=0pt \dp0=0pt \box0
      \hbox{ b}}
\end{disp}
while a slight variant\begin{verbatim}
\vbox{\hbox{a}
      \setbox0=\hbox{(}
      \ht0=0pt \dp0=0pt \box0
      \hbox{ b}}
\end{verbatim}
gives
\begin{disp}\offinterlineskip\leavevmode\vbox{\hbox{a}
      \setbox0=\hbox{(}
      \ht0=0pt \dp0=0pt
      \box0
      \hbox{ b}}
\end{disp}
The difference is caused by the fact that horizontal boxes
are placed with respect to their reference point, but vertical
boxes with respect to their upper left corner.

%%\spoint Hanging the \cs{lastbox} back in the list
%\subsection{Hanging the \cs{lastbox} back in the list}
%\spoint Hanging the \cs{lastbox} back in the list
\subsection{Hanging the \cs{lastbox} back in the list}

%You can pick the last box off a vertical list that has been
%compiled in (internal) vertical mode.
%However, if you try to hang it back in the list the vertical
%spacing may go haywire. If you just hang it back,
%\begin{verbatim}
%\setbox\tmpbox=\lastbox
%\usethetmpbox \box\tmpbox
%\end{verbatim}
%baselineskip glue is added a second time. If you `unskip' prior
%to hanging the box back,
%\begin{verbatim}
%\setbox\tmpbox=\lastbox \unskip
%\usethetmpbox \box\tmpbox
%\end{verbatim}
%things go wrong in a more subtle way.
%The \gram{internal dimen} \cs{prevdepth}
%(which controls interline glue; see Chapter~\ref{baseline})
%will have a
%value based on the last box, but what you need for the proper
%interline glue is a depth based on one box earlier.
%The solution is not to unskip,
%but to specify \cs{nointerlineskip}:
%\begin{verbatim}
%\setbox\tmpbox=\lastbox
%\usethetmpbox \nointerlineskip \box\tmpbox
%\end{verbatim}
You can pick the last box off a vertical list that has been
compiled in (internal) vertical mode.
However, if you try to hang it back in the list the vertical
spacing may go haywire. If you just hang it back,
\begin{verbatim}
\setbox\tmpbox=\lastbox
\usethetmpbox \box\tmpbox
\end{verbatim}
baselineskip glue is added a second time. If you `unskip' prior
to hanging the box back,
\begin{verbatim}
\setbox\tmpbox=\lastbox \unskip
\usethetmpbox \box\tmpbox
\end{verbatim}
things go wrong in a more subtle way.
The \gram{internal dimen} \cs{prevdepth}
(which controls interline glue; see Chapter~\ref{baseline})
will have a
value based on the last box, but what you need for the proper
interline glue is a depth based on one box earlier.
The solution is not to unskip,
but to specify \cs{nointerlineskip}:
\begin{verbatim}
\setbox\tmpbox=\lastbox
\usethetmpbox \nointerlineskip \box\tmpbox
\end{verbatim}


%%\spoint[varioset] Dissecting paragraphs with \cs{lastbox}
%\subsection{Dissecting paragraphs with \cs{lastbox}}
%\label{varioset}
%\spoint[varioset] Dissecting paragraphs with \cs{lastbox}
\subsection{Dissecting paragraphs with \cs{lastbox}}
\label{varioset}

%Repeatedly applying \cs{last...} and \cs{un...} macros
%\howto Take a paragraph apart\par
%can be used to take a paragraph apart.
%Here is an example of that.
Repeatedly applying \cs{last...} and \cs{un...} macros
\howto Take a paragraph apart\par
can be used to take a paragraph apart.
Here is an example of that.

%\indent\vbox{\message{Check vario look!}
%\hyphenpenalty10000 \exhyphenpenalty10000 \parindent 10pt\relax
%\advance\hsize by -2\parindent \rightskip 0pt\relax
%\newif\ifsnap \spaceskip=\fontdimen2\font plus 2\fontdimen2\font
%\def\eatlines{
%    \setbox2\lastbox    % check the last line
%    \ifvoid2\global\snaptrue
%    \else                      % if it's not empty
%    \unskip\unpenalty          % take whatever is
%    {\eatlines}                % above it;
%    \setbox4\hbox{\unhcopy2}   % collapse this line
%\message{^^Jsnap \the\wd4 vs \the\wd2^^J}
%%\hbox{\copy4 .4} \hbox{\copy 2 .2}
%    \ifdim\wd4<.97\wd2 % if the difference is too large, maybe snap
%        \ifsnap \message{^^JNot snapping just now ^^J}
%          \hbox to \hsize{\unhbox2} \global\snapfalse % we just snapped: not again
%        \else   \message{^^JSnapping!^^J}
%          \box4 \global\snaptrue    % safe to snap; remember snap
%        \fi
%    \else \message{^^Jfull length, no snap^^J}
%      \box2 \global\snapfalse
%    \fi
%    \fi}
%\advance\hsize by -.8in \advance \leftskip by .5 in\relax
%\noindent In typesetting advertisement copy, a way of justifying
%paragraphs has become popular in recent years
%that is somewhere between flushright and raggedright
%setting.
%Lines that would stretch beyond certain limits
%are set with their glue at natural width. This single paragraph
%is but an example of this procedure; the macros are given
% next.\par\eatlines}\par
\indent\vbox{\message{Check vario look!}
\hyphenpenalty10000 \exhyphenpenalty10000 \parindent 10pt\relax
\advance\hsize by -2\parindent \rightskip 0pt\relax
\newif\ifsnap \spaceskip=\fontdimen2\font plus 2\fontdimen2\font
\def\eatlines{
    \setbox2\lastbox    % check the last line
    \ifvoid2\global\snaptrue
    \else                      % if it's not empty
    \unskip\unpenalty          % take whatever is
    {\eatlines}                % above it;
    \setbox4\hbox{\unhcopy2}   % collapse this line
\message{^^Jsnap \the\wd4 vs \the\wd2^^J}
%\hbox{\copy4 .4} \hbox{\copy 2 .2}
    \ifdim\wd4<.97\wd2 % if the difference is too large, maybe snap
        \ifsnap \message{^^JNot snapping just now ^^J}
          \hbox to \hsize{\unhbox2} \global\snapfalse % we just snapped: not again
        \else   \message{^^JSnapping!^^J}
          \box4 \global\snaptrue    % safe to snap; remember snap
        \fi
    \else \message{^^Jfull length, no snap^^J}
      \box2 \global\snapfalse
    \fi
    \fi}
\advance\hsize by -.8in \advance \leftskip by .5 in\relax
\noindent In typesetting advertisement copy, a way of justifying
paragraphs has become popular in recent years
that is somewhere between flushright and raggedright
setting.
Lines that would stretch beyond certain limits
are set with their glue at natural width. This single paragraph
is but an example of this procedure; the macros are given
 next.\par\eatlines}\par

%\begin{verbatim}
%\newbox\linebox \newbox\snapbox
%\def\eatlines{
%    \setbox\linebox\lastbox    % check the last line
%    \ifvoid\linebox
%    \else                      % if it's not empty
%    \unskip\unpenalty          % take whatever is
%    {\eatlines}                % above it;
%                               % collapse the line
%    \setbox\snapbox\hbox{\unhcopy\linebox}
%                       % depending on the difference
%    \ifdim\wd\snapbox<.98\wd\linebox
%          \box\snapbox % take the one or the other,
%    \else \box\linebox \fi
%    \fi}
%\end{verbatim}
%This macro can be called as
%\begin{verbatim}
%\vbox{ ... some text ... \par\eatlines}
%\end{verbatim}
%or it can be inserted automatically
%with \cs{everypar}; see~\cite{E1}.
\begin{verbatim}
\newbox\linebox \newbox\snapbox
\def\eatlines{
    \setbox\linebox\lastbox    % check the last line
    \ifvoid\linebox
    \else                      % if it's not empty
    \unskip\unpenalty          % take whatever is
    {\eatlines}                % above it;
                               % collapse the line
    \setbox\snapbox\hbox{\unhcopy\linebox}
                       % depending on the difference
    \ifdim\wd\snapbox<.98\wd\linebox
          \box\snapbox % take the one or the other,
    \else \box\linebox \fi
    \fi}
\end{verbatim}
This macro can be called as
\begin{verbatim}
\vbox{ ... some text ... \par\eatlines}
\end{verbatim}
or it can be inserted automatically
with \cs{everypar}; see~\cite{E1}.

%In the macro \cs{eatlines}, the \cs{lastbox} is taken
%from a vertical list. If the list is empty
%the last box will test true on \cs{ifvoid}.
%These boxes containing lines from a paragraph
%are actually horizontal boxes: the test
%\cs{ifhbox} applied to them would give a true
%result.
In the macro \cs{eatlines}, the \cs{lastbox} is taken
from a vertical list. If the list is empty
the last box will test true on \cs{ifvoid}.
These boxes containing lines from a paragraph
are actually horizontal boxes: the test
\cs{ifhbox} applied to them would give a true
result.

%\index{boxes|)}
%\endofchapter
%%%%% end of input file [boxes]
\index{boxes|)}
\endofchapter
%%%% end of input file [boxes]

\end{document}
