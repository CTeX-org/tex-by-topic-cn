% -*- coding: utf-8 -*-
\documentclass{book}


%\nofiles

\usepackage{comment,makeidx,multicol}
\usepackage{svn-multi}
\svnid{$Id: TeXbyTopic.tex 3 2008-02-02 11:19:13Z victor $}
\usepackage{newcent}
\renewcommand{\ttdefault}{cmtt}
\addtolength{\oddsidemargin}{-.5in}
\addtolength{\evensidemargin}{-.5in}
\addtolength{\textwidth}{1.2in}
\addtolength{\textheight}{15pt}

\usepackage[colorlinks,plainpages=true,pagebackref=true]{hyperref}

\begin{comment}
\usepackage{german}
% german
\righthyphenmin=3
\mdqoff
\captionsenglish
\end{comment}
\usepackage[english]{babel} 
{\catcode`"=13 \gdef"#1{\ifx#1"\discretionary{}{}{}\fi\relax}}
\def\mdqon{\catcode`"=13\relax}
\def\mdqoff{\catcode`"=12\relax}
\makeindex
\hyphenation{ex-em-pli-fies}

\usepackage{fancyhdr}
% headers & footers
\pagestyle{fancy}
% foot
\lfoot[\thepage]{\protect\small\protect\it Victor Eijkhout -- \protect\TeX\ by Topic}
\rfoot[{\protect\small\protect\it Victor Eijkhout -- \protect\TeX\ by Topic}]{\thepage}
\cfoot{}
% head is actually defined later.
\chead{}
\def\normalheads{
\lhead[\let\\\relax \let\uppercase\relax \leftmark]{\relax}
\rhead[\relax]{\let\\\relax \let\uppercase\relax \rightmark}
}
\lhead{}\rhead{}

\newdimen\tempdima \newdimen\tempdimb

% these are fine
\def\handbreak{\\ \message{^^JManual break!!!!^^J}}
\def\nl{\protect\\}\def\n#1{{\tt #1}}\def\cs#1{{\tt\char`\\#1}}\let\csc\cs
\def\lb{{\tt\char`\{}}\def\rb{{\tt\char`\}}}
\def\gr#1{$\langle$#1$\rangle$}\def\key#1{{\tt#1}}
\def\alt{}\def\altt{}%this way in manstijl
\def\ldash{\unskip\ --\nobreak\ \ignorespaces}
\def\rdash{\unskip\nobreak\ --\ \ignorespaces}
% check these
\def\hex{{\tt"}}
\def\ascii{{\sc ascii}}
\def\ebcdic{{\sc ebcdic}}
\def\IniTeX{Ini\TeX}\def\LamsTeX{LAMS\TeX}\def\VirTeX{Vir\TeX}
\def\AmsTeX{Ams\TeX}
\def\TeXbook{the \TeX\ book}\def\web{{\sc web}}
% needs major thinking
\newenvironment{disp}{\begin{quotation}}{\end{quotation}}
\newenvironment{Disp}{\begin{quotation}}{\end{quotation}}
\newenvironment{tdisp}{\begin{quotation}}{\end{quotation}}
\newenvironment{example}{\begin{quotation}}{\end{quotation}}
\newenvironment{inventory}{\begin{description}\raggedright}{\end{description}}
\newenvironment{glossinventory}{\begin{description}}{\end{description}}
\def\gram#1{\gr{#1}}%???
%
% index
%
\def\indexterm#1{\emph{#1}\index{#1}}
\def\indextermsub#1#2{\emph{#1 #2}\index{#1!#2}}
\def\indextermbus#1#2{\emph{#1 #2}\index{#2!#1}}
\def\term#1\par{\index{#1}}
\def\howto#1\par{}
\def\cstoidx#1\par{\index{#1@\cs{#1}@}}
\def\thecstoidx#1\par{\index{#1@\protect\csname #1\endcsname}}
\def\thecstoidxsub#1#2{\index{#1, #2@\protect\csname #1\endcsname, #2}\ignorespaces}
\def\csterm#1\par{\cstoidx #1\par\cs{#1}}
\def\csidx#1{\cstoidx #1\par\cs{#1}}

\def\tmc{\tracingmacros=2 \tracingcommands\tracingmacros}

%%%%%%%%%%%%%%%%%%%
\makeatletter
\def\snugbox{\hbox\bgroup\setbox\z@\vbox\bgroup
    \leftskip\z@
    \bgroup\aftergroup\make@snug
    \let\next=}
\def\make@snug{\par\sn@gify\egroup \box\z@\egroup}
\def\sn@gify
   {\skip\z@=\lastskip \unskip
    \advance\skip\z@\lastskip \unskip
    \unpenalty
    \setbox\z@\lastbox
    \ifvoid\z@ \nointerlineskip \else {\sn@gify} \fi
    \hbox{\unhbox\z@}\nointerlineskip
    \vskip\skip\z@
    }

\newdimen\fbh \fbh=60pt % dimension for easy scaling:
\newdimen\fbw \fbw=60pt % height and width of character box

\newdimen\dh \newdimen\dw % height and width of current character box
\newdimen\lh % height of previous character box
\newdimen\lw \lw=.4pt % line weight, instead of default .4pt

\def\hdotfill{\noindent
    \leaders\hbox{\vrule width 1pt height\lw 
                  \kern4pt 
                  \vrule width.5pt height\lw}\hfill\hbox{}
    \par}
\def\hlinefill{\noindent
    \leaders\hbox{\vrule width 5.5pt height\lw         }\hfill\hbox{}
    \par}
\def\stippel{$\qquad\qquad\qquad\qquad$}
\makeatother
%%%%%%%%%%%%%%%%%%%

\begin{comment}
\def\SansSerif{\Typeface:macHelvetica }
\def\SerifFont{\Typeface:macTimes }
\def\SansSerif{\Typeface:bsGillSans }
\def\SerifFont{\Typeface:bsBaskerville }
\end{comment}
\let\SansSerif\relax \def\italic{\it}
\let\SerifFont\relax \def\MainFont{\rm}
\let\SansSerif\relax
\let\SerifFont\relax
\let\PopIndentLevel\relax \let\PushIndentLevel\relax
\let\ToVerso\relax \let\ToRecto\relax

\begin{comment}
\def\stop@command@suffix{stop}
\let\PopListLevel\PopIndentLevel
\let\FlushRight\relax
\let\flushright\FlushRight
\let\SetListIndent\LevelIndent
\def\awp{\ifhmode\vadjust{\penalty-10000 }\else
    \penalty-10000 \fi}
\end{comment}
\let\awp\relax
\let\PopIndentLevel\relax \let\PopListLevel\relax

\showboxdepth=-1

%\input figs
\def\endofchapter{\vfill\noindent}

\setcounter{chapter}{25}

\begin{document}

%\chapter{Page Shape}\label{page:shape}
\chapter{页面形状}\label{page:shape}

%This chapter treats some of the parameters that 
%determine the size of the page and how it appears on paper.
本章讨论一些决定页面大小及其在纸面上显示的参数。

%\label{cschap:topskip}\label{cschap:hoffset}\label{cschap:voffset}\label{cschap:vsize2}\label{cschap:maxdepth}\label{cschap:splitmaxdepth2}
%\begin{inventory}
%\item [\cs{topskip}] 
%      Minimum distance between the top of the page box
%      and the baseline of the first box on the page. 
%      Plain \TeX\ default:~\n{10pt}
\label{cschap:topskip}\label{cschap:hoffset}\label{cschap:voffset}\label{cschap:vsize2}\label{cschap:maxdepth}\label{cschap:splitmaxdepth2}
\begin{inventory}
\item [\cs{topskip}] 
      页面盒子的顶部到本页第一个盒子基线的最小距离。
      Plain \TeX\ 默认值:~\n{10pt}

%\item [\cs{hoffset \cs{voffset}}]
%\mdqon
%      Distance by which the page is shifted right/""down 
%\mdqoff
%      with respect to the reference point.
\item [\cs{hoffset \cs{voffset}}]
\mdqon
      页面相对参考点向左/""向下移动的距离。
\mdqoff

%\item [\cs{vsize}] 
%      Height of the page box.
%      Plain \TeX\ default:~\n{8.9in}
\item [\cs{vsize}] 
      页面盒子的高度。
      Plain \TeX\ 默认值:~\n{8.9in}

%\item [\cs{maxdepth}] 
%      Maximum depth of the page box.
%      Plain \TeX\ default:~\n{4pt}
\item [\cs{maxdepth}] 
      页面盒子的最大深度。
      Plain \TeX\ 默认值:~\n{4pt}

%\item [\cs{splitmaxdepth}] 
%      Maximum depth of a box split off by a \cs{vsplit} operation. 
%      Plain \TeX\ default: \cs{maxdimen}
\item [\cs{splitmaxdepth}] 
      一个盒子能被一个 \cs{vsplit} 操作分割的最大深度。
      Plain \TeX\ 默认值: \cs{maxdimen}

%\end{inventory}
\end{inventory}

%%\point The reference point for global positioning
%\section{The reference point for global positioning}
%\point The reference point for global positioning
\section{全局定位基准点}

%The \indexterm{page positioning} on the paper is governed by
%a \TeX\ convention, to which output device drivers
%must adhere, that the top left point of the page is
%one inch from the page edges. Unfortunately this
%may lead to lots of trouble, for instance if a printer
%(or the page description language it uses)
%takes, say, the {\em lower\/} left corner as the
%reference point, and is factory set to US paper sizes,
%but is used with European standard A4 paper.
纸张上的页面定位(\indexterm{page positioning})由 \TeX\ 的惯例决定:即页面的左上角距离页边1英寸,这也是输出设备必须如实反映的。
不幸的是,这可能会导致许多麻烦,考虑以下情形:一台打印机,或打印机使用的页面描述语言把\emph{左下角}当作基准点,并且打印机出厂默认设为美国纸张大小,但打印机用了欧洲的标准A4纸。

%The page is shifted on the paper if one assigns non-zero
%values to \csidx{hoffset} or \csidx{voffset}: positive values
%shift to the right and down respectively.
如果指定 \csidx{hoffset} 或 \csidx{voffset} 为一个非零的值,页面会在纸面上移动:正值代表向右下移动,水平和竖直方向的移动是独立的。

%%\point \cs{topskip}
%\section{\protect\cs{topskip}}
%\point \cs{topskip}
\section{\protect\cs{topskip}}

%The \csidx{topskip} ensures to a certain point
%that the first baseline of a page
%will be at the same location from page to page,
%even if font sizes
%are switched between pages or if the first line has
%no ascenders.
\csidx{topskip} 确保了一点:每一页的第一条基线都处于同一位置,即使页面之间的字号不相同或者第一行不含有升部\footnote{译注(woclass):升部指的是某字母比字母x还要高的部分}的字母。

%Before the first box on each page some glue is inserted.
%This glue has the same stretch and shrink as \cs{topskip}, but
%the natural size is the natural size of \cs{topskip}
%minus the height of the first box, or zero if this
%would be negative.
每页第一个盒子之前都嵌入了铅空。
铅空可以和 \cs{topskip} 一起伸缩,通常它的大小是 \cs{topskip} 减去第一个盒子的高度,当结果为负数时,其值为零。

%Plain \TeX\ sets \cs{topskip} to {\tt 10pt}.
%Thus the top lines of pages will have their baselines
%at the same place if
%the top portion of the characters is ten point or less.
%For the Computer Modern fonts this condition is satisfied
%if the font size is less than (about) 13~points; 
%for larger fonts
%the baseline of the top line will drop.
Plain \TeX\ 将 \cs{topskip} 设为 {\tt 10pt}。因此,如果最上面一行的字母的顶部小于等于10点,每一页顶行的基线就处于相同的位置。
对于现代计算机字体来说,字号小于大约13点就能满足这个条件;对于大号字体来说,顶行的基线会下沉。

%The height of the page box for a page containing only
%text (and assuming a zero \cs{parskip})
%will be the \cs{topskip} plus a number of times
%the \cs{baselineskip}. Thus one can define a macro
%to compute the \cs{vsize} from the number of lines
%on a page:
%\howto Specify page height in lines\par
%\begin{verbatim}
%\def\HeightInLines#1{\count@=#1\relax
%    \advance\count@ by -1\relax
%    \vsize=\baselineskip
%    \multiply\vsize by \count@
%    \advance\vsize by \topskip}
%\end{verbatim}
%Calculating the \cs{vsize} this way will prevent
%underfull boxes for text-only pages.
假设 \cs{parskip} 为零,对于一个只含有文字的页面来说,页面盒子的高度等于 \cs{topskip} 加上 \cs{baselineskip} 的倍数。
因此我们可以定义一个宏,通过页面的行数来计算 \cs{vsize}:
\howto Specify page height in lines\par
\begin{verbatim}
\def\HeightInLines#1{\count@=#1\relax
    \advance\count@ by -1\relax
    \vsize=\baselineskip
    \multiply\vsize by \count@
    \advance\vsize by \topskip}
\end{verbatim}
对于纯文本页面,像这样计算 \cs{vsize} 可以避免未充满的盒子。

%In cases where the page does not start with a line of text
%(for instance a rule), the topskip may give unwanted effects.
%To prevent these, start the page with
%\begin{verbatim}
%\hbox{}\kern-\topskip
%\end{verbatim}
%followed by what you wanted on top. 
当页面的首行不是文字时(例如有奇怪的规定要求这样做),topskip可能造成非预期的影响。为了避免这个问题,我们可以以
\begin{verbatim}
\hbox{}\kern-\topskip
\end{verbatim}
作为一页的开头,后面再写你想在页面顶部出现的东西。

%Analogous to the \cs{topskip}, there is a \cs{splittopskip}
%for pages generated by a \cs{vsplit} operation; see
%the next chapter.
类似 \cs{topskip},还有一个 \cs{splittopskip},它适用于由 \cs{vsplit} 操作产生的页面,详见下一章。

%%\point Page height and depth
%\section{Page height and depth}
%\point Page height and depth
\section{页面高度与深度}

%\index{page!height|(}
%\index{page!depth|(}
\index{page!height|(}
\index{page!depth|(}

%\TeX\ tries to build pages as a \cs{vbox} of height \csidx{vsize};
%\alt
%see also \cs{pagegoal} in the next chapter.
\TeX\ 会尝试将页面构建为一个高 \csidx{vsize}的 \cs{vbox} ;参见下下一章的 \cs{pagegoal}。

%If the last item on a page has an excessive depth,
%that page would be noticeably longer than other pages.
%To prevent this phenomenon \TeX\ uses \csidx{maxdepth} as
%the maximum depth of the page box. If adding an item to the
%page would make the depth exceed this quantity, then the
%reference point of the page is moved down to make the depth
%exactly \cs{maxdepth}.
如果某页的最后一项深度过长,这一页会明显比其他页更长。
为了避免这种现象 \TeX\ 使用 \csidx{maxdepth} 限制页面盒子的最大深度。
如果继续添加一项会导致页面深度超限,这时参考点会下移,直至页面深度恰好等于 \cs{maxdepth}。

%The `raggedbottom' effect is obtained in plain \TeX\
%\cstoidx raggedbottom\par
%by giving the \cs{topskip} some finite stretchability:
%\hbox{\n{10pt plus 60pt}}.
%Thus the natural height of box~255 can vary when it reaches
%the output routine.
%Pages are then shipped out (more or less) as
%\begin{verbatim}
%\dimen0=\dp255 \unvbox255
%\ifraggedbottom \kern-\dimen0 \vfil \fi 
%\end{verbatim}
%The \cs{vfil} causes the topskip to be set at natural
%width, so the effect is one of a fixed top line  and a
%variable bottom line of the page.
通过赋给 \cs{topskip} 无限的弹性:\hbox{\n{10pt plus 60pt}},我们可以在plain \TeX\ 中观察到“辣鸡页底”的效应。
因此当 box~255 到达输出程序时,它的自然高度通常有变化。
随后,页面以差不多如下形式送出:
\begin{verbatim}
\dimen0=\dp255 \unvbox255
\ifraggedbottom \kern-\dimen0 \vfil \fi 
\end{verbatim}
\cs{vfil} 使 \cs{topskip} 设置为自然宽度,这样做会产生以下两种效果之一:页面有固定顶部线或可变底部线。

%Before \cs{box255} is unboxed in the plain \TeX\ output routine,
%\cs{boxmaxdepth} is set to \cs{maxdepth}
%so that this box will made under the same assumptions
%that the page builder used when putting together \cs{box255}.
在 \cs{box255} 于 plain \TeX\ 的输出程序中被解包之前,\cs{boxmaxdepth} 被设成 \cs{maxdepth},这样创建这个盒子时,页面生成器使用的假设会和把 \cs{box255} 放在一起时相同。

%The depth of box split off by a \cs{vsplit} operation
%is controlled by the \cs{splitmaxdepth} parameter.
\cs{vsplit} 操作分割盒子的深度由 \cs{splitmaxdepth} 参数控制。

%\index{page!height|)}
%\index{page!depth|)}
\index{page!height|)}
\index{page!depth|)}

%\endofchapter
\endofchapter

\end{document}
